\chapter{Introduction to Significance Tests} 
\label{chap:sigtests} 

Suppose (just for fun, but with the same pattern as in more serious
examples) you have a coin that will be flipped at the Super Bowl to see
who gets the first kickoff.  (We'll assume slightly different rules
here.  The coin is not ``called.''  Instead, it is agreed beforehand
that if the coin comes up heads, Team A will get the kickoff, and
otherwise it will be Team B.) You want to assess for ``fairness.''  Let
p be the probability of heads for the coin.  

You could toss the coin, say, 100 times, and then form a confidence
interval for p using (\ref{propci}).  The width of the interval would
tell you the margin of error, i.e. it tells you whether 100 tosses were
enough for the accuracy you want, and the location of the interval would
tell you whether the coin is ``fair'' enough.

For instance, if your interval were (0.49,0.54), you might feel
satisfied that this coin is reasonably fair.  In fact, {\bf note
carefully that even if the interval were, say, (0.502,0.506), you would
still consider the coin to be reasonably fair}; the fact that the
interval did not contain 0.5 is irrelevant, as the entire interval would
be reasonably near 0.5.

However, this process would not be the way it's traditionally done.
Most users of statistics would use the toss data to test the {\bf null
hypothesis} 

\begin{equation}
H_0:  p = 0.5
\end{equation}

against the {\bf alternate hypothesis} 

\begin{equation}
H_A:  p \neq 0.5
\end{equation}

For reasons that will be explained below, this procedure is called {\bf
significance testing}.  It forms the very core of statistical inference
as practiced today.  This, however, is unfortunate, as there are some
serious problems that have been recognized with this procedure.  We will
first discuss the mechanics of the procedure, and then look closely at
the problems with it in Section \ref{whatswrong}.

\section{The Basics}
\label{sigtests}

Here's how significance testing works.

The approach is to consider $H_0$ ``innocent until proven guilty,''
meaning that we assume $H_0$ is true unless the data give strong
evidence to the contrary.  {\bf KEEP THIS IN MIND!}---we are continually
asking, ``What if...?''

The basic plan of attack is this:

\begin{quote}

We will toss the coin n times.  Then we will believe that the coin is
fair unless the number of heads is ``suspiciously'' extreme, i.e. much
less than n/2 or much more than n/2.

\end{quote}

Let p denote the true probability of heads for our coin.  As in Section
\ref{derivation}, let $\widehat{p}$ denote the proportion of heads in
our sample of n tosses.  We observed in that section that $\widehat{p}$
is a special case of a sample mean (it's a mean of 1s and 0s).  We also
found that the standard deviation of $\widehat{p}$ is $\sqrt{p (1 -
p)/n}$, so that the standard error (Section \ref{notionofse}) is 
$\sqrt{\widehat{p} (1-\widehat{p}) / n }$.

In other words, the material surrounding (\ref{thetahatstderr})
tells us that 

\begin{equation}
\label{tmpfraction}
\frac{\widehat{p}-p}
{\sqrt{
\frac{1}{n} 
\cdot \widehat{p} (1-\widehat{p})
}
} 
\end{equation}

has an approximate N(0,1) distribution.

But remember, we are going to assume $H_0$ for now, until and unless we
find strong evidence to the contrary.  So, in the numerator of
(\ref{tmpfraction}), p = 0.5.  And in the denominator, why use the
standard error when we ``know'' (under our provisional assumption) that
p = 0.5?  The exact standard deviation of $\widehat{p}$ is $\sqrt{p
(1-p)/ n}$. So, replace (\ref{tmpfraction}) by

\begin{equation}
\label{pz}
Z = \frac{\widehat{p}-0.5}{\sqrt{\frac{1}{n} \cdot 0.5(1-0.5)}} 
\end{equation}

then Z has an approximate N(0,1) distribution under the assumption that
$H_0$ is true.

Now recall from the derivation of (\ref{meanci}) that -1.96 and 1.96 are
the lower- and upper-2.5\% points of the N(0,1) distribution.  Thus,

\begin{equation}
P(Z < -1.96 \textrm{ or } Z > 1.96) \approx 0.05
\end{equation}

Now here is the point:  After we collect our data, in this case by
tossing the coin n times, we compute $\widehat{p}$ from that data,
and then compute Z from (\ref{pz}).  If Z is smaller than -1.96 or
larger than 1.96, we reason as follows:

\begin{quote}
Hmmm, Z would stray that far from 0 only 5\% of the time.  So, either I
have to believe that a rare event has occurred, or I must abandon my
assumption that $H_0$ is true.
\end{quote}

For instance, say n = 100 and we get 62 heads in our sample.  That gives
us Z = 2.4, in that ``rare'' range.  We then {\bf reject} $H_0$, and
announce to the world that this is an unfair coin.  We say, ``The value
of p is significantly different from 0.5.'' 

The 5\% ``suspicion criterion'' used above is called the {\bf
significance level}, typically denoted $\alpha$.  One common statement
is ``We rejected $H_0$ at the 5\% level.''

On the other hand, suppose we get 47 heads in our sample.  Then Z =
-0.60.  Again, taking 5\% as our significance level, this value of Z
would not be deemed suspicious, as it occurs frequently.  We would then
say ``We accept $H_0$ at the 5\% level,'' or ``We find that p is not
significantly different from 0.5.''

The word {\it significant} is misleading.  It should NOT be confused
with {\it important}.  It simply is saying we don't believe the observed
value of Z is a rare event, which it would be under $H_0$; we have
instead decided to abandon our believe that $H_0$ is true.

Note by the way that Z values of -1.96 and 1.96 correspond getting $50 -
1.96 \cdot 0.5 \cdot \sqrt{100}$ or $50 + 1.96 \cdot 0.5 \cdot
\sqrt{100}$ heads, i.e. roughly 40 or 60.  In other words, we can
describe our rejection rule to be ``Reject if we get fewer than 40 or
more than 60 heads, out of our 100 tosses.''

\section{General Testing Based on Normally Distributed Estimators}
\label{genhyp}

In Section \ref{stderrest}, we developed a method of constructing
confidence intervals for general approximately normally distributed
estimators.  Now we do the same for significance testing.

Suppose $\widehat{\theta}$ is an approximately normally distributed
estimator of some population value $\theta$.  Then to test $H_0: \theta
= c$, form the test statistic

\begin{equation}
\label{genz}
Z = \frac{\widehat{\theta} - c}{s.e.(\widehat{\theta})}
\end{equation}

where $s.e.(\widehat{\theta})$ is the standard error of
$\widehat{\theta}$,\footnote{See Section \ref{stderrest}.  Or, if we
know the exact standard deviation of $\widehat{\theta}$ under $H_0$,
which was the case in our coin example above, we could use that, for a
better normal approximation.} and proceed as before:  

\begin{quote}

Reject $H_0:  \theta = c$ at the significance level of  $\alpha=0.05$ if
$|Z| \geq 1.96$.

\end{quote}

\section{Example: Network Security}

Let's look at the network security example in Section \ref{netsecapp}
again.  Here $\widehat{\theta} = \overline{X} - \overline{Y}$, and c is
presumably 0 (depending on the goals of Mano {\it et al}).  From
\ref{sediff}, the standard error works out to 0.61.  So, our test
statistic (\ref{genz}) is

\begin{equation}
Z = \frac{\overline{X} - \overline{Y} - 0}{0.61} = \frac{11.52-2.00}{0.61} = 15.61
\end{equation}

This is definitely larger in absolute value than 1.96, so we reject
$H_0$, and conclude that the population mean round-trip times are
different in the wired and wireless cases.

\section{The Notion of ``p-Values''}

Recall the coin example in Section \ref{sigtests}, in which we got 62
heads, i.e. Z = 2.4.  Since 2.4 is considerably larger than 1.96, our
cutoff for rejection, we might say that in some sense we not only
rejected $H_0$, we actually \underline{strongly} rejected it.

To quantify that notion, we compute something called the {\bf observed
significance level}, more often called the {\bf p-value}. 

We ask,

\begin{quote}
We rejected $H_0$ at the 5\% level.  Clearly, we would have rejected it
even at some small---thus more stringent---levels.  What is the smallest
such level?  Call this the p-value of the test.
\end{quote}

By checking a table of the N(0,1) distribution, or by calling {\bf
pnorm(2.40)} in R, we would find that the N(0,1) distribution has area
0.008 to the right of 2.40, and of course by symmetry there is an equal
area to the left of -2.40.  That's a total area of 0.016.  In other
words, we would have been able to reject $H_0$ even at the much more
stringent significance level of 0.016 (the 1.6\% level) instead of 0.05.
So, Z = 2.40 would be considered even more significant than Z = 1.96.
In the research community it is customary to say, ``The p-value was
0.016.'' \footnote{The `p' in ``p-value'' of course stands for
``probability,'' meaning the probably that a N(0,1) random variable
would stray as far, or further, from 0 as our observed Z here.  By the
way, be careful not to confuse this with the quantity p in our coin
example, the probability of heads.}  The smaller the p-value, the more
significant the results are considered.  
 
% \checkpoint

In our network security example above in which Z was 15.61, the value is
literally ``off the chart''; {\bf pnorm(15.61)} returns a value of 1.
Of course, it's a tiny bit less than 1, but it is so far out in the
right tail of the N(0,1) distribution that the area to the right is
essentially 0.  So the p-value would be essentially 0, and the result
would be treated as very, very highly significant.

In computer output or research reports, we often see small p-values
being denoted by asterisks.  There is generally one asterisk for p under
0.05, two for p less than 0.01, three for 0.001, etc.  The more
asterisks, the more significant the data is supposed to be.  See for
instance the R regression output on page \pageref{asterisksexample}.

% \section{What's Random and What Is Not}
% 
% It is crucial to keep in mind that $H_0$ is not an event or any other
% kind of random entity.  This coin either has p = 0.5 or it doesn't.  If
% we repeat the experiment, we will get a different value of X, but p
% doesn't change; it's still the same coin!  So for example, it would be
% wrong and meaningless to speak of the ``probability that $H_0$ is
% true.''  
% 
% Similarly, it would be wrong and meaningless to write $0.05 = P(|Z| >
% 1.96 | H_0)$, again because $H_0$ is not an event and this kind of
% conditional probability would not make sense.  What is customarily
% written is something like
% 
% \begin{equation}
% \label{z196}
% 0.05 = P_{H_0}(|Z| > 1.96)
% \end{equation}
% 
% This is read aloud as ``the probability that $|Z|$ is larger than 1.96
% under $H_0$,'' with the phrase {\it under $H_0$} referring to the
% probability measure if $H_0$ is true.  Again, it's all an exercise in
% ``What if?'' thinking.\footnote{Nothing wrong with this!  The problems
% with significance testing, discussed in Section \ref{whatswrong} below,
% do not stem from ``What if?'' thinking.}
% 

\section{Example:  Bank Data}

Consider again the bank marketing data in Section \ref{ucibank}.
Our comparison was between marketing campaign success rates for married
and unmarried customers.  The p-value was quite tiny, $2.2 \times
10^{-16}$, but be careful interpreting this.

First, don't take that p-value as exact by any means.  Though our sample
sizes are certainly large enough for the Central Limit Theorem to work
well, that is in the heart of the distribution, not the far tails.  So,
just take the p-value as ``tiny,'' and leave it at that.

Second, although the standard description for a test with such a small
p-value is ``very highly significant,'' keep in mind that the difference
between the two groups was not that large.  The confidence interval we
are 95\% confident that the population success rate is between 3.3\% and
4.6\% less for married people.  That is an interesting difference and
possibly of some use to the marketing people, but it is NOT large.

\section{One-Sided $H_A$}

Suppose that---somehow---we are sure that our coin in the example above
is either fair or it is more heavily weighted towards heads.  Then we
would take our alternate hypothesis to be

\begin{equation}
\label{onesideha}
H_A: p > 0.5
\end{equation}

A ``rare event'' which could make us abandon our belief in $H_0$ would
now be if Z in (\ref{pz}) is very large in the positive direction.  So,
with $\alpha = 0.05$, we call {\bf qnorm(0.95)}, and find that our rule
would now be to reject $H_0$ if $Z > 1.65$.

One-sided tests are not common, as their assumptions are often difficult
to justify.

% The same would be the case if our null hypothesis were
% 
% \begin{equation}
% H_A: p \leq 0.5
% \end{equation}
% 
% instead of 
% 
% \begin{equation}
% H_A: p = 0.5
% \end{equation}
% 
% Then (\ref{z196}) would change to
% 
% \begin{equation}
% \label{z196new}
% 0.05 \geq P_{H_0}(|Z| > 1.65)
% \end{equation}

\section{Exact Tests}

Remember, the tests we've seen so far are all approximate.  In
(\ref{pz}), for instance, $\widehat{p}$ had an approximate normal
distribution, so that the distribution of Z was approximately N(0,1).
Thus the significance level $\alpha$ was approximate, as were the
p-values and so on.\footnote{Another class of probabilities which would
be approximate would be the {\bf power} values.  These are the
probabilities of rejecting $H_0$ if the latter is not true.  We would
speak, for instance, of the power of our test at p = 0.55, meaning the
chances that we would reject the null hypothesis if the true population
value of p were 0.55.}

But the only reason our tests were approximate is that we only had the
{\it approximate} distribution of our test statistic Z, or equivalently,
we only had the approximate distribution of our estimator, e.g.
$\widehat{p}$.  If we have an {\it exact} distribution to work with,
then we can perform an exact test.

\subsection{Example: Test for Biased Coin}

Let's consider the coin example again, with the one-sided alternative
(\ref{onesideha}) .  To keep things simple, let's suppose we toss the
coin 10 times.  We will make our decision based on X, the number of
heads out of 10 tosses.  Suppose we set our threshhold for ``strong
evidence'' again $H_0$ to be 8 heads, i.e. we will reject $H_0$ if $X
\geq 8$.  What will $\alpha$ be?

\begin{equation}
\alpha = \sum_{i=8}^{10} P(X = i) = 
\sum_{i=8}^{10} 
 \left (
\begin{array}{c}
10 \\
i
\end{array}
\right )
\left ( \frac{1}{2} \right )^{10} = 0.055
\end{equation}

That's not the usual 0.05.  Clearly we cannot get an exact significance
level of 0.05,\footnote{Actually, it could be done by introducing some
randomization to our test.} but our $\alpha$ is exactly 0.055, so this
is an exact test.

So, we will believe that this coin is perfectly balanced, unless we get
eight or more heads in our 10 tosses.  The latter event would be very
unlikely (probability only 5.5\%) if $H_0$ were true, so we decide not
to believe that $H_0$ is true.

\subsection{Example:  Improved Light Bulbs}
\label{bulbexact}

Suppose lifetimes of lightbulbs are exponentially distributed with mean
$\mu$. In the past, $\mu = 1000$, but there is a claim that the new
light bulbs are improved and $\mu > 1000$. To test that claim, we will
sample 10 lightbulbs, getting lifetimes $X_1,...,X_{10}$, and compute
the sample mean $\overline{X}$.  We will then perform a significance
test of 

\begin{equation}
H_0: \mu = 1000
\end{equation}

vs. 

\begin{equation}
H_A: \mu > 1000
\end{equation}

It is natural to have our test take the form in which we reject
$H_0$ if 

\begin{equation}
\overline{X} > w
\end{equation}

for some constant w chosen so that

\begin{equation}
\label{exactbulb}
P(\overline{X} > w) = 0.05
\end{equation}

under $H_0$.  Suppose we want an exact test, not one based on a normal
approximation. 

Remember, we are making our calculations under the assumption that $H_0$
is true.  Now recall (Section \ref{gammaprops}) that $10 \overline{X}$,
the sum of the $X_i$, has a gamma distribution, with r = 10 and $\lambda
= 0.001$.  So, we can find the w for which $P(\overline{X} > w) = 0.05$
by using R's {\bf qgamma()}:

\begin{Verbatim}[fontsize=\relsize{-2}]
> qgamma(0.95,10,0.001)
[1] 15705.22
\end{Verbatim}

So, we reject $H_0$ if our sample mean is larger than 1570.5.

Now suppose it turns out that $\overline{X} = 1624.2$.  Under $H_0$
there was only a 0.05 chance that $\overline{X}$ would exceed 1570.5, so
we would reject $H_0$ with $\alpha = 0.05$.  But what if we had set w to
1624.2?  We didn't do so, of course, but what if?  The computation 

\begin{lstlisting}
> 1 - pgamma(16242,10,0.001)
[1] 0.03840629
\end{lstlisting}

shows that we would have rejected $H_0$ even if we had originally set
$\alpha$ to the more stringent criterion of 0.038 instead of 0.05.  So
we report that the p-value was 0.038.

The idea of a p-value is to indicate in our report ``how strongly'' we
rejected $H_0$.  Arguably there is a bit of game-playing in p-values, as
there is with significance testing in general.  This will be pursued in
Section \ref{whatswrong}.

\subsection{Example:  Test Based on Range Data}

Suppose lifetimes of some electronic component formerly had an
exponential distribution with mean 100.0.  However, it's claimed that
now the mean has increased.  (Suppose we are somehow sure it has not
decreased.)  Someone has tested 50 of these new components, and has
recorded their lifetimes, $X_1,...,X_{50}$.  Unfortunately, they only
reported to us the range of the data, $R = \max_i X_i - \min_i X_i$, not
the individual $X_i$.  We will need to do a significance test with this
limited data, at the 0.05 level.  

Recall that the variance of an exponential random variable is the square
of its mean.  Intutively, then, the larger this population mean of X,
the larger the mean of the range R.  In other words, the form of the
test should be to reject $H_0$ if R is greater than some cutoff value c.
So, we need to find the value of c to make $\alpha$ equal to 0.05.

Unfortunately, we can't do this analytically, i.e. mathematically, as
the distribution of R is far too complex.  This we'll have to resort to
simulation.\footnote{I am still referring to the following as an exact
test, as we are not using any statistical approximation, such as the
Central Limit Theorm.}  Here is code to do that:

\begin{lstlisting}[numbers=left]
# code to determine the cutoff point for significance 
# at 0.05 level

nreps <- 200000
n <- 50

rvec <- vector(length=nreps)
for (i in 1:nreps) {
   x <- rexp(n,0.01)
   rng <- range(x)
   rvec[i] <- rng[2] - rng[1]
}

rvec <- sort(rvec)
cutoff <- rvec[ceiling(0.95*nreps)]
cat("reject H0 if R >",rvec[cutoff],"\n")
\end{lstlisting}

Here we general {\bf nreps} samples of size 50 from an exponential
distribution having mean 100.  Note that since we are setting $\alpha$,
a probability defined in the setting in which $H_0$ is true, we assume
the mean is 100.  For each of the {\bf nreps} samples we find the value
of R, recording it in {\bf rvec}.  We then take the 95$^{th}$ percentile
of those values, which is the c for which P(R $>$ c) = 0.05.\footnote{Of
course, this is approximate.  The greater the value of {\bf nreps}, the
better the approximation.}

The value of c output by the code was 220.4991.  A second run yielded,
220.9304, and a third 220.7099.  The fact that these values varied little
among themselves indicates that our value of {\bf nreps}, 200000, was
sufficiently large.

\subsection{Exact Tests under a Normal Distribution Assumption}

If you are willing to assume that you are sampling from a
normally-distributed population, then the Student-t test is nominally
exact.  The R function {\bf t.test()} performs this operation, with the
argument {\bf alternative} set to be either {\bf "less"} or {\bf
"greater"}.

\section{Don't Speak of ``the Probability That $H_0$ Is True''}

It is very important to understand that throughout this chapter, we
cannot speak of ``the probability that $H_0$ is true,'' because we have
no probablistic structure on $H_0$.

Consider the fair-coin problem at the beginning of this chapter.
Suppose we hope to make a statement like, say, ``Given that we got 62
heads out of 100 tosses, we find that the probability that this is a
fair coin is 0.04.''  What kind of derivation would need to go into
this?  It would go along the following lines:

\begin{eqnarray}
P(H_0 \textrm{ is true } ~|~ \textrm{ our data}) &=& 
\frac
{P(H_0 \textrm{ is true and our data})} 
{P(\textrm{our data})} \\ 
&=& 
\nonumber
\frac
{P(H_0 \textrm{ is true and our data})} 
{P(H_0 \textrm{ is true and our data}) + P(H_0 \textrm{ is false and our data})}
\\
&=& 
\nonumber
\frac
{P(H_0 \textrm{ true}) ~ P(\textrm{our data} ~|~ H_0 \textrm{ true})} 
{P(H_0 \textrm{ true}) ~ P(\textrm{our data }  ~|~ H_0 \textrm{ true} ) +
P(H_0 \textrm{ false}) ~ P(\textrm{our data}  ~|~ H_0 \textrm{ false} )} 
\end{eqnarray}

Through our modeling process, e.g. the discussion surrounding
(\ref{pz}), we can calculate $P(\textrm{our data} ~|~ H_0 \textrm{ is
true})$.  (The false case would be more complicated, since there are
many different kinds of false cases here, for different values of p, but
could be handled similarly.) But what we don't have is $P(H_0 \textrm{
is true})$.

We could certainly try to model that latter quantity, say by taking a
sample of all possible pennies (if our coin is a penny), doing very
extensive testing of them;\footnote{Say, 100,000 tosses per coin.} the
proportion found to be fair would then be $P(H_0 \textrm{ is true})$.
But lacking that, we have no probablistic structure for $P(H_0 \textrm{
is true})$, and thus cannot use language like ``the probability that
$H_0$ is true,'' 

\section{R Computation}

The R function {\bf t.test()}, discussed in Section \ref{rcomp}, does
both confidence intervals and tests, including p-values in the latter
case.

\section{The Power of a Test} 

In addition to the significance level of a test, we may also be
interested in its {\bf power} (or its many power values, as will be
seen).

\subsection{Example: Coin Fairness}

For example, consider our first example in this chapter, in which we
were testing a coin for fairness (Section \ref{sigtests}).  Our rule for
a test at a 0.05 significance level turned out to be that we reject
$H_0$ if we get fewer than 40 or more than 60 heads out of our 100
tosses.  We might ask the question, say:

\begin{quote}
Suppose the true heads probability is 0.65.  We don't know, of course,
but what if that were the case.  That's a pretty substantial departure
from $H_0$, so hopefully we would reject.  Well, what is the probability
that we would indeed reject?
\end{quote}

We could calculate this.  Let N denote the number of heads.  Then the
desired probability is $P(N < 40 \textrm{ or } N > 60) =  P(N < 40) +
P(N > 60)$.  Let's find the latter.\footnote{The former would be found
similarly, but would come out quite small.}

Once again, since N has a binomial distribution, it is approximately
normal, in this case with mean $np = 100 \times 0.65 = 65$ and variance
$np(1-p) = 100 \times 0.65 \times 0.35 = 22.75$.  Then $P(N > 60)$ is about

\begin{lstlisting}
1 - pnorm(60,65,sqrt(22.75))
\end{lstlisting}

% \begin{equation}
% P(N > 60) = 
% P \left (Z > \frac{60 - 100 \cdot 0.65}{\sqrt{100 \cdot 0.65 \cdot
% 0.35}} \right )
% \end{equation}
% 
% where now Z is 
% 
% \begin{equation}
% \frac{N - 100 \cdot 0.65}{\sqrt{100 \cdot 0.65 \cdot 0.35}}
% \end{equation}

or about 0.85.  So we would be quite likely to decide this is an unfair
coin if (unknown to us) the true probability of heads is 0.65.

We say that the power of this test {\it at} p = 0.65 is 0.85.  There
is a different power for each p.

\subsection{Example:  Improved Light Bulbs}

Let's find the power of the test in Section \ref{bulbexact}, at
$\mu = 1250$.  Recall that we reject $H_0$ if $\overline{X} > 1570.522$.
Thus our power is

\begin{lstlisting}
1 - pgamma(15705.22,10,1/1250)
\end{lstlisting}

This turns out to be about 0.197.  So, if (remember, this is just a
``what if?'') the true new mean were 1250, we'd only have about a 20\%
chance of discovering that the new bulbs are improved.

\section{What's Wrong with Significance Testing---and What to Do Instead}
\label{whatswrong}

{\it 
The first principle is that you must not fool yourself---and you are
the easiest person to fool. So you have to be very careful about
that. After you've not fooled yourself, it's easy not to fool other
scientists.}---Richard Feynman, Nobel laureate in physics

{\it ``Sir Ronald [Fisher] has befuddled us, mesmerized us, and led us
down the primrose path''}---Paul Meehl, professor of psychology and the
philosophy of science

\bigskip

{\bf Significance testing is a time-honored approach, used by tens of thousands
of people every day.}  But it is ``wrong.'' I use the quotation marks
here because, although significance testing is mathematically correct, it
is at best noninformative and at worst seriously misleading.  

\subsection{History of Significance Testing, and Where We Are Today} 

We'll see why significance testing has serious problems shortly, but
first a bit of history.  

When the concept of significance testing, especially the 5\% value for
$\alpha$, was developed in the 1920s by Sir Ronald Fisher, many
prominent statisticians opposed the idea---for good reason, as we'll see
below.  But Fisher was so influential that he prevailed, and thus
significance testing became the core operation of statistics.

So, significance testing became entrenched in the field, in spite of
being widely recognized as faulty, to this day.  Most modern
statisticians understand this, even if many continue to engage in the
practice.\footnote{Many are forced to do so, e.g. to comply with
government standards in pharmaceutical testing.  My own approach in such
situations is to quote the test results but then point out the problems,
and present confidence intervals as well.}  Here are a few places you
can read criticism of testing:

\begin{itemize}

\item There is an entire book on the subject, {\it The Cult of
Statistical Significance}, by S. Ziliak and D. McCloskey.
Interestingly, on page 2, they note the prominent people who have
criticized testing.  Their list is a virtual ``who's who'' of
statistics, as well as physics Nobel laureate Richard Feynman and
economics Nobelists Kenneth Arrow and Milton Friedman. 

\item See \url{http://www.indiana.edu/~stigtsts/quotsagn.html} for a
nice collection of quotes from famous statisticians on this point.

\item There is an entire chapter devoted to this issue in one of the
best-selling elementary statistics textbooks in the
nation.\footnote{{\it Statistics}, third edition, by David Freedman,
Robert Pisani, Roger Purves, pub. by W.W.  Norton, 1997.} 

\item The Federal Judicial Center, which is the educational and research
arm of the federal court system, commissioned two prominent academics,
one a statistics professor and the other a law professor, to write a
guide to statistics for judges:  {\it Reference Guide on Statistics}.
David H.  Kaye. David A. Freedman, at
\begin{Verbatim}[fontsize=\relsize{-2}]
http://www.fjc.gov/public/pdf.nsf/lookup/sciman02.pdf/$file/sciman02.pdf
\end{Verbatim}

There is quite a bit here on the problems of significance testing, and
especially p.129.

\end{itemize}

\subsection{The Basic Fallacy}

To begin with, {\bf it's absurd to test $H_0$ in the first place},
because we know {\it a priori} that $H_0$ is false.  

Consider the coin example, for instance.  No coin is absolutely
perfectly balanced, and yet that is the question that significance
testing is asking:

\begin{equation}
H_0: p = 0.5000000000000000000000000000...  
\end{equation}

We know before even collecting any data that the hypothesis we are
testing is false, and thus it's nonsense to test it.

{\bf But much worse is this word ``significant.''}  Say our coin actually
has p = 0.502.  From anyone's point of view, that's a fair coin!  But
look what happens in (\ref{pz}) as the sample size n grows.  If we have
a large enough sample, eventually the denominator in (\ref{pz}) will be
small enough, and $\widehat{p}$ will be close enough to 0.502, that Z will
be larger than 1.96 and we will declare that p is ``significantly''
different from 0.5.  But it isn't!  Yes, 0.502 is different from 0.5, but
NOT in any significant sense in terms of our deciding whether to use
this coin in the Super Bowl.

The same is true for government testing of new pharmaceuticals.  We
might be comparing a new drug to an old drug.  Suppose the new drug
works only, say, 0.4\% (i.e. 0.004) better than the old one.  Do we want
to say that the new one is ``signficantly'' better?  This wouldn't be
right, especially if the new drug has much worse side effects and costs
a lot more (a given, for a new drug).

Note that in our analysis above, in which we considered what would
happen in (\ref{pz}) as the sample size increases, we found that
eventually {\it everything} becomes ``signficiant''---even if there is no
practical difference.  This is especially a problem in computer science
applications of statistics, because they often use very large data sets.
A data mining application, for instance, may consist of hundreds of
thousands of retail purchases.  The same is true for data on visits to a
Web site, network traffic data and so on.  In all of these, the standard
use of significance testing can result in our pouncing on very small
differences that are quite insignificant to us, yet will be declared
``significant'' by the test.

% \checkpoint

Conversely, if our sample is too small, we can miss a difference that
actually {\it is} significant---i.e. important to us---and we would
declare that p is NOT significantly different from 0.5.  In the example
of the new drug, this would mean that it would be declared as ``not
significantly better'' than the old drug, even if the new one is much
better but our sample size wasn't large enough to show it.

In summary, the basic problems with significance testing are

\begin{itemize}

\item $H_0$ is improperly specified.  What we are really interested in
here is whether p is {\it near} 0.5, not whether it is {\it exactly} 0.5
(which we know is not the case anyway).

\item Use of the word {\it significant} is grossly improper (or, if you
wish, grossly misinterpreted).

\end{itemize}

Significance testing forms the very core usage of statistics, yet you
can now see that it is, as I said above, ``at best noninformative and at
worst seriously misleading.'' This is widely recognized by thinking
statisticians and prominent scientists, as noted above.  But the
practice of significance testing is too deeply entrenched for things to
have any prospect of changing.

\subsection{You Be the Judge!}

This book has been written from the point of view that every educated
person should understand statistics.  It impacts many vital aspects of
our daily lives, and many people with technical degrees find a need for
it at some point in their careers.  

In other words, statistics is something to be {\it used}, not just
learned for a course.  You should think about it critically, especially
this material here on the problems of significance testing.  You
yourself should decide whether the latter's widespread usage is
justified.

\subsection{What to Do Instead}

Note carefully that I am not saying that we should not make a decision.
We {\it do} have to decide, e.g. decide whether a new hypertension drug
is safe or in this case decide whether this coin is ``fair'' enough for
practical purposes, say for determining which team gets the kickoff in
the Super Bowl.  But it should be an informed decision, and even testing
the modified $H_0$ above would be much less informative than a
confidence interval.

In fact, the real problem with significance tests is that they {\bf take
the decision out of our hands}.  They make our decision mechanically for
us, not allowing us to interject issues of importance to us, such
possible side effects in the drug case.

So, what can we do instead?

In the coin example, we could set limits of fairness, say require that p
be no more than 0.01 from 0.5 in order to consider it fair.  We could
then test the hypothesis

\begin{equation}
H_0:  0.49 \leq p \leq 0.51
\end{equation}

Such an approach is almost never used in practice, as it is somewhat
difficult to use and explain.  But even more importantly, what if the
true value of p were, say, 0.51001?  Would we still really want to
reject the coin in such a scenario?

{\bf Forming a confidence interval is the far superior approach.}  The width
of the interval shows us whether n is large enough for $\widehat{p}$ to be
reasonably accurate, and the location of the interval tells us whether
the coin is fair enough for our purposes.

{\bf Note that in making such a decision, we do NOT simply check whether
0.5 is in the interval.}  That would make the confidence interval reduce
to a significance test, which is what we are trying to avoid.  If for
example the interval is (0.502,0.505), we would probably be quite
satisfied that the coin is fair enough for our purposes, even though 0.5
is not in the interval.

On the other hand, say the interval comparing the new drug to the old
one is quite wide and more or less equal positive and negative
territory.  Then the interval is telling us that the sample size just
isn't large enough to say much at all.

Significance testing is also used for model building, such as for
predictor variable selection in regression analysis (a method to be
covered in Chapter \ref{chap:linreg}).  The problem is even worse there,
because there is no reason to use $\alpha = 0.05$ as the cutoff point
for selecting a variable.  In fact, even if one uses significance
testing for this purpose---again, very questionable---some studies have
found that the best values of $\alpha$ for this kind of application are
in the range 0.25 to 0.40, far outside the range people use in testing.

In model building, we still can and should use confidence intervals.
However, it does take more work to do so.  We will return to this point
in our unit on modeling, Chapter \ref{chap:mod}.

\subsection{Decide on the Basis of ``the Preponderance of Evidence''}
\label{preponderance}

{\it I was in search of a one-armed economist, so that the guy could
never make a statement and then say: ``on the other hand''}---President
Harry S Truman

{\it If all economists were laid end to end, they would not reach a
conclusion}---Irish writer George Bernard Shaw

\bigskip

In the movies, you see stories of murder trials in which the accused
must be ``proven guilty beyond the shadow of a doubt.''  But in most
noncriminal trials, the standard of proof is considerably lighter, {\bf
preponderance of evidence}.  This is the standard you must use when
making decisions based on statistical data.  Such data cannot ``prove''
anything in a mathematical sense.  Instead, it should be taken merely as
evidence.  The width of the confidence interval tells us the likely
accuracy of that evidence.  We must then weigh that evidence against
other information we have about the subject being studied, and then
ultimately make a decision on the basis of the preponderance of all the
evidence.

Yes, juries must make a decision.  But they don't base their verdict on
some formula.  Similarly, you the data analyst should not base your
decision on the blind application of a method that is usually of little
relevance to the problem at hand---significance testing.

\subsection{Example:  the Forest Cover Data}

In Section \ref{forestcover}, we found that an approximate 95\%
confidence interval for $\mu_1 - \mu_2$ was

\begin{equation}
223.8 - 226.3 \pm 2.3
= (-4.8,-0.3)
\end{equation}

Clearly, the difference in HS12 between cover types 1 and 2 is tiny when
compared to the general size of HS12, in the 200s.  Thus HS12 is not
going to help us guess which cover type exists at a given location.  Yet
with the same data, we would reject the hypothesis

\begin{equation}
H_0:  \mu_1 = \mu_2
\end{equation}

and say that the two means are ``significantly'' different, which sounds
like there is an important difference---which there is not.

\subsection{Example:  Assessing Your Candidate's Chances for Election}

Imagine an election between Ms. Smith and Mr. Jones, with you serving as
campaign manager for Smith.  You've just gotten the results of a very
small voter poll, and the confidence interval for p, the fraction of
voters who say they'll vote for Smith, is (0.45,0.85).  Most of the
points in this interval are greater than 0.5, so you would be highly
encouraged!  You are certainly not sure of the final election result, as
a small part of the interval is below 0.5, and anyway voters might
change their minds between now and the election.  But the results would
be highly encouraging.

Yet a significance test would say ``There is no significant difference
between the two candidates.  It's a dead heat.''  Clearly that is not
telling the whole story.  The point, once again, is that {\bf the
confidence interval is giving you much more information than is the
significance test.}

% Another way to describe this is that the preponderance of evidence is
% that Smith is winning the election.  This can be formalized by noting
% that the point at the center of the interval has the highest likelihood,
% while the further a point is from the center, the lower its likelihood.
% (I mean this in the sense of Section \ref{mle}, NOT in a Bayesian sense,
% which I disapprove of.)

