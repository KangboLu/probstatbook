\chapter{R Quick Start}
\label{chap:rquickstart}

Here we present a quick introduction to the R data/statistical
programming language.  Further learning resources are listed at
\url{http://heather.cs.ucdavis.edu//r.html}.

R syntax is similar to that of C. It is object-oriented (in the sense of
{\it encapsulation}, {\it polymorphism} and everything being an object)
and is a functional language (i.e. almost no side effects, every action
is a function call, etc.).

\section{Correspondences}

\begin{tabular}{|l|l|l|}
\hline
aspect & C/C++ & R \\ \hline 
\hline
assignment & = & \verb#<-# (or =) \\ \hline 
array terminology & array & vector, matrix, array \\ \hline 
subscripts & start at 0 & start at 1 \\ \hline
array notation & m[2][3] & m[2,3] \\ \hline 
2-D array storage & row-major order & column-major order \\ \hline 
mixed container &struct, members accessed by . & list, members acessed by
\$ or [[ ]] \\ \hline 
return mechanism & return & return() or last value computed \\ \hline
primitive types & int, float, double, char, bool & integer, float,
double, character, logical \\ \hline
logical values & true, false & TRUE, FALSE (abbreviated T, F) \\ \hline
mechanism for combining modules & include, link & library() \\ \hline
run method & batch & interactive, batch \\ \hline
comment symbol & // & \#\\ \hline
\end{tabular}

\section{Starting R}

To invoke R, just type ``R'' into a terminal window, e.g.\ {\bf xterm}
in Linux or Macs, {\bf CMD} in Windows.

If you prefer to run from an IDE, you may wish to consider ESS for
Emacs, StatET for Eclipse or RStudio, all open source.  ESS is the
favorite among the ``hard core coder'' types, while the colorful,
easy-to-use, RStudio is a big general crowd pleaser.  If you are already
an Eclipse user, StatET will be just what you need.\footnote{I personally use
{\bf vim}, as I want to have the same text editor no matter what kind of
work I am doing.  But I have my own macros to help with R work.}

R is normally run in interactive mode, with $>$ as the prompt.  Among 
other things, that makes it easy to try little experiments to learn
from; remember my slogan, ``When in doubt, try it out!''  For batch
work, use {\bf Rscript}, which is in the R package.

\section{First Sample Programming Session}

Below is a commented R session, to introduce the concepts. I had a text
editor open in another window, constantly changing my code, then loading
it via R's {\bf source()} command.  The original contents of the file
{\bf odd.R} were:

\begin{lstlisting}[numbers=left]
oddcount <- function(x)  {
   k <- 0  # assign 0 to k
   for (n in x)  {
      if (n %% 2 == 1) k <- k+1  # %% is the modulo operator
   }
   return(k)
}
\end{lstlisting}

By the way, we could have written that last statement as simply 

\begin{lstlisting}
   k
\end{lstlisting}

because the last computed value of an R function is returned
automatically.  This is actually preferred style in the R community.

The R session is shown below.  You may wish to type it yourself as you
go along, trying little experiments of your own along the
way.\footnote{The source code for this file is at
\url{https://raw.githubusercontent.com/matloff/probstatbook/master/R5MinIntro.tex}
You can download the file, and copy/paste the text from there.}

\begin{lstlisting}[numbers=left]
> source("odd.R")  # load code from the given file
> ls()  # what objects do we have?
[1] "oddcount"
> # what kind of object is oddcount (well, we already know)?
> class(oddcount)
[1] "function"
> # while in interactive mode, and not inside a function, can print 
> # any object by typing its name; otherwise use print(), e.g. print(x+y)
> oddcount  # a function is an object, so can print it
function(x)  {
   k <- 0  # assign 0 to k
   for (n in x)  {
      if (n %% 2 == 1) k <- k+1  # %% is the modulo operator
   }
   return(k)
}

> # let's test oddcount(), but look at some properties of vectors first
> y <- c(5,12,13,8,88)  # c() is the concatenate function
> y
[1]  5 12 13  8 88
> y[2]  # R subscripts begin at 1, not 0
[1] 12
> y[2:4]  # extract elements 2, 3 and 4 of y
[1] 12 13  8
> y[c(1,3:5)]  # elements 1, 3, 4 and 5
[1]  5 13  8 88
> oddcount(y)  # should report 2 odd numbers
[1] 2

> # change code (in the other window) to vectorize the count operation,
> # for much faster execution
> source("odd.R")
> oddcount
function(x)  {
   x1 <- (x %% 2 == 1)  # x1 now a vector of TRUEs and FALSEs
   x2 <- x[x1]  # x2 now has the elements of x that were TRUE in x1
   return(length(x2))
}

> # try it on subset of y, elements 2 through 3
> oddcount(y[2:3])
[1] 1
> # try it on subset of y, elements 2, 4 and 5
> oddcount(y[c(2,4,5)])
[1] 0

> # further compactify the code
> source("odd.R")
> oddcount
function(x)  {
   length(x[x %% 2 == 1])  # last value computed is auto returned
}
> oddcount(y)  # test it
[1] 2

# and even more compactification, making use of the fact that TRUE and
# FALSE are treated as 1 and 0
> oddcount <- function(x) sum(x %% 2 == 1)
# make sure you understand the steps that that involves:  x is a vector,
# and thus x %% 2 is a new vector, the result of applying the mod 2
# operation to every element of x; then x %% 2 == 1 applies the == 1
# operation to each element of that result, yielding a new vector of TRUE
# and FALSE values; sum() then adds them (as 1s and 0s)

# we can also determine which elements are odd
> which(y %% 2 == 1)
[1] 1 3
\end{lstlisting}

Note that, as I like to say, ``the function of the R function {\bf
function()} is to produce functions!'' Thus assignment is used. For
example, here is what {\bf odd.R} looked like at the end of the above
session:

\begin{lstlisting}[numbers=left]
oddcount <- function(x)  {
   x1 <- x[x %% 2 == 1]
   return(list(odds=x1, numodds=length(x1)))
}
\end{lstlisting}
  
We created some code, and then used {\bf function()} to create a function
object, which we assigned to {\bf oddcount}.

\section{Vectorization}

Note that we eventually {\bf vectorized} our function {\bf oddcount()}.
This means taking advantage of the vector-based, functional language
nature of R, exploiting R's built-in functions instead of loops.  This
changes the venue from interpreted R to C level, with a potentially
large increase in speed.  For example:

\begin{lstlisting}[numbers=left]
> x <- runif(1000000)  # 1000000 random numbers from the interval (0,1)
> system.time(sum(x))
   user  system elapsed 
  0.008   0.000   0.006 
> system.time({s <- 0; for (i in 1:1000000) s <- s + x[i]})
   user  system elapsed 
  2.776   0.004   2.859 
\end{lstlisting}

\section{Second Sample Programming Session}

A matrix is a special case of a vector, with added class attributes,
the numbers of rows and columns.

\begin{lstlisting}[numbers=left]
> # "rowbind() function combines rows of matrices; there's a cbind() too
> m1 <- rbind(1:2,c(5,8))
> m1
     [,1] [,2]
[1,]    1    2
[2,]    5    8
> rbind(m1,c(6,-1))
     [,1] [,2]
[1,]    1    2
[2,]    5    8
[3,]    6   -1

> # form matrix from 1,2,3,4,5,6, in 2 rows; R uses column-major storage
> m2 <- matrix(1:6,nrow=2)  
> m2
     [,1] [,2] [,3]
[1,]    1    3    5
[2,]    2    4    6
> ncol(m2)
[1] 3
> nrow(m2)
[1] 2
> m2[2,3]  # extract element in row 2, col 3
[1] 6
# get submatrix of m2, cols 2 and 3, any row
> m3 <- m2[,2:3]
> m3
     [,1] [,2]
[1,]    3    5
[2,]    4    6

# or write to that submatrix
> m2[,2:3] <- cbind(c(5,12),c(8,0))
> m2
     [,1] [,2] [,3]
[1,]    1    5    8
[2,]    2   12    0

> m1 * m3  # elementwise multiplication
     [,1] [,2]
[1,]    3   10
[2,]   20   48
> 2.5 * m3  # scalar multiplication (but see below)
     [,1] [,2]
[1,]  7.5 12.5
[2,] 10.0 15.0
> m1 %*% m3  # linear algebra matrix multiplication
     [,1] [,2]
[1,]   11   17
[2,]   47   73

> # matrices are special cases of vectors, so can treat them as vectors 
> sum(m1)
[1] 16
> ifelse(m2 %%3 == 1,0,m2) # (see below)
     [,1] [,2] [,3]
[1,]    0    3    5
[2,]    2    0    6
\end{lstlisting}

\section{Recycling}

The ``scalar multiplication'' above is not quite what you may think,
even though the result may be.  Here's why:

In R, scalars don't really exist; they are just one-element vectors.
However, R usually uses {\bf recycling}, i.e. replication, to make
vector sizes match.  In the example above in which we evaluated the
express \verb^2.5 * m3^, the number 2.5 was recycled to the matrix 

\begin{equation}
\left (
\begin{array}{rr}
2.5 & 2.5 \\
2.5 & 2.5 
\end{array}
\right )
\end{equation}

in order to conform with {\bf m3} for (elementwise) multiplication.

\section{More on Vectorization}

The {\bf ifelse()} function is another example of vectorization.  Its
call has the form

\begin{lstlisting}
ifelse(boolean vectorexpression1, vectorexpression2, vectorexpression3)
\end{lstlisting}

All three vector expressions must be the same length, though R will
lengthen some via recycling.  The action will be to return a vector of
the same length (and if matrices are involved, then the result also has
the same shape).  Each element of the result will be set to its
corresponding element in {\bf vectorexpression2} or {\bf
vectorexpression3}, depending on whether the corresponding element in
{\bf vectorexpression1} is TRUE or FALSE.

In our example above,

\begin{lstlisting}
> ifelse(m2 %%3 == 1,0,m2) # (see below)
\end{lstlisting}

the expression \verb^m2 %%3 == 1^ evaluated to the boolean matrix

\begin{equation}
\left (
\begin{array}{rrr}
T & F & F\\
F & T & F
\end{array}
\right )
\end{equation}

(TRUE and FALSE may be abbreviated to T and F.)

The 0 was recycled to the matrix

\begin{equation}
\left (
\begin{array}{rrr}
0 & 0 & 0 \\
0 & 0 & 0
\end{array}
\right )
\end{equation}

while {\bf vectorexpression3}, {\bf m2}, evaluated to itself.

\section{Third Sample Programming Session}

This time, we focus on vectors and matrices.

\begin{lstlisting}
> m <- rbind(1:3,c(5,12,13))  
> m
     [,1] [,2] [,3]
[1,]    1    2    3
[2,]    5   12   13
> t(m)  # transpose
     [,1] [,2]
[1,]    1    5
[2,]    2   12
[3,]    3   13
> ma <- m[,1:2]
> ma
     [,1] [,2]
[1,]    1    2
[2,]    5   12
> rep(1,2)  # "repeat," make multiple copies
[1] 1 1
> ma %*% rep(1,2)  # matrix multiply
     [,1]
[1,]    3
[2,]   17
> solve(ma,c(3,17))  # solve linear system
[1] 1 1
> solve(ma)  # matrix inverse
     [,1] [,2]
[1,]  6.0 -1.0
[2,] -2.5  0.5
\end{lstlisting}

\section{Default Argument Values}

Consider the {\bf sort()} function, which is built-in to R, though the
following points hold for any function, including ones you write
yourself.

The online help for this function, invoked by

\begin{lstlisting}
> ?sort
\end{lstlisting}

shows that the call form (the simplest version) is

\begin{lstlisting}
sort(x, decreasing = FALSE, ...)
\end{lstlisting}

Here is an example:

\begin{lstlisting}
> x <- c(12,5,13)
> sort(x)
[1]  5 12 13
> sort(x,decreasing=FALSE)
[1] 13 12  5
\end{lstlisting}

So, the default is to sort in ascending order, i.e.\ the argument {\bf
decreasing} has TRUE as its default value.  If we want the default, 
we need not specify this argument.  If we want a descending-order sort,
we must say so.

\section{The R List Type}

The R {\bf list} type is, after vectors, the most important R construct.
A list is like a vector, except that the components are generally of
mixed types.

\subsection{The Basics}

Here is example usage:

\begin{lstlisting}
> g <- list(x = 4:6, s = "abc")
> g
$x
[1] 4 5 6

$s
[1] "abc"

> g$x  # can reference by component name
[1] 4 5 6
> g$s
[1] "abc"
> g[[1]]  # can reference by index, but note double brackets
[1] 4 5 6
> g[[2]]
[1] "abc"
> for (i in 1:length(g)) print(g[[i]])
[1] 4 5 6
[1] "abc"

# now have ftn oddcount() return odd count AND the odd numbers themselves, 
# using the R list type
> source("odd.R")
> oddcount
function(x)  {
   x1 <- x[x %% 2 == 1]
   list(odds=x1, numodds=length(x1))  # last computed value auto-returned
}
> # R's list type can contain any type; components delineated by $
> oddcount(y)
$odds
[1]  5 13

$numodds
[1] 2

> ocy <- oddcount(y)  # save the output in ocy, which will be a list
> ocy  
$odds
[1]  5 13

$numodds
[1] 2

> ocy$odds
[1]  5 13
> ocy[[1]]  # can get list elements using [[ ]] instead of $
[1]  5 13
> ocy[[2]]
[1] 2
\end{lstlisting}

\subsection{The Reduce() Function}

One often needs to combine elements of a list in some way.  One approach
to this is to use {\bf Reduce()}:

\begin{lstlisting}
> x <- list(4:6,c(1,6,8))
> x
[[1]]
[1] 4 5 6

[[2]]
[1] 1 6 8

> sum(x)
Error in sum(x) : invalid 'type' (list) of argument
> Reduce(sum,x)
[1] 30
\end{lstlisting}

Here {\bf Reduce()} cumulatively applied R's {\bf sum()} to {\bf x}.  Of
course, you can use it with functions you write yourself too.

Continuing the above example:

\begin{lstlisting}
> Reduce(c,x)  
[1] 4 5 6 1 6 8
\end{lstlisting}

\subsection{S3 Classes}

R is an object-oriented (and functional) language.  It features two
types of classes, S3 and S4.  I'll introduce S3 here.

An S3 object is simply a list, with a class name added as an {\it
attribute}:

\begin{lstlisting}
> j <- list(name="Joe", salary=55000, union=T)
> class(j) <- "employee"
> m <- list(name="Joe", salary=55000, union=F)
> class(m) <- "employee"
\end{lstlisting}

So now we have two objects of a class we've chosen to name {\bf
"employee"}.  Note the quotation marks.

We can write class {\it generic functions}:

\begin{lstlisting}
> print.employee <- function(wrkr) {
+    cat(wrkr$name,"\n")
+    cat("salary",wrkr$salary,"\n")
+    cat("union member",wrkr$union,"\n")
+ }
> print(j)
Joe 
salary 55000 
union member TRUE 
> j
Joe 
salary 55000 
union member TRUE 
\end{lstlisting}

What just happened?  Well, {\bf print()} in R is a {\it generic}
function, meaning that it is just a placeholder for a function specific
to a given class.  When we printed {\bf j} above, the R interpreter
searched for a function {\bf print.employee()}, which we had indeed
created, and that is what was executed.  Lacking this, R would have used
the print function for R lists, as before:

\begin{lstlisting}
> rm(print.employee)  # remove the function, to see what happens with print
> j
$name
[1] "Joe"

$salary
[1] 55000

$union
[1] TRUE

attr(,"class")
[1] "employee"
\end{lstlisting}

\section{Some Workhorse Functions}

\begin{lstlisting}
> m <- matrix(sample(1:5,12,replace=TRUE),ncol=2)
> m
[,1] [,2]
[1,]    2    1
[2,]    2    5
[3,]    5    4
[4,]    5    1
[5,]    2    1
[6,]    1    3
# call sum() on each row
> apply(m,1,sum)
[1] 3 7 9 6 3 4
# call sum() on each column
> apply(m,2,sum)
[1] 17 15
> f <- function(x) sum(x[x >= 4])
# call f() on each row
> apply(m,1,f)
[1] 0 5 9 5 0 0
> l <- list(x = 5, y = 12, z = 13)
# apply the given funciton to each element of l, producing a new list
> lapply(l,function(a) a+1)
$x
[1] 6                       

$y 
[1] 13
   
$z
[1] 14
# group the first column of m by the second
> sout <- split(m[,1],m[,2])               
> sout
$`1`
[1] 2 5 2
$`3`
[1] 1
$`4`          
[1] 5         
                                                                                $`5`
[1] 2
# find the size of each group, by applying the length() function
> lapply(sout,length)
$`1`
[1] 3
                                                                                $`3`          
[1] 1                                                                           
$`4`
[1] 1

$`5`                        
[1] 1
# like lapply(), but sapply() attempts to make vector output
> sapply(sout,length)
1 3 4 5
3 1 1 1
\end{lstlisting}

\section{Handy Utilities}

R functions written by others, e.g. in base R or in the CRAN repository
for user-contributed code, often return values which are class objects.
It is common, for instance, to have lists within lists.  In many cases
these objects are quite intricate, and not thoroughly documented.  In
order to explore the contents of an object---even one you write
yourself---here are some handy utilities:

\begin{itemize}

\item {\bf names()}:  Returns the names of a list.

\item {\bf str()}:  Shows the first few elements of each component.

\item {\bf summary()}:  General function.  The author of a class {\bf
x} can write a version specific to {\bf x}, i.e. {\bf summary.x()}, to
print out the important parts; otherwise the default will print some
bare-bones information.

\end{itemize}

For example:

\begin{lstlisting}
> z <- list(a = runif(50), b = list(u=sample(1:100,25), v="blue sky"))
> z
$a
 [1] 0.301676229 0.679918518 0.208713522 0.510032893 0.405027042
0.412388038
 [7] 0.900498062 0.119936222 0.154996457 0.251126218 0.928304164
0.979945937
[13] 0.902377363 0.941813898 0.027964137 0.992137908 0.207571134
0.049504986
[19] 0.092011899 0.564024424 0.247162004 0.730086786 0.530251779
0.562163986
[25] 0.360718988 0.392522242 0.830468427 0.883086752 0.009853107
0.148819125
[31] 0.381143870 0.027740959 0.173798926 0.338813042 0.371025885
0.417984331
[37] 0.777219084 0.588650413 0.916212011 0.181104510 0.377617399
0.856198893
[43] 0.629269146 0.921698394 0.878412398 0.771662408 0.595483477
0.940457376
[49] 0.228829858 0.700500359

$b
$b$u
 [1] 33 67 32 76 29  3 42 54 97 41 57 87 36 92 81 31 78 12 85 73 26 44
86 40 43

$b$v
[1] "blue sky"
> names(z)
[1] "a" "b"
> str(z)
List of 2
 $ a: num [1:50] 0.302 0.68 0.209 0.51 0.405 ...
 $ b:List of 2
  ..$ u: int [1:25] 33 67 32 76 29 3 42 54 97 41 ...
  ..$ v: chr "blue sky"
> names(z$b)
[1] "u" "v"
> summary(z)
  Length Class  Mode   
a 50     -none- numeric
b  2     -none- list   
\end{lstlisting}

\section{Data Frames}

Another workhorse in R is the {\it data frame}.  A data frame works in
many ways like a matrix, but differs from a matrix in that it can mix
data of different modes.  One column may consist of integers, while
another can consist of character strings and so on.  Within a column,
though, all elements must be of the same mode, and all columns must have
the same length.

We might have a 4-column data frame on people, for instance, with
columns for height, weight, age and name---3 numeric columns and 1
character string column.

Technically, a data frame is an R list, with one list element per
column; each column is a vector.  Thus columns can be referred to by
name, using the {\bf \$} symbol as with all lists, or by column number,
as with matrices.  The matrix {\bf a[i,j]} notation for the element of
{\bf a} in row {\bf i}, column {\bf j}, applies to data frames.  So do
the {\bf rbind()} and {\bf cbind()} functions, and various other matrix
operations, such as filtering.

Here is an example using the dataset {\bf airquality}, built in to R for
illustration purposes.  You can learn about the data through R's online
help, i.e. 

\begin{lstlisting}
> ?airquality
\end{lstlisting}

Let's try a few operations:

\begin{lstlisting}
> names(airquality)
[1] "Ozone"   "Solar.R" "Wind"    "Temp"    "Month"   "Day"    
> head(airquality)  # look at the first few rows
  Ozone Solar.R Wind Temp Month Day
1    41     190  7.4   67     5   1
2    36     118  8.0   72     5   2
3    12     149 12.6   74     5   3
4    18     313 11.5   62     5   4
5    NA      NA 14.3   56     5   5
6    28      NA 14.9   66     5   6
> airquality[5,3]  # wind on the 5th day
[1] 14.3
> airquality$Wind[3]  # same
[1] 12.6
> nrow(airquality)  # number of days observed
[1] 153
> ncol(airquality)  # number of variables
[1] 6
> airquality$Celsius <- (5/9) * (airquality[,4] - 32)  # new variable
> names(airquality)
[1] "Ozone"   "Solar.R" "Wind"    "Temp"    "Month"   "Day"     "Celsius"
> ncol(airquality)
[1] 7
> airquality[1:3,]
  Ozone Solar.R Wind Temp Month Day  Celsius
1    41     190  7.4   67     5   1 19.44444
2    36     118  8.0   72     5   2 22.22222
3    12     149 12.6   74     5   3 23.33333
> aqjune <- airquality[airquality$Month == 6,]  # filter op
> nrow(aqjune)
[1] 30
> mean(aqjune$Temp)
[1] 79.1
> write.table(aqjune,"AQJune")  # write data frame to file
> aqj <- read.table("AQJune",header=T)  # read it in
\end{lstlisting}

\section{Graphics}

R excels at graphics, offering a rich set of capabilities, from
beginning to advanced.  In addition to the functions in base R,
extensive graphics packages are available, such as {\bf lattice} and
{\bf ggplot2}.

One point of confusion for beginners involves saving an R graph that
is currently displayed on the screen to a file.  Here is a function for
this, which I include in my R startup file, {\bf .Rprofile}, in my home
directory:

\begin{lstlisting}
pr2file
function (filename) 
{
    origdev <- dev.cur()
    parts <- strsplit(filename, ".", fixed = TRUE)
    nparts <- length(parts[[1]])
    suff <- parts[[1]][nparts]
    if (suff == "pdf") {
        pdf(filename)
    }
    else if (suff == "png") {
        png(filename)
    }
    else jpeg(filename)
    devnum <- dev.cur()
    dev.set(origdev)
    dev.copy(which = devnum)
    dev.set(devnum)
    dev.off()
    dev.set(origdev)
}
\end{lstlisting}

The code, which I won't go into here, mostly involves manipulation of
various R graphics devices.  I've set it up so that you can save to a
file of type either PDF, PNG or JPEG, implied by the file name you give.

\section{Packages}

The analog of a library in C/C++ in R is called a {\bf package} (and
often loosely referred to as a {\bf library}).  Some are already included
in base R, while others can be downloaded, or written by yourself.

\begin{lstlisting}
> library(parallel)  # load the package named 'parallel'
> ls(package:parallel)  # let's see what functions it gave us
 [1] "clusterApply"        "clusterApplyLB"      "clusterCall"        
 [4] "clusterEvalQ"        "clusterExport"       "clusterMap"         
 [7] "clusterSetRNGStream" "clusterSplit"        "detectCores"        
[10] "makeCluster"         "makeForkCluster"     "makePSOCKcluster"   
[13] "mc.reset.stream"     "mcaffinity"          "mccollect"          
[16] "mclapply"            "mcMap"               "mcmapply"           
[19] "mcparallel"          "nextRNGStream"       "nextRNGSubStream"   
[22] "parApply"            "parCapply"           "parLapply"          
[25] "parLapplyLB"         "parRapply"           "parSapply"          
[28] "parSapplyLB"         "pvec"                "setDefaultCluster"  
[31] "splitIndices"        "stopCluster"        
> ?pvec  # let's see how one of them works
\end{lstlisting}

The CRAN repository of contributed R code has thousands of R packages
available.  It also includes a number of ``tables of contents'' for
specific areas, say time series, in the form of CRAN Task Views.  See
the R home page, or simply Google ``CRAN Task View."

\begin{lstlisting}
> install.packages("cts","~/myr")  # download into desired directory
--- Please select a CRAN mirror for use in this session ---
...
downloaded 533 Kb


The downloaded binary packages are in
        /var/folders/jk/dh9zkds97sj23kjcfkr5v6q00000gn/T//RtmplkKzOU/downloaded_packages
> ?library
> library(cts,lib.loc="~/myr")

Attaching package: ‘cts’
...
\end{lstlisting}

\section{Other Sources for Learning R}

There are tons of resources for R on the Web.  You may wish to start
with the links at \url{http://heather.cs.ucdavis.edu/~matloff/r.html}.

\section{Online Help} 

R's {\bf help()} function, which can be invoked also with a
question mark, gives short descriptions of the R functions. For
example, typing

\begin{Verbatim}[fontsize=\relsize{-2}]
> ?rep
\end{Verbatim}
  
will give you a description of R's {\bf rep()} function.

An especially nice feature of R is its {\bf example()} function, which
gives nice examples of whatever function you wish to query. For
instance, typing

\begin{lstlisting}
> example(wireframe())
\end{lstlisting}
  
will show examples---R code and resulting pictures---of {\bf wireframe()},
one of R's 3-dimensional graphics functions.

\section{Debugging in R}

The internal debugging tool in R, {\bf debug()}, is usable but rather
primitive.  Here are some alternatives:

\begin{itemize}

\item The RStudio IDE has a built-in debugging tool.

\item For Emacs users, there is {\bf ess-tracebug}.

\item The StatET IDE for R on Eclipse has a nice debugging tool.  Works
on all major platforms, but can be tricky to install.

\item My own debugging tool, {\bf debugR}, is extensive and easy to
install, but for the time being is limited to Linux, Mac and other
Unix-family systems.  See {\it
http://heather.cs.ucdavis.edu/debugR.html}.

\end{itemize}

\section{Complex Numbers}

If you have need for complex numbers, R does handle them.  Here is a
sample of use of the main functions of interest:

\begin{lstlisting}
> za <- complex(real=2,imaginary=3.5)
> za
[1] 2+3.5i
> zb <- complex(real=1,imaginary=-5)
> zb
[1] 1-5i
> za * zb
[1] 19.5-6.5i
> Re(za)
[1] 2
> Im(za)
[1] 3.5
> za^2
[1] -8.25+14i
> abs(za)
[1] 4.031129
> exp(complex(real=0,imaginary=pi/4))
[1] 0.7071068+0.7071068i
> cos(pi/4)
[1] 0.7071068
> sin(pi/4)
[1] 0.7071068
\end{lstlisting}

Note that operations with complex-valued vectors and matrices work as
usual; there are no special complex functions.

\section{Further Reading}

For further information about R as a programming language, there is my
book, {\it The Art of R Programming: a Tour of Statistical Software
Design}, NSP, 2011, as well as Hadley Wickham's {\it Advanced R},
Chapman and Hall, 2014.

For R's statistical functions, a plethora of excellent books is
available. such as {\it The R Book} (2nd Ed.), Michael Crowley, Wiley,
2012.   I also very much like {\it R in a Nutshell} (2nd Ed.), Joseph Adler,
O'Reilly, 2012, and even Andrie de Vries' {\it R for Dummies}, 2012.
