\chapter{Stop and Review:  Probability Structures}
\label{stopandreview}

There's quite a lot of material in the preceding chapters, but it's
crucial that you have a good command of it before proceeding, as the
coming chapters will continue to build on it.

With that aim, here are the highlights of what we've covered so far,
with links to the places at which they were covered:

\begin{itemize}

\item {\bf expected value} (Section \ref{expval}):

Consider random variables X and Y (not assumed independent), and 
constants $c_1$ and $c_2$.  We have:

\begin{equation}
E(X + Y) = EX + EY
\end{equation}

\begin{equation}
E(c_1 X) = c_1 EX
\end{equation}

\begin{equation}
E(c_1 X + c_2 Y) = c_1 EX + c_2 EY
\end{equation}

By induction,

\begin{equation}
E(a_1 U_1 + ... + a_k U_k) = a_1 EX_1 + ... + a_k EX_k
\end{equation}

for random variables $U_i$ and constants $a_i$.

\item {\bf variance} (Section \ref{variance}):

For any variable W,

\begin{equation}
Var(W) = E[(W - EW)^2] = E(W^2) - (EW)^2
\end{equation}

Consider random variables X and Y (now assumed independent), and 
constants $c_1$ and $c_2$.  We have:

\begin{equation}
Var(X+Y) = Var(X) + Var(Y)
\end{equation}

\begin{equation}
Var(c_1 X) = c^2_1 Var(X)
\end{equation}

By induction,

\begin{equation}
Var(a_1 U_1 + ... + a_k U_k) = a^2_1 Var(U_1) + ... + a^2_k Var(U_k)
\end{equation}

for independent random variables $U_i$ and constants $a_i$.

\item {\bf indicator random variables} (Section \ref{indicator}):

Equal 1 or 0, depending on whether a specified event A occurs.

If T is an indicator random variable for the event A, then

\begin{equation}
ET = P(A), ~~ Var(T) = P(A) [1-P(A)]
\end{equation}

\item {\bf distributions}:

   \begin{itemize}

   \item {\bf cdfs}(Section \ref{presentsaproblem}):

   For any random variable X,

   \begin{equation}
   F_X(t) = P(X \leq t),  ~~~~ -\infty < t < \infty
   \end{equation}

   \item {\bf pmfs} (Section \ref{dstrdef}): 

   For a discrete random variable X,

   \begin{equation}
   p_X(k) = P(X = k)
   \end{equation}

   \item {\bf density functions} (Section \ref{dstrdef}): 

   For a continuous random variable X,

   \begin{equation}
   f_X(t) = \frac{d}{dt} F_X(t), ~~~~ -\infty < t < \infty
   \end{equation}

   and

   \begin{equation}
   P(X \textrm{ in } A) = \int_A f_X(s) ~ ds
   \end{equation}

   \end{itemize}

\item {\bf famous parametric families of distributions}:

Just like one can have a family of curves, say $sin(2\pi n \theta(t)$
(different curve for each n and $\theta$), certain families of
distributions have been found useful.  They're called {\it parametric
families}, because they are indexed by one or more parameters,
anlagously to n and $\theta$ above.

discrete:

   \begin{itemize}

   \item {\bf geometric} (Section \ref{geom})

   Number of i.i.d. trials until first success.  For success probability p: 

   \begin{equation}
   p_N(k) = (1-p)^k p
   \end{equation}

   \begin{equation}
   EN = 1/p, ~~ Var(N) =  \frac{1-p}{p^2}
   \end{equation}

   \item {\bf binomial} (Section \ref{binom}):

   Number of successes in n i.i.d. trials, probability p of success per
   trial:

   \begin{equation}
   p_N(k) = \binom{n}{k} p^k (1-p)^{n-k}  
   \end{equation}

   \begin{equation}
   EN = np, ~~ Var(N) = np(1-p)
   \end{equation}

   \item {\bf Poisson} (Section \ref{poisfam}):

   Has often been found to be a good model for counts over time periods.

   One parameter, often called $\lambda$.  Then

   \begin{equation}
   p_N(k) = \frac{e^{- \lambda} \lambda^k}{k!}, k = 0,1,2,...
   \end{equation}

   \begin{equation}
   EN = Var(N) = \lambda
   \end{equation}

   \item {\bf negative binomial} (Section \ref{negbin}):

   Number of i.i.d. trials until r$^{th}$ success.  For success
   probability p: 

   \begin{equation}
   p_N(k) =  \binom{k-1}{r-1} (1-p)^{k-r} p^r, k = r, r+1, ...
   \end{equation}

   \begin{equation}
   E(N) = r \cdot \frac{1}{p}, ~~ Var(N) = r \cdot \frac{1-p}{p^2}
   \end{equation}

   \end{itemize}

continuous:

   \begin{itemize}

   \item {\bf uniform} (Section \ref{unifprops}):

   All points ``equally likely.''  If the interval is (q,r),

   \begin{equation}
   f_X(t) = \frac{1}{r-q}, ~~~~ q < t < r
   \end{equation}

   \begin{equation}
   EX = = \frac{q+r}{2}, ~~ Var(D) = \frac{1}{12} (r-q)^2
   \end{equation}

   \item {\bf normal (Gaussian)} (Section \ref{normalfam}):

   ``Bell-shaped curves.''  Useful due to Central Limit Theorem (Section
   \ref{theclt}.  (Thus good approximation to binomial distribution.)

   Closed under affine transformations (Section \ref{affine})!

   Parameterized by mean and variance, $\mu$ and $\sigma^2$:

   \begin{equation}
   f_X(t) =  \frac{1}{\sqrt{2\pi} \sigma} 
   ~ e^{- 0.5 \left (\frac{t-\mu}{\sigma} \right )^2}, -\infty < t < \infty
   \end{equation}

   {\bf exponential} (Section \ref{exponfam}):

   \item Memoryless!  One parameter, usually called $\lambda$.
   Connectedt to Poisson family.

   \begin{equation}
   f_X(t) = = \lambda e^{-\lambda t}, 0 < t < \infty
   \end{equation}

   \begin{equation}
   EX = 1/\lambda, ~~ Var(X) = 1/\lambda^2
   \end{equation}

   \item {\bf gamma} (Section \ref{gammafam}):

   Special case, Erlang family, arises as the distribution of the sum of
   i.i.d. exponential random variables.

   \begin{equation}
   f_X(t) = \frac{1}{\Gamma(r)} \lambda^r t^{r-1} e^{-\lambda t}, ~ t > 0
   \end{equation}

   \end{itemize}

\item {\bf iterated expected values:}

   \begin{itemize}

   \item For discrete U (\ref{adamsdiscrete}),

   \begin{equation}
   E(V) = \sum_c P(U = c) ~ E(V ~|~ U = c) 
   \end{equation}

   \item For continuous V (\ref{adamscontin}),

   \begin{equation}
   E(W) = \int_{-\infty}^{\infty} f_V(t) ~ E(W ~|~ V = t)~ dt
   \end{equation}



   \end{itemize}

\end{itemize}


