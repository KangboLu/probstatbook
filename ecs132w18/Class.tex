\chapter{Classification} 
\label{chap:class}

% \section{Nonlinear Parametric Regression Models}
% \label{nonlin}
% 
% We pointed out in Section \ref{whatmeans} that the word {\it linear} in
% {\it linear regression model} means linear in $\beta$, not in t.  This
% is the most popular approach, as it is computationally easy, but
% nonlinear models are often used.
% 
% The most famous of these is the {\bf logistic} model, for the case in
% which $Y$ takes on only the values 0 and 1.  As we have seen before
% (Section \ref{indicator}), in this case the expected value becomes a
% probability.  The logistic model for a nonvector $X$ is then
% 
% \begin{equation}
% \label{logit}
% m_{Y;X}(t) = P(Y = 1 | X = t) = \frac{1}{1+e^{-(\beta_0+\beta_1 t})}
% \end{equation}
% 
% It extends to the case of vector-valued $X$ in the obvious way. 
% 
% The logistic model is quite widely used in computer science, in medicine,
% economics, psychology and so on.  We'll return to this model in Section
% \ref{logitsec}.
% 
% Here is an example of a nonlinear model used in kinetics of chemical
% reactions, with r = 3: \footnote{See 
% \url{http://www.mathworks.com/access/helpdesk/help/toolbox/stats/rsmdemo.html}.}
% 
% \begin{equation}
% m_{Y;X}(t) = \frac
% {\beta_1 t^{(2)} - t^{(3)}/\beta_5}
% {1+\beta_2 t^{(1)} + \beta_3 t^{(2)} + \beta_4 t^{(3)}}
% \end{equation}
% 
% Here the X vector is (hydrogen, n-pentane, isopentane) and Y is the
% reaction rate.
% 
% Unfortunately, in most cases, the least-squares estimates of the
% parameters in nonlinear regression do not have closed-form solutions,
% and numerical methods must be used.  But R does that for you, via the
% {\bf nls()} function in general, and via {\bf glm()} for the logistic
% and related models in particular.

In prediction problems, in the special case in which Y is an indicator
variable, with the value 1 if the object is in a class and 0 if not, the
regression problem is called the {\bf classification
problem}.\footnote{The case of c $>$ 2 classes will be treated in
Section \ref{multiclass}.}

We'll formalize this idea in Section \ref{meanisprob}, but first,
here are some examples:

\begin{itemize}

\item A forest fire is now in progress.  Will the fire reach a certain
populated neighborhood?  Here Y would be 1 if the fire reaches the
neighborhood, 0 otherwise.  The predictors might be wind direction,
distance of the fire from the neighborhood, air temperature and
humidity, and so on.

\item Is a patient likely to develop diabetes?  This problem has been
studied by many researchers, e.g.  Using Neural Networks To Predict the
Onset of Diabetes Mellitus, Murali S. Shanker {\it J. Chem. Inf. Comput.
Sci.}, 1996, 36 (1), pp 35–41.  A famous data set involves Pima Indian
women, with Y being 1 or 0, depending on whether the patient does
ultimately develop diabetes, and the predictors being the number of
times pregnant, plasma glucose concentration, diastolic blood pressure,
triceps skin fold thickness, serum insulin level, body mass index,
diabetes pedigree function and age. 

\item Is a disk drive likely to fail soon?  This has been studied for
example in Machine Learning Methods for Predicting Failures in Hard
Drives: A Multiple-Instance Application, by Joseph F. Murray, Gordon F.
Hughes, and Kenneth Kreutz-Delgado, {\it Journal of Machine Learning
Research} 6 (2005) 783-816.  Y was 1 or 0, depending on whether the
drive failed, and the predictors were temperature, number of read
errors, and so on.

\item An online service has many customers come and go.  It would like
to predict who is about to leave, so as to offer them a special deal for
staying with this firm.

\item Of course, a big application is character recognition, based on
pixel data.  This is different from the above examples, as there are
more than two classes, many more.  We'll return to this point soon.

\end{itemize}

In electrical engineering the classification is called {\bf pattern
recognition}, and the predictors are called {\bf features}.  In computer
science the term {\bf machine learning} usually refers to classification
problems.  Different terms, same concept.

\section{Classification = Regression}
\label{meanisprob}

All of the many machine learning algorithms, despite their
complexity, really boil down to regression at their core.  Here's why:

\subsection{What Happens with Regression in the Case Y = 0,1?}

As we have frequently noted the mean of any indicator random variable is
the probability that the variable is equal to 1 (Section
\ref{indicator}).  Thus in the case in which our response variable Y
takes on only the values 0 and 1, i.e.  classification problems, the
regression function reduces to

\begin{equation}
m_{Y;X}(t) = P(Y = 1 | X = t)
\end{equation}

(Remember that X and t are vector-valued.)

As a simple but handy example, suppose Y is gender (1 for male, 0 for
female), $X^{(1)}$ is height and $X^{(2)}$ is weight, i.e. we are
predicting a person's gender from the person's height and weight.  Then
for example, $m_{Y;X}(70,150)$ is the probability that a person of
height 70 inches and weight 150 pounds is a man.  Note again that this
probability is a population fraction, the fraction of men among all
people of height 70 and weight 150 in our population.

Make a mental note of the optimal prediction rule, if we know the
population regression function:

\begin{quote}
Given X = t, the optimal prediction rule is to predict that Y = 1 if and
only if $m_{Y;X}(t) > 0.5$.
\end{quote}

So, if we known a certain person is of height 70 and weight 150, our
best guess for the person's gender is to predict the person is male
if and only if $m_{Y;X}(70,150) > 0.5$.\footnote{Things change in the
multiclass case, though, as will be seen in Section \ref{multiclass}.}

The optimality makes intuitive sense, and is shown in Section
\ref{optimal2}.   

\section{Logistic Regression:  a Common Parametric Model for the
Regression Function in Classification Problems}
\label{logitsec}

Remember, we often try a parametric model for our regression function
first, as it means we are estimating a finite number of quantities,
instead of an infinite number.  Probably the most commonly-used model is
that of the {\bf logistic function} (often called ``logit'').  Its
r-predictor form is 

\begin{equation}
\label{logit2}
m_{Y;X}(t) = P(Y = 1 | X = t) = \frac{1}{1+e^{-(\beta_0+\beta_1
t_1+...+\beta_r t_r)}}
\end{equation}

For instance, consider the patent example in Section \ref{examples}.
Under the logistic model, the population proportion of all patents that
are publicly funded, among those that contain the word ``NSF,'' do not
contain ``NIH,'' and make five claims would have the value

\begin{equation}
\frac{1}{1+e^{-(\beta_0 + \beta_1 + 5\beta_3)}}
\end{equation}

\subsection{The Logistic Model:  Motivations}
\label{logitmotivations}

The logistic function itself, 

\begin{equation}
\frac{1}{1+e^{-u}}
\end{equation}

has values between 0 and 1, and is thus a good candidate for modeling a
probability.  Also, it is monotonic in u, making it further attractive,
as in many classification problems we believe that $m_{Y;X}(t)$ should
be monotonic in the predictor variables.

But there are additional reasons to use the logit model, as it includes
many common parametric models for X.  To see this, note that we can
write, for vector-valued discrete X and t,

\begin{eqnarray}
\label{oneonelogodds}
P(Y = 1 | X = t) &=&  \frac{P(Y = 1 ~ \textrm{and} ~ X = t)}{P(X = t)} \\
&=& \frac{P(Y = 1) P(X = t | Y = 1)}{P(X = t)} \\
&=& \frac{P(Y = 1) P(X = t | Y = 1)}
{P(Y = 1) P(X = t | Y = 1) + P(Y = 0) P(X = t | Y = 0)} \\
\label{intermedone}
&=& \frac{1}
{1 + \frac{(1-q) P(X = t | Y = 0)}{q P(X = t | Y = 1)}}
\label{finaloneone}
\end{eqnarray}

where $q = P(Y = 1)$ is the proportion of members of the population which
have $Y = 1$.  (Keep in mind that this probability is unconditional!!!!
In the patent example, for instance, if say $q = 0.12$, then 12\% of all
patents in the patent population---without regard to words used, numbers
of claims, etc.---are publicly funded.) 

% \checkpoint

If $X$ is a continuous random vector, then the analog of
(\ref{finaloneone}) is

\begin{equation}
\label{oneonelogodds2}
P(Y = 1 | X = t) = \frac{1}
{1 + \frac{(1-q) f_{X|Y=0}(t)}{q f_{X|Y=1}(t)}}
\end{equation}

Now for simplicity, suppose $X$ is scalar, i.e. r = 1.  And suppose
that, given Y, X has a normal distribution.  In other words, within each
class, $Y$ is normally distributed.  Suppose also that the two
within-class variances of X are equal, with common value $\sigma^2$, but
with means $\mu_0$ and $\mu_1$.  Then

\begin{equation}
f_{X|Y=i}(t) = \frac{1}{\sqrt{2\pi\sigma}}
\exp{\left [ -0.5 \left (\frac{t-\mu_i}{\sigma} \right )^2 \right ]}
\end{equation}

After doing some elementary but rather tedious algebra,
(\ref{oneonelogodds2}) reduces to the logistic form

\begin{equation}
\label{logitform}
\frac{1}{1+e^{-(\beta_0 + \beta_1 t)}}
\end{equation}

where 

\begin{equation}
\label{logitbeta0}
\beta_0 = -\ln{\left ( \frac{1-q}{q} \right )} + 
\frac{\mu^2_0 - \mu^2_1}{2\sigma^2},
\end{equation}

and

\begin{equation}
\label{logitbeta1}
\beta_1 = \frac{\mu_1 - \mu_0}{\sigma^2},
\end{equation} 

{\bf In other words, if X is normally distributed in both classes, with
the same variance but different means, then $m_{Y;X}()$ has the logistic
form!} And the same is true if X is multivariate normal in each class,
with different mean vectors but equal covariance matrices.  (The algebra
is even more tedious here, but it does work out.)  Given the central
importance of the multivariate normal family---the word {\it central}
here is a pun, alluding to the (multivariate) Central Limit
Theorem---this makes the logit model even more useful.

If you retrace the derivation above, you will see that the logit model
will hold for any within-class distributions for which 

\begin{equation}
\label{logitlogodds}
\ln \left ( \frac{f_{X|Y=0}(t)}{f_{X|Y=1}(t)} \right )
\end{equation}

(or its discrete analog) is linear in t.  Well guess what---this
condition is true for exponential distributions too!  Work it out for
yourself.

% \checkpoint

In fact, a number of famous distributions imply the logit model.  So,
logit is not only a good intuitive model, as discussed above, but in
addition there are some good theoretical recommendations for it.

\subsection{Esimation and Inference for Logit Coefficients}

We fit a logit model in R using the {\bf glm()} function, with the
argument {\bf family=binomial}.  The function finds Maximum Likelihood
Estimates (Section \ref{mle}) of the $\beta_i$.\footnote{As in the case
of linear regression, estimation and inference are done conditionally on
the values of the predictor variables $X_i$.} 

The output gives standard errors for the $\widehat{\beta}_i$ as in the
linear model case.  This enables the formation of confidence intervals
and significance tests on individual $\widehat{\beta}_i$.  For inference
on linear combinations of the $\widehat{\beta}_i$, use the {\bf vcov()}
function as in the linear model case.

\section{Example:  Forest Cover Data}
\label{forest}

Let's look again at the forest cover data we saw in Section
\ref{forestcover}.\footnote{There is a problem here, to be discussed in
Section \ref{ymarg}, but which will not affect the contents of this
section.} Recall that this application has the Prediction goal, rather
than the Description goal;\footnote{Recall these concepts from Section
\ref{goals}.} we wish to predict the type of forest cover.  There were
seven classes of forest cover.  

\subsubsection{R Code}

For simplicity, I restricted my analysis to classes 1 and
2.\footnote{This will be generalized in Section \ref{multiclass}.}  In
my R analysis I had the class 1 and 2 data in objects {\bf cov1} and
{\bf cov2}, respectively.  I combined them,

\begin{Verbatim}[fontsize=\relsize{-2}]
> cov1and2 <- rbind(cov1,cov2)
\end{Verbatim}

and created a new variable to serve as Y, recoding the 1,2 class names
to 1,0:

\begin{Verbatim}[fontsize=\relsize{-2}]
cov1and2[,56] <- ifelse(cov1and2[,55] == 1,1,0)
\end{Verbatim}

Let's see how well we can predict a site's class from the variable HS12
(hillside shade at noon) that we investigated in Chapter
\ref{chap:sigtests}, using a logistic model.

As noted earlier, in R we fit logistic models via the {\bf glm()}
function, for generalized linear models.  The word {\it generalized}
here refers to models in which some function of $m_{Y;X}(t)$ is linear
in parameters $\beta_i$.  For the classification model, 

\begin{equation}
\ln \left (  m_{Y;X}(t) / [1-m_{Y;X}(t)] \right )
= \beta_0 + \beta_1 t^{(1)} + ... + \beta_r t^{(r)}
\end{equation}

(Recall the discussion surrounding (\ref{logitlogodds}).)

This kind of generalized linear model is specified in R by setting the
named argument {\bf family} to {\bf binomial}.  Here is the call:

\begin{Verbatim}[fontsize=\relsize{-2}]
> g <- glm(cov1and2[,56] ~ cov1and2[,8],family=binomial)
\end{Verbatim}

The result was:

\begin{Verbatim}[fontsize=\relsize{-2}]
> summary(g)

Call:
glm(formula = cov1and2[, 56] ~ cov1and2[, 8], family = binomial)

Deviance Residuals:
   Min      1Q  Median      3Q     Max
-1.165  -0.820  -0.775   1.504   1.741

Coefficients:
               Estimate Std. Error z value Pr(>|z|)
(Intercept)    1.515820   1.148665   1.320   0.1870
cov1and2[, 8] -0.010960   0.005103  -2.148   0.0317 *
---
Signif. codes:  0 ‘***’ 0.001 ‘**’ 0.01 ‘*’ 0.05 ‘.’ 0.1 ‘ ’ 1

(Dispersion parameter for binomial family taken to be 1)

    Null deviance: 959.72  on 810  degrees of freedom
Residual deviance: 955.14  on 809  degrees of freedom
AIC: 959.14

Number of Fisher Scoring iterations: 4
\end{Verbatim}

\subsection{Analysis of the Results}

You'll immediately notice the similarity to the output of {\bf lm()}.
In particular, note the Coefficients section.  There we have the
estimates of the population coefficients $\beta_i$, their standard
errors, and p-values for the tests of $H_0: \beta_i = 0$.

One difference from the linear case is that in that case, the tests of 

\begin{equation}
 H_0:  \beta_i = 0
\end{equation}

were ``exact,'' based on the Student-t distribution, rather than being
approximate tests based on the Central Limit Theorem.  The assumption is
that the conditional distribution of the response given the predictors
is exactly normal.  As noted before, those tests can't possibly be
exact, since the assumption cannot exactly hold.

But in the logit case, no ``exact'' test is available anyway, so R does
indeed do approximate tests based on the Central Limit Theorem.
Accordingly, the test column in the output is labeled ``z value,''
rather than ``t value'' as before.

At any rate, we see that for example $\widehat{\beta}_1 = -0.01$.  This
is tiny, reflecting our analysis of this data in Chapter
\ref{chap:sigtests}.  There we found that the estimated mean values of
HS12 for cover types 1 and 2 were 223.8 and 226.3, a difference of only
2.5, minuscule in comparison to the estimated means themselves.  That
difference in essence now gets multiplied by 0.01.  Let's see the effect
on the regression function, i.e. the probability of cover type 1 given
HS12.

Note first, though, that things are a little more complicated than they
were in the linear case.  Recall our first baseball data analysis, in
Section \ref{baseball1}.  Our model for $m_{Y;X}()$ was

\begin{equation}
\textrm{mean weight = } 
\beta_0 + \beta_1 \textrm{ height}
\end{equation}

After estimating the $\beta_i$ from the data, we had the estimated
regression equation,

\begin{lstlisting}
mean weight = -155.092 + 4.841 height
\end{lstlisting}

In presenting these results to others, we can illustrate the above
equation by noting, for instance, that a 3-inch difference in height
corresponds to an estimated $3 \times 4.841 = 14.523$ difference in mean
weight.  The key point, though, is that this difference is the same
whether we are comparing 70-inch-tall players with 73-inch-tall ones, or
comparing 72-inch-tall players with 75-inch-tall ones, etc.  

In our nonlinear case, the logit, the different in regression value will
NOT depend only on the difference between the predictor values, in this
case HS12.  We cannot simply say something like ``If two forest sites
differ in HS12 by 15, then the difference in probability of cover type 1
differs by such-and-such an amount.''  

Instead we'll have to choose two specific HS11 values for our
illustration.  Let's use the estimated mean HS12 values for the two
cover types, 223.8 and then 226.3, found earlier.  

So, in (\ref{logit2}), let's plug in our estimates 1.52 and -0.01 from
our R output above, twice, first with HS12 = 223.8 and then with HS12 =
226.3   

In other words, let's imagine two forest sites, with unknown cover type,
but known HS12 values 223.8 and 226.8 that are right in the center of
the HS12 distribution for the two cover types.  What would we predict
for the cover types to be for those two sites?

Plugging in to (\ref{logit2}), the results are 0.328 and 0.322,
respectively.  Remember, these numbers are the estimated probabilities
that we have cover type 1, given HS12.  So, our guess---predicting
whether we have cover type 1 or 2---isn't being helped much by knowing
HS12.

In other words, HS12 isn't having much effect on the probability of
cover type 1, and so it cannot be a good predictor of cover type.  

{\bf And yet...} the R output says that $\beta_1$ is ``significantly''
different from 0, with a p-value of 0.03.  Thus, we see once again that
significance testing does not achieve our goal.  

