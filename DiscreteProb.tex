\chapter{Discrete Random Variables}
\label{dis}

This chapter will introduce entities called {\it discrete random
variables}.  Some properties will be derived for means of such
variables, with most of these properties actually holding for random
variables in general.  Well, all of that seems abstract to you at this
point, so let's get started.

\section{Random Variables}

\begin{definition}
A {\bf random variable} is a numerical outcome of our experiment.
\end{definition}

For instance, consider our old example in which we roll two dice, with X
and Y denoting the number of dots we get on the blue and yellow dice,
respectively.  Then X and Y are random variables, as they are numerical
outcomes of the experiment.  Moreover, X+Y, 2XY, sin(XY) and so on are
also random variables.

In a more mathematical formulation, with a formal sample space defined,
a random variable would be defined to be a real-valued function whose
domain is the sample space.

\section{Discrete Random Variables}
\label{support}

In our dice example, the random variable X could take on six values
in the set \{1,2,3,4,5,6\}.  We say that the {\bf support} of X is 
\{1,2,3,4,5,6\}.  This is a finite set.  

In the ALOHA example, $X_1$ and $X_2$ each have support \{0,1,2\}, again
a finite set.\footnote{We could even say that $X_1$ takes on only values
in the set \{1,2\}, but if we were to look at many epochs rather than
just two, it would be easier not to make an exceptional case.}

Now think of another experiment, in which we toss a coin until we get
heads.  Let N be the number of tosses needed.  Then the support of N is
the set \{1,2,3,...\}  This is a countably infinite set.\footnote{This
is a concept from the fundamental theory of mathematics.  Roughly
speaking, it means that the set can be assigned an integer labeling,
i.e. item number 1, item number 2 and so on.  The set of positive even
numbers is countable, as we can say 2 is item number 1, 4 is item number
2 and so on.  It can be shown that even the set of all rational numbers
is countable.}

Now think of one more experiment, in which we throw a dart at the
interval (0,1), and assume that the place that is hit, R, can take on
any of the values between 0 and 1.  Here the support is an uncountably
infinite set.

We say that X, $X_1$, $X_2$ and N are {\bf discrete} random variables,
while R is {\bf continuous}.  We'll discuss continuous random variables
in a later chapter.

\section{Independent Random Variables}
\label{indepdef}

We already have a definition for the independence of events; what about
independence of random variables?  Here it is:

\begin{quote}

Random variables X and Y are said to be {\bf independent} if for any
sets I and J, the events \{X is in I\} and \{Y is in J\} are
independent, i.e.  P(X is in I and Y is in J) = P(X is in I) P(Y is in
J). 

\end{quote}

Sounds innocuous, but the notion of independent random variables is
absolutely central to the field of probability and statistics, and will
pervade this entire book.  

\section{Example:  The Monty Hall Problem}

This is an example of how the use of random variables in ``translating''
a probability problem to mathematical terms can simplify and clarify
one's thinking.  Imagine, {\bf this simple device of introducing named
random variables into our analysis makes a problem that has vexed famous
mathematicians quite easy to solve!}

The Monty Hall Problem, which gets its name from a popular TV game show
host, involves a contestant choosing one of three doors.  Behind one
door is a new automobile, while the other two doors lead to goats.  The
contestant chooses a door and receives the prize behind the door.

The host knows which door leads to the car.  To make things interesting,
after the contestant chooses, the host will open one of the other doors
not chosen, showing that it leads to a goat.  Should the contestant now
change her choice to the remaining door, i.e.\ the one that she didn't
choose and the host didn't open?

Many people answer No, reasoning that the two doors not opened yet each
have probability 1/2 of leading to the car.  But the correct answer is
actually that the remaining door (not chosen by the contestant and not
opened by the host) has probability 2/3, and thus the contestant should
switch to it.  Let's see why.

Let

\begin{itemize}

\item C = contestant's choice of door (1, 2 or 3)
\item H = host's choice of door (1, 2 or 3)
\item A = door that leads to the automobile
 
\end{itemize}

We can make things more concrete by considering the case $C = 1, ~ H = 2$.
The mathematical formulation of the problem is then to find

\begin{equation}
\label{montyhall1}
P(A = 3 ~|~ C = 1, ~ H = 2) =
\frac
{P(A = 3, ~ C = 1, ~ H = 2)}
{P(C = 1, ~ H = 2)}
\end{equation}

The key point, commonly missed even by mathematically sophisticated
people, is the role of the host.  Write the numerator above as

\begin{equation}
\label{mhnumerator}
P(A = 3, ~ C = 1) ~ P(H = 2 ~|~ A = 3, ~ C = 1)
\end{equation}

Since $C$ and $A$ are independent random variables, the value of the
first factor in (\ref{mhnumerator}) is 

\begin{equation}
\frac{1}{3} \cdot
\frac{1}{3} = \frac{1}{9}
\end{equation}

What about the second factor?  Remember, the host knows that $A = 3$,
and since the contestant has chosen door 1, the host will open the only
remaining door that conceals a goat, i.e.\ door 2.  In other words,

\begin{equation}
P(H = 2 ~|~ A = 3, ~ C = 1) = 1
\end{equation}

On the other hand, if say $A = 1$, the host would randomly choose
between doors 2 and 3, so that

\begin{equation}
P(H = 2 ~|~ A = 1, ~ C = 1) = \frac{1}{2}
\end{equation}

It is left to the reader to complete the analysis, calculating the
denominator of (\ref{montyhall1}), and then showing in the end that

\begin{equation}
P(A = 3 ~|~ C = 1, ~ H = 2) = \frac{2}{3}
\end{equation}

According to the ``Monty Hall problem'' entry in Wikipedia, even Paul
Erd{\"o}s, one of the most famous mathematicians in history, gave the wrong
answer to this problem.  Presumably he would have avoided this by
writing out his analysis in terms of random variables, as above, rather
than say, a wordy, imprecise and ultimately wrong solution.

\section{Expected Value}
\label{expval}

\subsection{Generality---Not Just for \underline{Discrete} Random
Variables}

The concepts and properties introduced in this section form the very
core of probability and statistics.  {\bf Except for some specific
calculations, these apply to both discrete and continuous random
variables.}

The properties developed for {\it variance}, defined later in this
chapter, also hold for both discrete and continuous random variables.

\subsubsection{What Is It?}
\label{whatisit}

The term ``expected value'' is one of the many misnomers one encounters
in tech circles.  The expected value is actually not something we
``expect'' to occur.  On the contrary, it's often pretty unlikely.

For instance, let H denote the number of heads we get in tossing a coin
1000 times.  The expected value, you'll see later, is 500.  This is not
surprising, given the symmetry of the situation, but P(H = 500) turns
out to be about 0.025.  In other words, we certainly should not
``expect'' H to be 500.

Of course, even worse is the example of the number of dots that come up
when we roll a fair die.  The expected value is 3.5, a value which not
only rarely comes up, but in fact never does.

In spite of being misnamed, expected value plays an absolutely central
role in probability and statistics.

\subsection{Definition}

Consider a repeatable experiment with random variable X.  We say that
the {\bf expected value} of X is the long-run average value of X, as we
repeat the experiment indefinitely.

In our notebook, there will be a column for X.  Let $X_i$ denote the
value of X in the i$^{th}$ row of the notebook.  Then the long-run
average of X is 

\begin{equation}
\label{longrunavg}
\lim_{n \rightarrow \infty} \frac{X_1+...+X_n}{n}
\end{equation}

Suppose for instance our experiment is to toss 10 coins.  Let X denote
the number of heads we get out of 10.  We might get four heads in the
first repetition of the experiment, i.e. $X_1 = 4$, seven heads in the
second repetition, so $X_2 = 7$, and so on.  Intuitively, the long-run
average value of X will be 5.  (This will be proven below.) Thus we say
that the expected value of X is 5, and write E(X) = 5.

\subsection{Existence of the Expected Value}

The above defintion puts the cart before the horse, as it presumes that
the limit exists.  Theoretically speaking, this might not be the case.
However, it does exist if the $X_i$ have finite lower and upper bounds,
which is always true in the real world.  For instance, no person has
height of 50 feet, say, and no one has negative height either.

For the remainder of this book, we will usually speak of ``the''
expected value of a random variable without adding the qualifier ``if it
exists.''

\subsection{Computation and Properties of Expected Value}

Continuing the coin toss example above, let $K_{in}$ be the number of
times the value i occurs among $X_1,...,X_n$, i = 0,...,10, n =
1,2,3,...  For instance, $K_{4,20}$ is the number of times we get four
heads, in the first 20 repetitions of our experiment.  Then

\begin{eqnarray}
E(X) &=& \lim_{n \rightarrow \infty} \frac{X_1+...+X_n}{n} \\
&=& \lim_{n \rightarrow \infty} \frac{0 \cdot K_{0n}+1 \cdot K_{1n}+2 \cdot
K_{2n}...+10 \cdot K_{10,n}}{n} \\
&=& \sum_{i=0}^{10} i \cdot \lim_{n \rightarrow \infty} \frac{K_{in}}{n}
\end{eqnarray}

To understand that second equation, suppose when n = 5 we have
2, 3, 1, 2 and 1 for our values of $X_1,X_2,X_3,X_4,X_5,$.  Then we can
group the 2s together and group the 1s together, and write

\begin{equation}
2 +3 + 1 + 2 + 1 = 2 \times 2 + 2 \times 1 + 1 \times 3
\end{equation}

But $\lim_{n \rightarrow \infty} \frac{K_{in}}{n}$ is the long-run
fraction of the time that X = i.  In other words, it's P(X = i)!
So,

\begin{equation}
E(X) = \sum_{i=0}^{10} i \cdot P(X = i)
\end{equation}

So in general we have: 

{\bf Property A:} 

The expected value of a discrete random variable X which takes values in
the set A is 

\begin{equation}
\label{a}
E(X) = \sum_{c \in A} c ~ P(X = c)
\end{equation}

Note that (\ref{a}) is the formula we'll use.  The preceding equations
were derivation, to motivate the formula.  Note too that (\ref{a}) is not
the {\it definition} of expected value; that was in \ref{longrunavg}.
It is quite important to distinguish between all of these, in terms of
goals.\footnote{The matter is made a little more confusing by the fact
that many books do in fact treat (\ref{a}) as the definition, with
(\ref{longrunavg}) being the consequence.}

By the way, note the word {\it discrete} above.  For the case of
continuous random variables, the sum in (\ref{a}) will become an
integral.

It will be shown in Section \ref{binom} that in our example above in
which X is the number of heads we get in 10 tosses of a coin,

\begin{equation}
P(X = i) =  \binom{10}{i} 0.5^i (1-0.5)^{10-i}  
\end{equation}

So

\begin{equation}
E(X) =  \sum_{i=0}^{10} i \binom{10}{i} 0.5^i (1-0.5)^{10-i}  
\end{equation}

It turns out that E(X) = 5.

For X in our dice example, 

\begin{equation}
E(X) = \sum_{c=1}^6 c \cdot \frac{1}{6} = 3.5
\end{equation}

It is customary to use capital letters for random variables, e.g. X
here, and lower-case letters for values taken on by a random variable,
e.g. c here.  Please adhere to this convention.  

By the way, it is also customary to write EX instead of E(X), whenever
removal of the parentheses does not cause any ambiguity.  An example in
which it would produce ambiguity is $E(U^2)$.  The expression $EU^2$
might be taken to mean either $E(U^2)$, which is what we want, or
$(EU)^2$, which is not what we want.

For S = X+Y in the dice example,

\begin{equation}
\label{es}
E(S) = 
2 \cdot \frac{1}{36} +
3 \cdot \frac{2}{36} +
4 \cdot \frac{3}{36} +
...
12 \cdot \frac{1}{36} = 7
\end{equation}

In the case of N, tossing a coin until we get a head:

\begin{equation}
E(N) = \sum_{c=1}^\infty c \cdot \frac{1}{2^c} =  2
\end{equation}

(We will not go into the details here concerning how the sum of this
particular infinite series is computed.  See Section \ref{geom}.)

Some people like to think of E(X) using a center of gravity analogy.
Forget that analogy!  Think notebook!  {\bf Intuitively, E(X) is the
long-run average value of X among all the lines of the notebook.}  So
for instance in our dice example, E(X) = 3.5, where X was the number of
dots on the blue die, means that if we do the experiment thousands of
times, with thousands of lines in our notebook, the average value of X
in those lines will be about 3.5.   With S = X+Y, E(S) = 7.  This means
that in the long-run average in column S in Table \ref{dicenotebook2} is
7.

\begin{table}
\begin{center}
\vskip 0.5in

\begin{tabular}{|l|l|l|l|}
\hline
notebook line & outcome & blue+yellow = 6? & S \\ \hline 
\hline
1 & blue 2, yellow 6 & No & 8 \\ \hline 
2 & blue 3, yellow 1 & No & 4 \\ \hline 
3 & blue 1, yellow 1 & No & 2 \\ \hline 
4 & blue 4, yellow 2 & Yes & 6 \\ \hline 
5 & blue 1, yellow 1 & No & 2 \\ \hline 
6 & blue 3, yellow 4 & No & 7 \\ \hline 
7 & blue 5, yellow 1 & Yes & 6 \\ \hline 
8 & blue 3, yellow 6 & No & 9 \\ \hline 
9 & blue 2, yellow 5 & No & 7 \\ \hline 
\end{tabular}

\end{center}
\caption{Expanded Notebook for the Dice Problem}
\label{dicenotebook2} 
\end{table}

Of course, by symmetry, E(Y) will be 3.5 too, where Y is the number of
dots showing on the yellow die.  That means we wasted our time
calculating in Equation (\ref{es}); we should have realized beforehand
that E(S) is $2 \times 3.5 = 7$. 

In other words: 

{\bf Property B:} 

For any random variables U and V, the expected value of a new random
variable D = U+V is the sum of the expected values of U and V:

\begin{equation}
\label{eofsum} 
E(U+V) = E(U) + E(V)
\end{equation}

Note carefully that U and V do NOT need to be independent random
variables for this relation to hold.  You should convince yourself of
this fact intuitively {\bf by thinking about the notebook notion.}
Say we look at 10000 lines of the notebook, which has columns for the
values of U, V and U+V.  It makes no difference whether we average U+V
in that column, or average U and V in their columns and then
add---either way, we'll get the same result.

While you are at it, use the notebook notion to convince yourself of the
following:

{\bf Properties C:}  

\begin{itemize}

\item For any random variable U and constant a, then

\begin{equation}
\label{eau}
E(aU) = a EU
\end{equation}

\item For random variables X and Y---not necessarily independent---and
constants a and b, we have

\begin{equation}
\label{aubv}
E(aX+bY) = a EX + b EY
\end{equation}

This follows by taking U = aX and V = bY in (\ref{eofsum}), and then
using (\ref{eau}).

By induction, for constants $a_1,...,a_k$ and random variables
$X_1,...,X_k$, form the new random variable 
$a_1 X_1 +...+ a_k X_k$.  Then

\begin{equation}
\label{elincomb}
E(a_1 X_1 +...+ a_n X_k) = 
a_1 EX_1 +...+ a_n EX_k
\end{equation}

\item For any constant {\it b}, we have

\begin{equation}
\label{eofconst}
E(b) = b
\end{equation}

\end{itemize}

For instance, say U is temperature in Celsius.  Then the temperature in
Fahrenheit is $W = \frac{9}{5} U + 32$.  So, W is a new random variable,
and we can get its expected value from that of U by using (\ref{aubv})
with $a = \frac{9}{5}$ and b = 32.

If you combine (\ref{eofconst}) with (\ref{aubv}), we have an important
special case:

\begin{equation}
\label{aub}
E(aX+b) = a EX + b
\end{equation}

Another important point:

{\bf Property D:} If U and V {\it are} independent, then

\begin{equation}
\label{factoring}
E(UV) = EU \cdot EV
\end{equation}


In the dice example, for instance, let D denote the product of the
numbers of blue dots and yellow dots, i.e. D = XY.  Then

\begin{equation}
E(D) = 3.5^2 = 12.25
\end{equation}

Equation (\ref{factoring}) doesn't have an easy ``notebook proof.'' It
is proved in Section \ref{theyfactor}.

Consider a function $g()$ of one variable, and let $W = g(X)$.  $W$ is then a
random variable too.  Say $X$ has support $A$, as in (\ref{a}).  Then $W$ has
support $B = \{g(c): c \epsilon A \}$.  (

For instance, say {\it g()} is the squaring function, and $X$ takes on
the values -1, 0 and 1, with probability 0.5, 0.4 and 0.1.  Then

\begin{equation}
A = \{ -1,0,1 \}
\end{equation}

and 

\begin{equation}
B = \{ 0,1 \}
\end{equation}

Define

\begin{equation}
A_d = \{c: c \in A, g(c) = d\}
\end{equation}

In our above squaring example, we will have

\begin{equation}
A_0 = \{ 0 \}, ~~ A_1 = \{ -1,1 \}
\end{equation}

Then 

\begin{equation}
P(W = d) = P(X \in A_d) 
\end{equation}

so

\begin{eqnarray}
E[g(X)] &=& E(W) \\
&=& \sum_{d \in B} d P(W = d) \\ 
&=& \sum_{d \in B} d \sum_{c \in A_d} P(X = c) \\
&=& \sum_{c \in A} g(c) P(X = c) 
\end{eqnarray}

(Going from the next-to-last equation here to the last one is rather
tricky.  Work through for the case of our squaring function example
above in order to see why the final equation does follow.

{\bf Property E:}

If E[g(X)] exists, then

\begin{equation}
\label{egofx}
E[g(X)] = \sum_{c \in A} g(c) \cdot P(X = c)
\end{equation}

where the sum ranges over all values c that can be taken on by X.

For example, suppose for some odd reason we are interested in finding
$E(\sqrt{X})$, where {\bf X} is the number of dots we get when we roll
one die.  Let $W = \sqrt{X})$.  Then {\bf W} is another random variable,
and is discrete, since it takes on only a finite number of values.  (The
fact that most of the values are not integers is irrelevant.)  We want
to find $EW$.

Well, $W$ is a function of X, with $g(t) = \sqrt{t}$.  So,
(\ref{egofx}) tells us to make a list of values in the support
of W, i.e.  $\sqrt{1}, \sqrt{2}, ..., \sqrt{6}$, and a list of the
corresponding probabilities for {\bf X}, which are all $\frac{1}{6}$.
Substituting into (\ref{egofx}), we find that

\begin{equation}
E(\sqrt{X}) = \frac{1}{6} \sum_{i=1}^6 \sqrt{i}
\end{equation}

What about a function of several variables?  Say for instance 
you are finding E(UV), where U has support, say, 1,2 and V has
support 5,12,13.  In order to find E(UV), you need to know the support
of UV, recognizing that it, the product UV, is a new random variable in
its own right.  Let's call it W.  Then in this little example, W has
support 5,12,13,10,24,26.  Then compute

$$
5 P(W = 5) + 12 P(W = 12) + ...
= 5 P(U = 1, V = 5) + 12 P(U = 1, V = 12) + ...
$$

{\bf Note:}  Equation (\ref{egofx}) will be one of the most heavily used
formulas in this book.  Make sure you keep it in mind.

\subsection{``Mailing Tubes''}
\label{mailingtubes2}

{\bf The properties of expected value discussed above are key to the
entire remainder of this book.  You should notice immediately when you
are in a setting in which they are applicable.  For instance, if you see
the expected value of the sum of two random variables, you should
instinctively think of (\ref{eofsum}) right away.}  

As discussed in Section \ref{mailingtubes1}, these properties are
``mailing tubes.'' For instance, (\ref{eofsum}) is a ``mailing
tube''--make a mental note to yourself saying, ``If I ever need to find
the expected value of the sum of two random variables, I can use
(\ref{eofsum}).''  Similarly, (\ref{egofx}) is a mailing tube;
tell yourself, ``If I ever see a new random variable that is a function
of one whose probabilities I already know, I can find the expected value
of the new random variable using (\ref{egofx}).''

You will encounter ``mailing tubes'' throughout this book.  For
instance, (\ref{varcu}) below is a very important ``mailing tube.''
Constantly remind yourself---``Remember the `mailing tubes'!''

\subsection{Casinos, Insurance Companies and ``Sum Users,'' Compared
to Others}

The expected value is intended as a {\bf measure of central tendency}
(also called a {\bf measure of location}, i.e.  as some sort of
definition of the probablistic ``middle'' in the range of a random
variable.  There are various other such measures one can use, such as
the {\bf median}, the halfway point of a distribution, and today they
are recognized as being superior to the mean in certain senses.  For
historical reasons, the mean plays an absolutely central role in
probability and statistics.  Yet one should understand its limitations.
(This discussion will be general, not limited to discrete random
variables.)

({\bf Warning:}  The concept of the mean is likely so ingrained in your
consciousness that you simply take it for granted that you know what the
mean means, no pun intended.  But try to take a step back, and think of
the mean afresh in what follows.)

First, the term {\it expected value} itself is a misnomer.  We do not
\underline{expect} the number of dots D to be 3.5 in the die example in
Section \ref{whatisit}; in fact, it is impossible for W to take on that
value.  

Second, the expected value is what we call the {\bf mean} in everyday
life.  And the mean is terribly overused.  Consider, for example, an
attempt to describe how wealthy (or not) people are in the city of
Davis.  If suddenly Bill Gates were to move into town, that would skew
the value of the mean beyond recognition.  

But even without Gates, there is a question as to whether the mean has
that much meaning.  After all, what is so meaningful about summing our
data and dividing by the number of data points?  The median has an easy
intuitive meaning, but although the mean has familiarity, one would be
hard pressed to justify it as a measure of central tendency.

What, for example, does Equation (\ref{longrunavg}) mean in the context
of people's heights in Davis?  We would sample a person at random and
record his/her height as $X_1$.  Then we'd sample another person, to get
$X_2$, and so on.  Fine, but in that context, what would
(\ref{longrunavg}) mean?  The answer is, not much.  So the significance
of the mean height of people in Davis would be hard to explain.

For a casino, though, (\ref{longrunavg}) means plenty.  Say X is the
amount a gambler wins on a play of a roulette wheel, and suppose
(\ref{longrunavg}) is equal to \$1.88.  Then after, say, 1000 plays of
the wheel (not necessarily by the same gambler), the casino knows from
\ref{longrunavg} it will have paid out a total of about \$1,880.  So
if the casino charges, say \$1.95 per play, it will have made a profit
of about \$70 over those 1000 plays.  It might be a bit more or less
than that amount, but the casino can be pretty sure that it will be
around \$70, and they can plan their business accordingly.

The same principle holds for insurance companies, concerning how much
they pay out in claims.  With a large number of customers, they know
(``expect''!) approximately how much they will pay out, and thus can
set their premiums accordingly.  Here the mean has a tangible, practical
meaning.

The key point in the casino and insurance companies examples is that
they are interested in {\it totals}, such as {\it total} payouts on a
blackjack table over a month's time, or {\it total} insurance claims
paid in a year.  Another example might be the number of defectives in a
batch of computer chips; the manufacturer is interested in the {\it
total} number of defectives chips produced, say in a month.  Since the
mean is by definition a {\it total} (divided by the number of data
points), the mean will be of direct interest to casinos etc.

By contrast, in describing how wealthy people of a town are, the total
height of all the residents is not relevant.  Similarly, in describing
how well students did on an exam, the sum of the scores of all the
students doesn't tell us much.  (Unless the professor gets \$10 for each
point in the exam scores of each of the students!) A better description
for heights and exam scores might be the median height or score.

Nevertheless, the mean has certain mathematical properties, such as
(\ref{eofsum}), that have allowed the rich development of the fields of
probability and statistics over the years.  The median, by contrast,
does not have nice mathematical properties.  In many cases, the mean
won't be too different from the median anyway (barring Bill Gates moving
into town), so you might think of the mean as a convenient substitute
for the median.  The mean has become entrenched in statistics, and we
will use it often.

\section{Variance}
\label{variance}

As in Section \ref{expval}, the concepts and properties introduced in
this section form the very core of probability and statistics.  {\bf
Except for some specific calculations, these apply to both discrete and
continuous random variables.}

\subsection{Definition}

While the expected value tells us the average value a random variable
takes on, we also need a measure of the random variable's
variability---how much does it wander from one line of the notebook to
another?  In other words, we want a measure of {\bf dispersion}.  The
classical measure is {\bf variance}, defined to be the mean squared
difference between a random variable and its mean:

\begin{definition}  For a random variable U for which the expected
values written below exist, the {\bf variance} of U is defined to be

\begin{equation}
\label{vardef}
Var(U) = E[(U-EU)^2]
\end{equation}

\end{definition}

For X in the die example, this would be

\begin{equation}
\label{dievar} 
Var(X) = E[(X-3.5)^2]
\end{equation}

Remember what this means:  We have a random variable {\bf X}, and we're
creating a new random variable, $W = (X-3.5)^2$, which is a function of
the old one.  We are then finding the expected value of that new random
variable W.

In the notebook view, $E[(X-3.5)^2]$ is the long-run average of the W
column:

\vskip 0.5in
\begin{tabular}{|r|r|r|}
\hline
line & X & W \\ \hline 
\hline
1 & 2 & 2.25 \\ \hline 
2 & 5 & 2.25 \\ \hline 
3 & 6 & 6.25 \\ \hline 
4 & 3 & 0.25 \\ \hline 
5 & 5 & 2.25 \\ \hline 
6 & 1 & 6.25 \\ \hline 
\end{tabular}

To evaluate this, apply (\ref{egofx}) with $g(c) = (c-3.5)^2$:

\begin{equation}
Var(X) = \sum_{c=1}^6 (c-3.5)^2 \cdot \frac{1}{6} = 2.92
\end{equation}

You can see that variance does indeed give us a measure of dispersion.
In the expression $Var(U) = E[(U-EU)^2]$, if the values of U are mostly
clustered near its mean, then $(U-EU)^2$ will usually be small, and thus
the variance of U will be small; if there is wide variation in U, the
variance will be large.

% The properties of E() in (\ref{eofsum}) and (\ref{aubv}) can be used
% to show:

{\bf Property F:}

\begin{equation} 
\label{varuformula}
Var(U) = E(U^2) - (EU)^2
\end{equation}

The term $E(U^2)$ is again evaluated using (\ref{egofx}). 

Thus for example, if X is the number of dots which come up when we roll
a die.   Then, from (\ref{varuformula}), 

\begin{equation}
Var(X) = E(X^2) - (EX)^2
\end{equation}

Let's find that first term (we already know the second is $3.5^2$).  From
(\ref{egofx}),

\begin{equation}
E(X^2) =  \sum_{i=1}^6 i^2 \cdot \frac{1}{6} = \frac{91}{6}
\end{equation}

Thus $Var(X) = E(X^2) - (EX)^2 = \frac{91}{6} - 3.5^2$

Remember, though, that (\ref{varuformula}) is a shortcut formula for
finding the variance, not the {\it definition} of variance.

Below is the derivation of (\ref{varuformula}).  Keep in mind that EU is
a constant.

\begin{eqnarray}
Var(U) &=& E[(U-EU)^2] \\ 
&=& E[U^2 -2EU \cdot U + (EU)^2] ~~ \textrm{(algebra)} \\
&=& E(U^2) + E(-2 EU \cdot U) + E[(EU)^2] ~~ (\ref{eofsum}) \\
&=& E(U^2) -2 EU \cdot EU + (EU)^2 ~~ (\ref{eau}), (\ref{eofconst}) \\
&=& E(U^2) - (EU)^2
\end{eqnarray}

An important behavior of variance is: 

{\bf Property G:}

\begin{equation}
\label{varcu}
Var(cU) = c^2 Var(U)
\end{equation}

for any random variable U and constant c. It should make sense to you:
If we multiply a random variable by 5, say, then its average squared
distance to its mean should increase by a factor of 25. 

Let's prove (\ref{varcu}).  Define V = cU.  Then

\begin{eqnarray}
Var(V) &=& E[(V-EV)^2] \textrm{  (def.)} \\ 
&=& E\{[cU - E(cU)]^2\} \textrm{  (subst.)} \\
&=& E\{[cU - cEU]^2\} \textrm{  ((\ref{aubv}))} \\
&=& E\{c^2 [U - EU]^2\} \textrm{  (algebra)} \\
&=& c^2 E\{[U - EU]^2\} \textrm{  ((\ref{aubv}))} \\
&=& c^2 Var(U) \textrm{  (def.)}
\end{eqnarray}

Shifting data over by a constant does not change the amount of variation
in them:

{\bf Property H:}

\begin{equation}
\label{affinevar}
Var(U+d) = Var(U)
\end{equation}

for any constant d.

Intuitively, the variance of a constant is 0---after all, it never
varies!  You can show this formally using (\ref{varuformula}):

\begin{equation}
Var(c) = E(c^2) - [E(c)]^2 = c^2 - c^2 = 0
\end{equation}

The square root of the variance is called the {\bf standard deviation}.

Again, we use variance as our main measure of dispersion for historical
and mathematical reasons, not because it's the most meaningful measure.
The squaring in the definition of variance produces some distortion, by
exaggerating the importance of the larger differences.  It would be more
natural to use the {\bf mean absolute deviation} (MAD), $E(|U-EU|)$.
However, this is less tractable mathematically, so the statistical
pioneers chose to use the mean squared difference, which lends itself to
lots of powerful and beautiful math, in which the Pythagorean Theorem
pops up in abstract vector spaces.  (See Section \ref{elegant} for
details.)

{\bf As with expected values, the properties of variance discussed
above, and also in Section \ref{covar} below, are key to the entire
remainder of this book.  You should notice immediately when you are in a
setting in which they are applicable.  For instance, if you see the
variance of the sum of two random variables, you should instinctively
think of (\ref{varsum}) right away, and check whether they are
independent.}

\subsection{More Practice with the Properties of Variance}

Suppose $X$ and $Y$ are independent random variables, with $EX = 1$, $EY
= 2$, $Var(X) = 3$ and $Var(Y) = 4$.  Let's find $Var(XY)$.  (The reader
should make sure to supply the reasons for each step, citing equation
numbers from the material above.)

\begin{eqnarray}
Var(XY) &=& E(X^2 Y^2) - [E(XY)]^2 \\
&=& E(X^2) \cdot E(Y^2) - (EX \cdot EY)^2 \\ 
&=& [Var(X) + (EX)^2] \cdot
[Var(Y) + (EY)^2] - (EX \cdot EY)^2 \label{genform} \\
&=& (3 + 1^2) (4 + 2^2) - (1 \cdot 2)^2 \\
&=& 28
\end{eqnarray}

\subsection{Central Importance of the Concept of Variance}

No one needs to be convinced that the mean is a fundamental descriptor
of the nature of a random variable.  But the variance is of central
importance too, and will be used constantly throughout the remainder of
this book.

The next section gives a quantitative look at our notion of variance as
a measure of dispersion.  

\subsection{Intuition Regarding the Size of Var(X)}

{\it A billion here, a billion there, pretty soon, you're talking real
money}---attributed to the late Senator Everitt Dirksen, replying to a
statement that some federal budget item cost ``only'' a billion dollars

\bigskip

Recall that the variance of a random variable X is supposed to be a
measure of the dispersion of X, meaning the amount that X varies from
one instance (one line in our notebook) to the next.  But if Var(X) is,
say, 2.5, is that a lot of variability or not?  We will pursue this
question here.

\subsubsection{Chebychev's Inequality}
\label{chebsection}

This inequality states that for a random variable X with mean $\mu$ and
variance $\sigma^2$, 

\begin{equation}
\label{cheb}
P(|X - \mu| \geq c \sigma) \leq \frac{1}{c^2}
\end{equation}

In other words, X strays more than, say, 3 standard deviations from its
mean at most only 1/9 of the time.  This gives some concrete meaning to
the concept of variance/standard deviation.

You've probably had exams in which the instructor says something like
``An A grade is 1.5 standard deviations above the mean.''  Here c in
(\ref{cheb}) would be 1.5.

We'll prove the inequality in Section \ref{chebproof}.

\subsubsection{The Coefficient of Variation}

Continuing our discussion of the magnitude of a variance, look at our
remark following (\ref{cheb}):

\begin{quote}
In other words, X does not often stray more than, say, 3 standard
deviations from its mean.  This gives some concrete meaning to the
concept of variance/standard deviation.
\end{quote}

Or, think of the price of, say, widgets.  If the price hovers around a
\$1 million, but the variation around that figure is only about a
dollar, you'd say there is essentially no variation.  But a variation
of about a dollar in the price of a hamburger would be a lot.

These considerations suggest that any discussion of the size of Var(X)
should relate to the size of E(X).  Accordingly, one often looks at the
{\bf coefficient of variation}, defined to be the ratio of the standard
deviation to the mean:

\begin{equation}
\textrm{coef. of var.} = \frac{\sqrt{Var(X)}}{EX}
\end{equation}

This is a scale-free measure (e.g. inches divided by inches), and serves
as a good way to judge whether a variance is large or not.

\section{A Useful Fact}
\label{usefulfact}

\label{mingcpage}
For a random variable X, consider the function

\begin{equation}
g(c) = E[(X-c)^2]
\end{equation}

Remember, the quantity $E[(X-c)^2]$ is a number, so g(c) really is a
function, mapping a real number c to some real output.  

We can ask the question, What value of c minimizes g(c)?  To answer that
question, write:

\begin{equation}
\label{gofc}
g(c) = E[(X-c)^2] = E(X^2 -2cX + c^2) = E(X^2) - 2c EX + c^2
\end{equation}

where we have used the various properties of expected value derived in
recent sections.

To make this concrete, suppose we are guessing people's
weights---without seeing them and without knowing anything about them at
all.  (This is a somewhat artificial question, but it will become highly
practical in Chapter \ref{linreg}.)  Since we know nothing at all about
these people, we will make the same guess for each of them.

What should that guess-in-common be?  Your first inclination would be to
guess everyone to be the mean weight of the population.  If that value
in our target population is, say, 142.8 pounds, then we'll guess
everyone to be that weight.  Actually, that guess turns out to be
optimal in a certain sense, as follows.

Say $X$ is a person's weight.  It's a random variable, because these
people are showing up at random from the population.  Then $X-c$ is our
prediction error.  How well will do in our predictions?  We can't
measure that as

\begin{equation}
E(\textrm{error}) 
\end{equation}

because that quantity is 0!  (What mailing tube is at work here?)

A reasonable measure would be

\begin{equation}
\label{minabs}
E(|X - c|) 
\end{equation}

However, due to tradition, we use

\begin{equation}
\label{minsq}
E[(X - c)^2] 
\end{equation}

Now differentiate with respect to c, and set the result to 0.
Remembering that $E(X^2)$ and EX are constants, we have

\begin{equation}
0 = -2 EX + 2c
\end{equation}

so the minimizing c is c = EX!  

In other words, the minimum value of $E[(X-c)^2]$ occurs at c = EX.  Our
intuition was right!

Moreover:  Plugging c = EX into (\ref{gofc}) shows that the minimum value
of g(c) is $E(X-EX)^2]$ , which is Var(X)!

In notebook terms, think of guessing many, many people, meaning many
lines in the notebook, one per person.  Then (\ref{minsq}) is the
long-run average squared error in our guesses, and we find that we
minimize that by guessing everyone's weight to be the population mean
weight.

But why look at average squared error?  It accentuates the large errors.
Instead, we could minimize (\ref{minabs}).  It turns out that the best
$c$ here is the population {\it median} weight.

\section{Covariance} 

This is a topic we'll cover fully in Chapter \ref{randvec}, but at least
introduce here.

A measure of the degree to which U and V vary together is their {\bf
covariance},

\begin{equation}
\label{covdef1}
Cov(U,V) = E[(U-EU)(V-EV)]
\end{equation}

Except for a divisor, this is essentially {\bf correlation}.  If U is
usually large (relative to its expectation) at the same time V is
small (relative to its expectation), for instance, then you can
see that the covariance between them will be negative.  On the other
hand, if they are usually large together or small together, the
covariance will be positive.

For example, suppose U and V are the height and weight, respectively, of
a person chosen at random from some population, and think in notebook
terms.  Each line shows the data for one person, and we'll have columns
for U, V, U - EU, V - EV and (U - EU) (V - EV).  Then (\ref{covdef1}) is
the long-run average of that last column.  Will it be positive or
negative?  Reason as follows:

Think of the lines in the notebook for people who are taller than
average, i.e.\ for whom $U - EU > 0$.  Most such people are also heavier
than average, i.e.\ $V - EV > 0$, so that $(U - EU) (V - EV) > 0$.  
On the other hand, shorter people also tend to be lighter, so most lines
with shorter people will have $U - EU <0$ and $V -EV < 0$---but still
$(U - EU) (V - EV) > 0$.  In other words, the long-run average of the (U
- EU) (V - EV) column will be positive.

The point is that, if two variables are positively related, e.g.\ height
and weight, their covariance should be positive.  This is the intuitive
underlying defining covariance as in (\ref{covdef1}).


Again, one can use the properties of E() to show that

\begin{equation}
\label{covshortcut}
Cov(U,V) = E(UV) - EU \cdot EV
\end{equation}

Again, this will be derived fully in Chapter \ref{chap:randvec}, but you
think about how to derive it yourself.  Just use our old mailing tubes,
e.g. E(X+Y) = EX + EY, E(cX) for a constant c, etc.  Note that EU and EV
are constants!

Also 

\begin{equation}
\label{genvarsum}
Var(U+V) = Var(U) + Var(V) + 2 Cov(U,V)
\end{equation}

and more generally,

\begin{equation}
\label{covabuv}
Var(aU+bV) = a^2 Var(U) + b^2 Var(V) + 2ab Cov(U,V)
\end{equation}

for any constants $a$ and $b$.

(\ref{covshortcut}) imply that Cov(U,V) = 0.  In that case,

\begin{equation}
\label{varsum}
Var(U+V) = Var(U) + Var(V) 
\end{equation}

By the way, (\ref{varsum}) is actually the Pythagorean Theorem in a
certain esoteric, infinite-dimesional vector space (related to a similar
remark made earlier).  This is pursued in Section \ref{elegant} for
the mathematically inclined.

Generalizing (\ref{covabuv}), for constants $a_1,...,a_k$ and random
variables $X_1,...,X_k$, form the new random variable $a_1 X_1 +...+ a_k
X_k$.  Then

\begin{equation}
Var(a_1 X_1 +...+ a_k X_k) = 
\sum_{i=1}^k a_i^2 Var(X_i) + 2 \sum_{1 \leq i < j \leq k}^k
Cov(X_i,X_j)
\end{equation}

If the $X_i$ are independent, then we have the special case

\begin{equation}
Var(a_1 X_1 +...+ a_k X_k) = 
\sum_{i=1}^k a_i^2 Var(X_i) 
\end{equation}

\section{Indicator Random Variables, and Their Means and Variances}
\label{indicator}

\begin{definition}

A random variable that has the value 1 or 0, according to whether a
specified event occurs or not is called an {\bf indicator random
variable} for that event.  

\end{definition}

You'll often see later in this book that the notion of an indicator
random variable is a very handy device in certain derivations.  But for
now, let's establish its properties in terms of mean and variance.

\begin{quote}

{\bf Handy facts:}  Suppose X is an indicator random variable for the
event A.  Let p denote P(A).  Then 

\begin{equation}
\label{eofxeqp}
E(X) = p
\end{equation}

\begin{equation}
\label{varofeqp1p}
Var(X) = p(1-p)
\end{equation}

\end{quote}

These two facts are easily derived.  In the first case we have, using our
properties for expected value,

\begin{equation}
EX = 1 \cdot P(X = 1) + 0 \cdot P(X = 0) = P(X = 1) = P(A) = p
\end{equation}

The derivation for Var(X) is similar (use (\ref{varuformula})).

For example, say Coin A has probability 0.6 of heads, Coin B is fair, and
Coin C has probability 0.2 of heads.  I toss A once, getting X heads,
then toss B once, getting Y heads, then toss C once, getting Z heads.
Let W = X + Y + Z, i.e. the total number of heads from the three tosses
(W ranges from 0 to 3).  Let's find P(W = 1) and Var(W).

The first one uses old methods:

\begin{eqnarray}
P(W = 1) &=&
   P(X = 1 \textrm{ and }  Y = 0 \textrm{ and }  Z = 0 \textrm{ or }  ...) \\
   &=& 0.6 \cdot 0.5 \cdot 0.8 +
       0.4 \cdot 0.5 \cdot 0.8 +
       0.4 \cdot 0.5 \cdot 0.2
\end{eqnarray}

For Var(W), let's use what we just learned about indicator random
variables; each of X, Y and Z are such variables.  Var(W) = Var(X) +
Var(Y) + Var(Z), by independence and (\ref{varsum}).  Since X is an
indicator random variable, $Var(X) = 0.6 \cdot 0.4$, etc.  The answer is
then

\begin{equation}
0.6 \cdot 0.4 +
0.5 \cdot 0.5 +
0.2 \cdot 0.8
\end{equation}

\subsection{Example:  Return Time for Library Books, Version I}

Suppose at some public library, patrons return books
exactly 7 days after borrowing them, never early or late.  However, they
are allowed to return their books to another branch, rather than the
branch where they borrowed their books.  In that situation, it takes 9
days for a book to return to its proper library, as opposed to the
normal 7.  Suppose 50\% of patrons return their books to a ``foreign''
library.  Find Var(T), where T is the time, either 7 or 9 days, for a
book to come back to its proper location.

Note that

\begin{equation}
T = 7 + 2 I, 
\end{equation}

where I is an indicator random variable for the event that the book is
returned to a ``foreign'' branch.  Then

\begin{equation}
Var(T) = Var(7 + 2I) =  4 Var(I) = 4 \cdot 0.5 (1 - 0.5)
\end{equation}

\subsection{Example:  Return Time for Library Books, Version II}

Now let's look at a somewhat more general model.  Here we will assume
that borrowers return books after 4, 5, 6 or 7 days, with probabilities
0.1, 0.2, 0.3, 0.4, respectively. As before, 50\% of patrons return
their books to a ``foreign'' branch, resulting in an extra 2-day delay
before the book arrives back to its proper location.  The library is
open 7 days a week.

Suppose you wish to borrow a certain book, and inquire at the library
near the close of business on Monday. Assume too that no one else is
waiting for the book.  You are told that it had been checked out the
previous Thursday. Find the probability that you will need to wait until
Wednesday evening to get the book. (You check every evening.)

Let B denote the time needed for the book to arrive back at its home
branch, and define I as before.  Then

\begin{eqnarray}
P(B = 6 ~|~ B > 4) &=& \frac{P(B = 6 \textrm{ and } B > 4)}{P(B > 4)}\\ 
&=& \frac{P(B = 6)}{P(B > 4)} \\
&=& \frac{P(B = 6 \textrm{ and I = 0 \textrm{ or } B = 6 \textrm{ and }
I = 1})}
{1 - P(B = 4)} \\
&=& \frac{0.5 \cdot 0.3 + 0.5 \cdot 0.1}
{1 - 0.5 \cdot 0.1} \\
&=& \frac{4}{19}
\end{eqnarray}

Here is a simulation check:

\begin{lstlisting}
libsim <- function(nreps) {
   # patron return time
   prt <- sample(c(4,5,6,7),nreps,replace=T,prob=c(0.1,0.2,0.3,0.4))
   # indicator for foreign branch
   i <- sample(c(0,1),nreps,replace=T)
   b <- prt + 2*i
   x <- cbind(prt,i,b)
   # look only at the relevant notebook lines
   bgt4 <- x[b > 4,]
   # among those lines, what proportion have B = 6?
   mean(bgt4[,3] == 6)
}                               
\end{lstlisting}

Note that in this simulation. all {\bf nreps} values of $I$, $B$ and the
patron return time are generated first.  This uses more memory space
(though not an issue in this small problem), but makes things easier to
code, as we can exploit R's vector operations.  Those not only are more
convenient, but also faster running.

\subsection{Example:  Indicator Variables in a Committee Problem}
\label{combex}

A committee of four people is drawn at random from a set of six men and
three women.  Suppose we are concerned that there may be quite a gender
imbalance in the membership of the committee.  Toward that end, let M
and W denote the numbers of men and women in our committee, and let D =
M-W.  Let's find E(D), in two different ways.

D has support consisting of the values 4-0, 3-1, 2-2 and 1-3, i.e. 4, 2,
0 and -2.  So from (\ref{a})

\begin{equation}
\label{ed}
ED = 
-2 \cdot P(D = -2) +
0 \cdot P(D = 0) +
2 \cdot P(D = 2) +
4 \cdot P(D = 4) 
\end{equation}

Now, using reasoning along the lines in Section \ref{comb}, we have

\begin{equation}
P(D = -2) = P(M = 1 \textrm{ and } W = 3) =
\frac
{\binom{6}{1} \binom{3}{3}}
{\binom{9}{4}} 
\end{equation}

After similar calculations for the other probabilities in (\ref{ed}),
we find the $ED = \frac{4}{3}$.  

Note what this means: If we were to perform this experiment many times,
i.e. choose committees again and again, on average we would have a
little more than one more man than women on the committee.

Now let's use our ``mailing tubes'' to derive ED a different way:

\begin{eqnarray}
ED &=& E(M - W) \\ 
&=& E[ M - (4-M)] \\
&=& E(2 M - 4) \\
&=& 2 EM - 4 ~~ (\textrm{from } (\ref{aubv})) 
\end{eqnarray}

Now, let's find EM by using indicator random variables.  Let $G_i$
denote the indicator random variable for the event that the i$^{th}$
person we pick is male, i = 1,2,3,4.  Then

\begin{equation}
\label{massum}
M = G_1 + G_2 + G_3 + G_4
\end{equation}

so 

\begin{eqnarray}
EM &=& E(G_1 + G_2 + G_3 + G_4) \\ 
&=& EG_1 + EG_2 + EG_3 + EG_4 ~~  [\textrm{ from } (\ref{eofsum})] \\
&=& P(G_1 = 1) + P(G_2 = 1) + P(G_3 = 1) + P(G_4 = 1) ~~ [\textrm{ from } (\ref{eofxeqp})]
\label{4g}
\end{eqnarray}

Note carefully that the second equality here, which uses (\ref{eofsum}),
is true in spite of the fact that the $G_i$ are not independent.
Equation (\ref{eofsum}) does \underline{not} require independence.

Another key point is that, due to symmetry, $P(G_i = 1)$ is the same for
all i.  Note that we did not write a {\it conditional} probability
here!  Once again, think of the notebook view:  {\large \bf By
definition}, $(P(G_2 = 1)$ is the long-run proportion of the number of
notebook lines in which $G_2 = 1$---regardless of the value of $G_1$ in
those lines.

Now, to see that $P(G_i = 1)$ is the same for all i, suppose the six men
that are available for the committee are named Alex, Bo, Carlo, David,
Eduardo and Frank.  When we select our first person, any of these men
has the same chance of being chosen (1/9).  {\it But that is also true
for the second pick.}  Think of a notebook, with a column named ``second
pick.''  In some lines, that column will say Alex, in some it will say
Bo, and so on, and in some lines there will be women's names.  But in
that column, Bo will appear the same fraction of the time as Alex, due
to symmetry, and that will be the same fraction as for, say, Alice,
again 1/9.

Now,

\begin{equation}
P(G_1 = 1) = \frac{6}{9} = \frac{2}{3}
\end{equation}

Thus 

\begin{equation}
ED = 2 \cdot (4 \cdot \frac{2}{3}) -4  = \frac{4}{3}
\end{equation}

\subsection{Example:  Spinner Game}

In a certain game, Person A spins a spinner and wins $S$ dollars, with
mean 10 and variance 5.  Person B flips a coin.  If it comes up heads,
Person A must give B whatever A won, but if it comes up tails, B wins
nothing.  Let T denote the amount B wins.  Let's find $Var(T)$.

We can use (\ref{genform}), in this case with X = I, where I is an
indicator variable for the event that B gets a head, and with Y = S.
Then $T = I \cdot S$, and I and S are independent, so

\begin{equation}
Var(T) = Var(IS) = 
[Var(I) + (EI)^2] \cdot
[Var(S) + (ES)^2] - (EI \cdot ES)^2 
\end{equation}

Then use the facts that I has mean 0.5 and variance 0.5(1-0.5)
(Equations (\ref{eofxeqp}) and (\ref{varofeqp1p}), with S having the
mean 10 and variance 5, as given in the problem.

\section{Expected Value, Etc. in the ALOHA Example}

Finding expected values etc. in the ALOHA example is straightforward.
For instance, 

\begin{equation}
EX_1 = 0 \cdot P(X_1 = 0) + 1 \cdot P(X_1 = 1) + 2 \cdot P(X_1 = 2)
= 1 \cdot 0.48 + 2 \cdot 0.52 = 1.52
\end{equation}

Here is R code to find various values approximately by simulation:

\begin{Verbatim}[fontsize=\relsize{-2},numbers=left]
# finds E(X1), E(X2), Var(X2), Cov(X1,X2)
sim <- function(p,q,nreps) {
   sumx1 <- 0
   sumx2 <- 0
   sumx2sq <- 0
   sumx1x2 <- 0
   for (i in 1:nreps) {
      numtrysend <- 
         sum(sample(0:1,2,replace=TRUE,prob=c(1-p,p)))
      if (numtrysend == 1)  X1 <- 1
      else X1 <- 2
      numactive <- X1
      if (X1 == 1 && runif(1) < q) numactive <- numactive + 1
      if (numactive == 1)
         if (runif(1) < p) X2 <- 0
         else X2 <- 1
      else {  # numactive = 2
         numtrysend <- 0
         for (i in 1:2)
            if (runif(1) < p) numtrysend <- numtrysend + 1
         if (numtrysend == 1) X2 <- 1
         else X2 <- 2
      }
      sumx1 <- sumx1 + X1
      sumx2 <- sumx2 + X2
      sumx2sq <- sumx2sq + X2^2
      sumx1x2 <- sumx1x2 + X1*X2
   }
   # print results
   meanx1 <- sumx1 /nreps
   cat("E(X1):",meanx1,"\n")
   meanx2 <- sumx2 /nreps
   cat("E(X2):",meanx2,"\n")
   cat("Var(X2):",sumx2sq/nreps - meanx2^2,"\n")
   cat("Cov(X1,X2):",sumx1x2/nreps - meanx1*meanx2,"\n")
}
\end{Verbatim}

As a check on your understanding so far, you should find at least one of
these values by hand, and see if it jibes with the simulation output.

\section{Example:  Measurements at Different Ages}

Say a large research program measures boys' heights at age
10 and age 15.  Call the two heights X and Y.  So, each boy has an X
and a Y.  Each boy is a ``notebook line'', and the notebook has two
columns, for X and Y.  We are interested in Var(Y-X).  Which of the
following is true?

\begin{itemize}

\item [(i)] $Var(Y-X) = Var(Y) + Var(X)$
\item [(ii)] $Var(Y-X) = Var(Y) - Var(X)$
\item [(iii)] $Var(Y-X) < Var(Y) + Var(X)$
\item [(iv)] $Var(Y-X) < Var(Y) - Var(X)$
\item [(v)] $Var(Y-X) > Var(Y) + Var(X)$
\item [(vi)] $Var(Y-X) > Var(Y) - Var(X)$
\item [(vii)] None of the above.

\end{itemize}

Use the mailing tube (\ref{covabuv}):

\begin{equation}
Var(Y-X) = Var[Y+(-X)] = 
Var(Y) + Var(X) - 2Cov(X,Y)
\end{equation}

Since X and Y are positively correlated, their covariance is positive,
so the answer is (iii).

\section{Example:  Bus Ridership Model}

In the bus ridership model, Section \ref{busridership}, let's find
$Var(L_1)$:

\begin{equation}
Var(L_1) = E(L^2_1) - (EL_1)^2
\end{equation}

\begin{equation}
EL_1 = EB_1 = 0 \cdot 0.5 + 1 \cdot 0.4 + 2 \cdot 0.1
\end{equation}

\begin{equation}
E(L^2_1) = 0^2 \cdot 0.5 + 1^2 \cdot 0.4 + 2^2 \cdot 0.1
\end{equation}

Then put it all together.

\section{Distributions}
\label{dstrdef}

The idea of the {\bf distribution} of a random variable is central to
probability and statistics.

\begin{definition}
Let U be a discrete random variable.  Then the distribution of U is 
simply the support of U, together with  the associated probabilities. 
\end{definition}

{\bf Example:}  Let X denote the number of dots one gets in rolling a
die.  Then the values X can take on are 1,2,3,4,5,6, each with
probability 1/6.  So

\begin{equation}
\textrm{distribution of } X =
\{
(1,\frac{1}{6}),
(2,\frac{1}{6}),
(3,\frac{1}{6}),
(4,\frac{1}{6}),
(5,\frac{1}{6}),
(6,\frac{1}{6})
\}
\end{equation}

{\bf Example:}  Recall the ALOHA example. There $X_1$ took on the values
1 and 2, with probabilities 0.48 and 0.52, respectively (the case of 0
was impossible).  So, 

\begin{equation}
\textrm{distribution of } X_1 =
\{
(0,0.00),
(1,0.48),
(2,0.52)
\}
\end{equation}

{\bf Example:}  Recall our example in which N is the number of tosses of a
coin needed to get the first head.  N has support 1,2,3,..., the 
probabilities of which we found earlier to be 1/2, 1/4, 1/8,...  So, 

\begin{equation}
\textrm{distribution of } N =
\label{geomdie}
\{
(1,\frac{1}{2}),
(2,\frac{1}{4}),
(3,\frac{1}{8}),
...
\}
\end{equation}

It is common to express this in functional notation:  

\begin{definition}
The {\bf probability mass function} (pmf) of a 
discrete random variable V, denoted $p_V$, as

\begin{equation}
p_V(k) = P(V = k)
\end{equation}

for any value k in the support of V.
\end{definition}

(Please keep in mind the notation.  It is customary to use the
lower-case p, with a subscript consisting of the name of the random
variable.)

Note that $p_V()$ is just a function, like any function (with integer
domain) you've had in your previous math courses.  For each input value,
there is an output value.

\subsection{Example:  Toss Coin Until First Head}

In (\ref{geomdie}),

\begin{equation}
\label{geompre}
p_N(k) = \frac{1}{2^k}, k = 1,2,...
\end{equation}

\subsection{Example:  Sum of Two Dice}

In the dice example, in which S = X+Y,

\begin{equation}
\label{dicesum}
p_S(k) = 
\begin{cases}
\frac{1}{36}, & k = 2 \cr
\frac{2}{36}, & k = 3 \cr
\frac{3}{36}, & k = 4 \cr
... \cr
\frac{1}{36}, & k = 12
\end{cases}
\end{equation}

It is important to note that there may not be some nice closed-form
expression for $p_V$ like that of (\ref{geompre}).  There was no such
form in (\ref{dicesum}), nor is there in our ALOHA example for
$p_{X_1}$ and $p_{X_2}$.

\subsection{Example:  Watts-Strogatz Random Graph Model}
\label{strogatz}

Random graph models are used to analyze many types of link systems, such
as power grids, social networks and even movie stars.  We saw our first
example in Section \ref{prefattach}, and here is another, a variation on
a famous model of that type, due to Duncan Watts and Steven Strogatz.

\subsubsection{The Model} 

We have a graph of n nodes, e.g. in which each node is a person).\footnote{The
word {\it graph} here doesn't mean ``graph'' in the sense of a picture.
Here we are using the computer science sense of the word, meaning a
system of vertices and edges.  It's common to call those {\it nodes} and
{\it links}.} Think of them as being linked in a circle---we're just
talking about relations here, not physical locations---so we already
have n links.  One can thus reach any node in the graph from any other,
by following the links of the circle.  (We'll assume all links are
bidirectional.)

We now randomly add k more links (k is thus a parameter of the model),
which will serve as ``shortcuts.''  There are$\binom{n}{2} = n(n-1)/2$
possible links between nodes, but remember, we already have n of those
in the graph, so there are only $n(n-1)/2 - n = n^2/2 - 3n/2$
possibilities left.  We'll be forming k new links, chosen at random from
those $n^2/2 - 3n/2$ possibilities.

Let M denote the number of links attached to a particular node, known
as the {\bf degree} of a node.  M is a random variable (we are choosing
the shortcut links randomly), so we can talk of its pmf, $p_M$, termed
the {\bf degree distribution} of M, which we'll calculate now.

Well, $p_M(r)$ is the probability that this node has r links.  Since the
node already had 2 links before the shortcuts were constructed, $p_M(r)$
is the probability that r-2 of the k shortcuts attach to this node.

This problem is similar in spirit to (though admittedly more difficult
to think about than) kings-and-hearts example of Section \ref{hearts}.
Other than the two neighboring links in the original circle and the
``link'' of a node to itself, there are n-3 possible shortcut links to
attach to our given node.  We're interested in the probability that r-2
of them are chosen, and that k-(r-2) are chosen from the other possible
links.  Thus our probability is:

\begin{equation}
p_M(r) =
\frac
{\binom{n-3}{r-2}
\binom{n^2/2-3n/2-(n-3)}{k-(r-2)}}
{\binom{n^2/2-3n/2}{k}} =
\frac
{\binom{n-3}{r-2}
\binom{n^2/2-5n/2+3}{k-(r-2)}}
{\binom{n^2/2-3n/2}{k}} 
\end{equation}

\subsubsection{Further Reading}

UCD professor Raissa D'Souza specializes in random graph models.  See
for instance Beyond Friendship: Modeling User activity Graphs on Social
Network-Based Gifting Applications, A. Nazir, A. Waagen, V.
Vijayaraghavan, C.-N. Chuah, R. M.  D'Souza, B. Krishnamurthy, {\it ACM
Internet Measurement Conference (IMC 2012)}, Nov 2012. 

\section{Proof of Chebychev's Inequality (optional section)}
\label{chebproof}

To prove (\ref{cheb}), let's first state and prove Markov's Inequality:
For any nonnegative random variable Y and positive constant d,

\begin{equation}
\label{mi}
P(Y \geq d) \leq \frac{EY}{d}
\end{equation}

To prove (\ref{mi}), let Z be the indicator random variable for the
event $Y \geq d$ (Section \ref{indicator}). 

Now note that

\begin{equation}
Y \geq d Z
\end{equation}

To see this, just think of a notebook, say with d = 3.  Then the
notebook might look like Table \ref{ygeqz}.

\begin{table}
\begin{center}
\vskip 0.5in

\begin{tabular}{|r|r|r|r|}
\hline
notebook line & Y & dZ & $Y \geq dZ$? \\ \hline
\hline
1 & 0.36 & 0 & yes \\ \hline
2 & 3.6 & 3 & yes \\ \hline
3 & 2.6 & 0 & yes \\ \hline
\end{tabular}

\end{center}
\caption{Illustration of Y and Z}
\label{ygeqz}
\end{table}

So

\begin{equation}
\label{eyez}
EY \geq d EZ
\end{equation}

(Again think of the notebook.  The long-run average in the Y column will
be $\geq$ the corresponding average for the dZ column.)

The right-hand side of (\ref{eyez}) is $d P(Y \geq d)$, so (\ref{mi})
follows.

Now to prove (\ref{cheb}), define

\begin{equation}
Y = (X-\mu)^2
\end{equation}

and set $d = c^2 \sigma^2$.  Then (\ref{mi}) says

\begin{equation}
\label{miout}
P [ (X-\mu)^2 \geq c^2 \sigma^2 ]
\leq 
\frac
{E [(X-\mu)^2]}
{c^2 \sigma^2}
\end{equation}

Since

\begin{equation}
(X-\mu)^2 \geq c^2 \sigma^2 \textrm{ if and only if }
|X-\mu| \geq c \sigma
\end{equation}

the left-hand side of (\ref{miout}) is the same as the left-hand side of
(\ref{cheb}).  The numerator of the right-hand size of (\ref{miout}) is
simply Var(X), i.e. $\sigma^2$, so we are done.

