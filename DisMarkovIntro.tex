\chapter{Introduction to Discrete Markov Chains}  
\label{dismarkov}

(From this point onward, we will be making heavy use of linear algebra.
The reader may find it helpful to review Appendix
\ref{chap:matrixreview}.)

Here we introduce Markov chains, a topic covered in much more detail in
Chapter \ref{mar}.  

The basic idea is that we have random variables $X_1, X_2, ...$, with
the index representing time.  Each one can take on any value in a given
set, called the {\bf state space}; $X_n$ is then the {\bf state} of the
system at time n.  The state space is assumed either finite or {\bf countably
infinite}.\footnote{The latter is a mathematical term meaning, in
essence, that it is possible to denote the space using integer
subscripts.  It can be shown that the set of all real numbers is not 
countably infinite, though perhaps surprisingly the set of all rational
numbers {\it is} countably infinite.} 

We sometimes also consider an initial state, $X_0$, which might be
modeled as either fixed or random.  However, this seldom comes into
play.

The key assumption is the {\bf Markov property}, which in rough terms
can be described as:

\begin{quote}
The probabilities of future states, given the present state and the past
states, depends only on the present state; the past is irrelevant.
\end{quote}

In formal terms:

\begin{equation}
\label{containshidden}
P(X_{t+1}=s_{t+1}|X_{t}=s_{t},X_{t-1}=s_{t-1},\ldots ,X_{0}=s_{0})=P(X_{t+1}=s_{t+1}|X_{t}=s_{t})
\end{equation}

Note that in (\ref{containshidden}), the two sides of the equation are
equal but their common value may depend on t.  We assume that this is
not the case; nondependence on t is known as {\bf
stationarity}.\footnote{Not to be confused with the notion of a
stationary distribution, coming below.} For instance, the probability of
going from state 2 to state 5 at time 29 is assumed to be the same as
the corresponding probability at time 333.

\section{Matrix Formulation}
\label{transmat}

We define $p_{ij}$ to be the probability of going from state i to state
j in one time step; note that this is a {\it conditional} probability,
i.e. $P(X_{n+1} = j ~|~ X_n = i)$.  These quantities form a matrix
P,\footnote{Unfortunately, we have some overloading of symbols here.
Both in this book and in the field in general, we usually use the letter
$P$ to denote this matrix, yet we continue to denote probabilities by P(
).  However, it will be clear from context which we mean.  The same is
true for our transition probabilities $p_{ij}$, which use a subscripted
letter $p$, which is also the case for probably mass functions.} whose
row i, column j element is $p_{ij}$, which is called the {\bf transition
matrix}.  Each row of P must sum to 1 (do you see why?).

For example, consider a three-state Markov chain with transition matrix

\begin{equation}
P =
\left (
\begin{array}{rrr}
\frac{1}{2} & 0 & \frac{1}{2} \\
\frac{1}{4} & \frac{1}{2} & \frac{1}{4} \\
1 & 0 & 0 \\
\end{array}
\right )
\end{equation}

This means that if we are now at state 1, the probabilities of going to
states 1, 2 and 3 are 1/2, 0 and 1/2, respectively.  Note that each
row's probabilities must sum to 1---after all, from any particular
state, we must go {\it somewhere}.

Actually, the m$^{th}$ power, $P^m$, of the transition matrix gives the
probabilities for m-step transitions.  In other words, the (i,j) element
of $P^m$ is $P(X_{t+m} = j | X_t = i)$.  This is clear for the case m =
2 (after which one can use mathematical induction), as follows.  

As usual, ``break big events down into small events.''  How can it
happen that $X_{t+2} = j$?  Well, break things down according to where
we might go first after leaving i.  We might go from i to 1, say, then
from 1 to j.  So,

\begin{equation}
P(X_{t+2} = j | X_t = i) = \sum_{k} p_{ik} ~ p_{kj}
\end{equation}

In view of the rule for multiplying matrices, the expression on the
right-hand side is simply the (i,j) element of $P^2$!

\section{Example:  Die Game}

As our first example of Markov chains, consider the following game.  One
repeatedly rolls a die, keeping a running total.  Each time the total
exceeds 10 (not equals 10), we receive one dollar, and continue playing,
resuming where we left off, mod 10.  Say for instance we have a total of
8, then roll a 5.  We receive a dollar, and now our total is 3.

It will simplify things if we assume that the player starts with one
free point.

This process clearly satisfies the Markov property, with our state being
our current total, 1, 2, ..., 10.  If our current total is 6, for
instance, then the probability that we next have a total of 9 is 1/6,
{\it regardless of what happened our previous rolls}.  We have $p_{25}$,
$p_{72}$ and so on all equal to 1/6, while for instance $p_{29} = 0$.
Here's the code to find the transition matrix P:

\begin{lstlisting}[numbers=left]
# 10 states, so 10x10 matrix
# since most elements will be 0s, set them all to 0 first, 
# then replace those that should be nonzero
p <- matrix(rep(0,100),nrow=10)  
onesixth <- 1/6
for (i in 1:10) {  # look at each row
   # since we are rolling a die, there are only 6 possible states we 
   # can go to from i, so check these
   for (j in 1:6) {  
      k <- i + j  # new total, but did we win?
      if (k > 10) k <- k - 10 
      p[i,k] <- onesixth
   }
}
\end{lstlisting}

Note that since we knew that many entries in the matrix would be zero, it
was easier just to make them all 0 first, and then fill in the nonzero
ones.

\section{Long-Run State Probabilities}

Let $N_{it}$ denote the number of times we have visited state i during
times 1,...,t.  For instance, in the die game, $N_{8,22}$ would be the
number of rolls among the first 22 that resulted in our having a total
of 8.

In typical applications we have that 

\begin{equation} 
\label{limntcpy} 
\pi_{i}=\lim_{t\rightarrow \infty}\frac{N_{it}}{t} 
\end{equation}
                 
exists for each state i, regardless of where we start.\footnote{A more
mathematically rigorous statement of (\ref{limntcpy}) would include the
qualifier ``with probability 1.''  What does does mean?

The infinite sequeunce $X_1, X_2,...$ is called a {\bf sample path}.  In
our notebook, we have a column for each $X_i$, so one row is one sample
path.  So, what (\ref{ limntcpy}) says, with the phrase ``with
probability 1'' added, is that the limit holds {\it for each row}. 
} 
Under a couple
more conditions,\footnote{Basically, we need the chain to not be {\bf
periodic}.  Consider a random walk, for instance:  We start at position
0 on the number line, at time 0.  The states are the integers.  (So,
this chain has an infinite state space.) At each time, we flip a coin to
determine whether to move right (heads) or left (tails) 1 unit.  A
little thought shows that if we start at 0, the only times we can return
to 0 are even-number times, i.e. $P(X_n = 0 ~| X_0 = 0)$ for all odd
numbers n.  This is a periodic chain.  By the way, (\ref{limntcpy})
turns out to be 0 for this chain.} we have the stronger result,

\begin{equation}
\label{limprob}
\lim_{t\rightarrow \infty }P(X_{t}=i)=\pi_{i}
\end{equation}

These quantities $\pi_i$ are typically the focus of analyses of Markov
chains.

We will use the symbol $\pi$ to name the column vector of all the
$\pi_i$:

\begin{equation}
\pi = (\pi_1, \pi_2, ...)'
\end{equation}

where $'$ means matrix transpose.

\subsection{Stationary Distribution}

The $\pi_i$ are called {\bf stationary probabilities}, because if the
initial state $X_0$ is a random variable with that distribution, then
all $X_i$ will have that distribution.  Here's why:

Using (\ref{limprob}), we have

\begin{eqnarray}
\label{stationeqns}
\pi_i 
&=& \lim_{n \rightarrow \infty} P(X_{n} = i)  \\
&=& \lim_{n \rightarrow \infty} \sum_k P(X_{n-1} = k) ~ p_{ki} \\ 
&=& \sum_k \pi_k ~ p_{ki} \label{lastpieqn} 
\end{eqnarray}

So, if $P(X_0 = i) = \pi_i$, then

\begin{eqnarray}
P(X_1 = i) 
&=& \sum_k P(X_{0} = k) ~ p_{ki} \\
&=& \sum_k \pi_k ~ p_{ki} \\
&=& \pi_i \label{piderive}
\end{eqnarray}

this last using (\ref{lastpieqn}).  So, if $X_0$ has distribution $\pi$,
then the same will be true for $X_1$, and contiuing in this manner we
see that $X_2, X_3,...$ will all have that distribution, thus
demostrating the claimed stationary property for $\pi$.  (This will be
illustrated more concretely in Section \ref{backstation}.)

Of course, (\ref{stationeqns}) holds for all states i.  So
in matrix terms, (\ref{stationeqns}) says

\begin{equation}
\label{pipip}
\pi' =  \pi' P
\end{equation}

\subsection{Calculation of $\pi$} 

Equation (\ref{pipip}) then shows us how to find the $\pi_i$, at least
in the case of finite state spaces, the subject of this section here, as
follows.

First, rewrite (\ref{pipip})

\begin{equation}
\label{iminuspi} 
(I-P') \pi = 0
\end{equation}

Here $I$ is the $n \times n$ identity matrix (for a chain with $n$
states), and again $'$ denotes matrix transpose.  

This equation has infinitely many solutions; if $\pi$ is a solution,
then so for example is $8 \pi$.  Moreover, the equation shows that the
matrix there, $I - P'$, cannot have an inverse; if it did, we could
multiply both sides by the inverse, and find that the unique solution is
$\pi = 0$, which can't be right.  This says in turn that the rows of $I
- P'$ are not linearly independent.  

The latter fact is quite important, for the following reason.  Recall
the close connection of matrix inverse and systems of linear equations.
(\ref{iminuspi}), a matrix equation, represents a system of n linear
equations in n unknowns, the latter being $\pi_1, \pi_2,.., \pi_n$.
So, the lack of linear independence of the rows of $I-P'$ means, 
in plain English, that at least one of those equations is redundant.

But we need n independent equations, and fortunately one is available:

\begin{equation}
\label{sum1}
\sum_{i}\pi_{i}=1
\end{equation}

Note that (\ref{sum1}) can be written as

\begin{equation}
\label{oprime1}
O' \pi = 1
\end{equation}

where $O$ is a vector of $n$ 1s.   Excellent, let's use it!

So, again, thinking of (\ref{iminuspi}) as a system of linear equations,
let's replace the last equation by (\ref{oprime1}).  Switching back to
the matrix view, that means that we replace the last row in the matrix
$I - P'$ in (\ref{iminuspi}) by $O'$, and correspondingly replace the
last element of the right-side vector by 1.  Now we have a nonzero
vector on the right side, and a full-rank (i.e.\ invertible) matrix on
the left side.  This is the basis for the following code, which we will
use for finding $\pi$.

\begin{lstlisting}[numbers=left]
findpi1 <- function(p) {
   n <- nrow(p)
   # find I-P'
   imp <- diag(n) - t(p)  
   # replace the last row of I-P' as discussed
   imp[n,] <- rep(1,n)
   # replace the corresponding element of the 
   # right side by (the scalar) 1
   rhs <- c(rep(0,n-1),1)
   # now use R's built-in solve()
   solve(imp,rhs)
}
\end{lstlisting}

\subsubsection{Example:  $\pi$ in Die Game}

Consider the die game example above.  Guess what!  Applying the above
code, all the $\pi_i$ turn out to be 1/10.  In retrospect, this should
be obvious.  If we were to draw the states 1 through 10 as a ring, with
1 following 10, it should be clear that all the states are completely
symmetric.

\subsubsection{Another Way to Find $\pi$}

Here is another way to compute $\pi$.  It is not commonly used, but {\bf
it will also help illustrate some of the concepts.}  

Suppose (\ref{limprob}) holds.  Recall that $P^m$ is the m-step
transition matrix, so that for instance row 1 of that matrix is the set
of probabilities of going from state 1 to the various states in m steps.
The same will be true for row 2 and so on.  Putting that together with
(\ref{limprob}), we have that 

\begin{equation}
\lim_{n \rightarrow \infty} P^n = \Pi
\end{equation}

where the $n \times n$ matrix $\Pi$ has each of its rows equal to
$\pi'$. 

We can use this to find $\pi$.  We take P to a large power m, and then
each of the rows will approximate $\pi$.  In fact, we can get an even
better appoximation by averaging the rows.

Moreover, we can save a lot of computation by noting the following.  Say
we want the 16$^{th}$ power of P.  We could set up a loop with 15
iterations, building up a product.  But actually we can do it with just
4 iterations.  We first square P, yielding $P^2$.  But then we square
{\it that}, yielding $P^4$.  Square twice more, yielding $P^8$ and
finally $P^{16}$.  This is especially fast on a GPU (graphics processing
unit).

\begin{lstlisting}
# finds stationary probabilities of a Markov chain using matrix powers

altfindpi <- function(p,k) {
   niters <- ceiling(log2(k))
   prd <- p
   for (i in 1:niters) {
      prd <- prd %*% prd
   }
   colMeans(prd) 
}
\end{lstlisting}

This approach has the advantage of being easy to parallelize, unlike
matrix inversion.

\section{Example:  3-Heads-in-a-Row Game}
\label{threeinrow}

How about the following game?  We keep tossing a coin until we get three
consecutive heads.  What is the expected value of the number of tosses
we need?

We can model this as a Markov chain with states 0, 1, 2 and 3, where
state i means that we have accumulated i consecutive heads so far. 
If we simply stop playing the game when we reach state 3, that state
would be known as an {\bf absorbing state}, one that we never leave.

We could proceed on this basis, but to keep things elementary, let's
just model the game as being played repeatedly, as in the die game
above.  You'll see that that will still allow us to answer the original
question.  Note that now that we are taking that approach, it will
suffice to have just three states, 0, 1 and 2; there is no state 3,
because as soon as we win, we immediately start a new game, in state 0.

\subsection{Markov Analysis}
\label{markovanalysis}

Clearly we have transition probabilities such as $p_{01}$, $p_{12}$,
$p_{10}$ and so on all equal to 1/2.  Note from state 2 we can only go
to state 0, so $p_{20} = 1$.

Here's the code below.  Of course, since R subscripts start at 1 instead
of 0, we must recode our states as 1, 2 an 3.

\label{codesnippets}
\begin{Verbatim}[fontsize=\relsize{-2}]
p <- matrix(rep(0,9),nrow=3)
p[1,1] <- 0.5
p[1,2] <- 0.5
p[2,3] <- 0.5
p[2,1] <- 0.5
p[3,1] <- 1
findpi1(p)
\end{Verbatim}

It turns out that

\begin{equation}
\label{pi57}
\pi = (0.5714286, 0.2857143, 0.1428571)
\end{equation}

So, in the long run, about 57.1\% of our tosses will be done while in state
0, 28.6\% while in state 1, and 14.3\% in state 2.

Now, look at that latter figure.  Of the tosses we do while in state 2,
half will be heads, so half will be wins.  In other words, about 0.071
of our tosses will be wins.  And THAT figure answers our original
question, through the following reasoning:

Think of, say, 10000 tosses.  There will be about 710 wins sprinkled
among those 10000 tosses.  Thus the average number of tosses between wins
will be about 10000/710 = 14.1.  In other words, the expected time until
we get three consecutive heads is about 14.1 tosses.

\subsection{Back to the word ``Stationary''}
\label{backstation}

Let's look at the term {\it stationary distribution} in the context of
this game.

At time 0, we haven't done any coin flips, so $X_0 = 0$.  But consider a
modified version of the game, in which at time 0 you are given
``credit'' or a ``head start.  For instance, you might be allowed to
start in state 1, meaning that you already have one consecutive head,
even though you actually haven't done any flips.

Now, suppose your head start is random, so that you start in state 0, 1
or 2 with certain probabilities.  And suppose those probabilities are
given by $\pi$!  In other words, you start in state 0, 1 or 2, with
probabilities 0.57, 0.29 and 0.14, as in (\ref{pi57}). What, then, will
be the distribution of $X_1$?

\begin{eqnarray}
p_{X_1}(i) &=& P(X_1 = i) \\
&=& \sum_{j=0}^2 P(X_0 = j) ~ P(X_1 = i ~| X_0 = j) \\
&=& \sum_{j=0}^2 \pi_j ~ p(j,i) \label{psubx1}
\end{eqnarray}

This is exactly what we saw in (\ref{piderive}) and the equations
leading up to it.  So we know that (\ref{psubx1}) will work out to
$\pi_i$.  In other words, $P(X_1 = 0) = 0.57$ and so on.
(Work it out numerically if you are in doubt.)  So, we see that

\begin{quote}
If $X_0$ has distribution $\pi$, then the same will be true for $X_1$.
We say that the distribution of the chain is {\it stationary}.
\end{quote}

\section{A Modified Notebook Analysis}
\label{markovnotebook}

Our previous notebook analysis (and most of our future ones, other than
for Markov chains), relied on imagining performing many independent
replications of the same experiment.  

\subsection{A Markov-Chain Notebook}

Consider Table \ref{alohanotebook}, for instance.  There our experiment
was to watch the network during epochs 1 and 2.  So, on the first line
of the notebook, we would watch the network during epochs 1 and 2 and
record the result.  On the second line, we watch a new, independent
replication of the network during epochs 1 and 2, and record the
results.

But instead of imagining a notebook recording infinitely many
replications of the two epochs, we could imagine watching just {\it one}
replication but over infinitely many epochs.  We'd watch the network
during epoch 1, epoch 2, epoch 3 and so on.  Now one line of the
notebook would record one epoch.  

For general Markov chains, each line would record one time step.  We
would have columns of the notebook labeled $n$ and $X_n$.  The reason
this approach would be natural is (\ref{limntcpy}).  In that context,
$\pi_i$ would be the long-run proportion of notebook lines in which $X_n
= i$.

\subsection{Example:  3-Heads-in-a-Row Game}

For instance, consider the 3-heads-in-a-row game.  Then (\ref{pi57})
says that about 57\% of the notebook lines would have a 0 in the $X_n$
column, with about 29\% and 14\% of the lines showing 1 and 2,
respectively.

Moreover, suppose we also have a notebook column labeled {\it Win}, with
an entry Yes in a certain line meaning, yes, that coin flip resulted in
a win, with a No entry meaning no win.  Then the mean time until a win,
which we found to be 14.1 above, would be described in notebook terms as
the long-run average number of lines between Yes entries in the {\it
Win} column.

\section{Simulation of Markov Chains}

Following up on Section \ref{markovnotebook}, recall that our previous
simulations have basically put into code form our notebook concept.  Our
simulations up until now have been based on the definition of
probability, which we had way back in Section \ref{probdefs}.  Our
simulation code modeled the notebook independent replications notion.
We can do a similar thing now, based on the ideas in Section
\ref{markovnotebook}.

In a time series kind of situation such as Markov chains, since
we are interested in long-run behavior in the sense of time, our
simulation is based on (\ref{limprob}).  In other words, we simulate
the evolution of $X_n$ as $n$ increases, and take long-run averages of
whatever is of interest.

Here is simulation code for the example in Section \ref{threeinrow},
caculating the approximate value of the long-run time between wins
(found to be about 14.1 by mathematical means above):

\begin{lstlisting}
# simulation of 3-in-a-row coin toss gaem

threeinrow <- function(ntimesteps) {
   consec <- 0  # number of consec Hs
   nwins <- 0  # number of wins
   wintimes <- 0  # total of times to win
   # startplay will be the time at which we started the current game
   startplay <- 0  # time step 0
   for (i in 1:ntimesteps) {
      if (toss() == 'H') {
         consec <- consec + 1 
         if (consec == 3) {
            nwins <- nwins + 1
            wintimes <- 
               wintimes + i - startplay
            consec <- 0
            startplay <- i
         }
      } else consec <- 0
   }
   wintimes / nwins
}

toss <- function() 
   if (runif(1) < 0.5) 'H' else 'T'
\end{lstlisting}l

\section{Example:  ALOHA}

Consider our old friend, the ALOHA network model.  (You
may wish to review the statement of the model in Section
\ref{basicprobcomp} before continuing.)  The key point in that system is
that it was ``memoryless,'' in that the probability of what happens at
time k+1 depends only on the state of the system at time k.

For instance, consider what might happen at time 6 if $X_5 = 2$.  Recall
that the latter means that at the end of epoch 5, both of our two
network nodes were active.  The possibilities for $X_6$ are then

\begin{itemize}

\item $X_6$ will be 2 again, with probability $p^2 + (1-p)^2$

\item $X_6$ will be 1, with probability $2p(1-p)$

\end{itemize}

The central point here is that the past history of the system---i.e. the
values of $X_1, X_2, X_3, \textrm{ and } X_4$---don't have any impact.
We can state that precisely:

\begin{quote}

The quantity

\begin{equation}
\label{condp6}
P(X_6 = j | X_1 = i_1, X_2 = i_2, X_3 = i_3, X_4 = i_4, X_5 = i)
\end{equation}

\noindent 
does not depend on $i_m, m = 1,...,4$.  Thus we can write (\ref{condp6})
simply as $P(X_6 = j | X_5 = i)$.  

\end{quote}

Furthermore, that probability is the same as $P(X_9 = j | X_8 = i)$ and
in general $P(X_{k+1} = j | X_k = i)$.  So, we do have a Markov chain.

Since this is a three-state chain, the $p_{ij}$ form a 3x3 matrix:

\begin{equation}
P = 
\left (
\begin{array}{ccc}
(1-q)^2 + 2q(1-q)p & 
   2q(1-q)(1-p) + 2q^2p(1-p) & 
   q^2 [p^2+(1-p)^2] \\
(1-q) p & 
   2qp(1-p) + (1-q)(1-p) & 
   q[p^2+(1-p)^2] \\
0 & 
   2p(1-p) & 
   p^2+(1-p)^2
\end{array}
\right )
\end{equation}

For instance, the element in row 0, column 2, $p_{02}$, is
$q^2[p^2+(1-p)^2]$, reflecting the fact that to go from state 0 to state
2 would require that both inactive nodes become active (which has
probability $q^2$, and then either both try to send or both refrain from
sending (probability $p^2+(1-p)^2$.

For the ALOHA example here, with p = 0.4 and q = 0.3, the solution is
$\pi_0 = 0.47$, $\pi_1 = 0.43$ and $\pi_2 = 0.10$.

So we know that in the long run, about 47\% of the epochs will have no
active nodes, 43\% will have one, and 10\% will have two.  From this we
see that the long-run average number of active nodes is

\begin{equation}
0 \cdot 0.47 + 1 \cdot 0.43 + 2 \cdot 0.10 = 0.63
\end{equation}

By the way, note that every row in a transition matrix must sum to 1.
(The probability that we go from state i to {\it somewhere} is 1, after
all, so we must have $\sum_j p_{ij} = 1$.)  That implies that we can
save some work in writing R code; the last column must be 1 minus the
others.  In our example above, we would write

\begin{Verbatim}[fontsize=\relsize{-2}]
transmat <- matrix(rep(0,9),nrow=3)
p1 <- 1 - p
q1 <- 1 - q
transmat[1,1] <- q1^2 + 2 * q * q1 *p
transmat[1,2] <- 2 * q * q1 * p1 + 2 * q^2 * p * p1
transmat[2,1] <- q1 * p
transmat[2,2] <- 2 * q * p * p1 + q1 * p1
transmat[3,1] <- 0
transmat[3,2] <- 2 * p * p1
transmat[,3] <- 1 - transmat[,1] - transmat[,2]
findpi1(transmat)
\end{Verbatim}

Note the vectorized addition and recycling (Section
\ref{improvingsimcode}). 

\section{Example:  Bus Ridership Problem}

Consider the bus ridership problem in Section \ref{busridership}.  Make
the same assumptions now, but add a new one:  There is a maximum
capacity of 20 passengers on the bus.

The random variables $L_i$, i = 1,2,3,... form a Markov chain.  Let's
look at some of the transition probabilities:

\begin{equation}
p_{00} = 0.5 
\end{equation}

\begin{equation}
p_{01} = 0.4 
\end{equation}

\begin{equation}
p_{11} = (1-0.2) \cdot 0.5 + 0.2 \cdot 0.4
\end{equation}

\begin{equation}
p_{20} = (0.2)^2 (0.5) = 0.02
\end{equation}

\begin{equation}
p_{20,20} = (0.8)^{20} (0.5+0.4+0.1) + 
\binom{20}{1}
(0.2)^1 (0.8)^{20-1} (0.4 + 0.1) + 
\binom{20}{2}
(0.2)^2 (0.8)^{18} (0.1)
\end{equation}

(Note that for clarity, there is a comma in $p_{20,20}$, as $p_{2020}$
would be confusing and in some other examples even ambiguous.  A comma
is not necessary in $p_{11}$, since there must be two subscripts; the 11
here can't be eleven.)

After finding the $\pi$ vector as above, we can find quantities such as
the long-run average number of passengers on the bus,

\begin{equation}
\label{avgonbus}
\sum_{i=0}^{20} \pi_i i
\end{equation}

We can also compute the long-run average number of would-be passengers who
fail to board the bus.  Denote by $A_i$ denote the number of passengers
on the bus as it {\it arrives} at stop i. The key point is that since
$A_i = L_{i-1}$, then (\ref{limntcpy}) and (\ref{limprob}) will give the
same result, no matter whether we look at the $L_j$ chain or the $A_j$ chain.

Now, armed with that knowledge, let $D_j$ denote the number of
disappointed people at stop j.  Then

\begin{equation}
ED_j = 1 \cdot P(D_j = 1) + 2 \cdot P(D_j = 2).
\end{equation}

That latter probability, for instance, is

\begin{eqnarray}
P(D_j = 2) 
&=& 
P(A_j = 20 \textrm{ and } B_j = 2 \textrm{ and } G_j = 0) \\ 
&=& P(A_j = 20) ~ P(B_j = 2) ~ P(G_j = 0 ~|~ A_j = 20) \\
&=& P(A_j = 20) \cdot 0.1 \cdot 0.8^{20}
\end{eqnarray}

where as before $B_j$ is the number who wish to board at stop j, and
$G_j$ is the number who get off the bus at that stop.  Using the same
reasoning, one can find $P(D_j = 1)$. (A number of cases to consider,
left as an exercise for the reader.)

Taking the limits as $j \rightarrow \infty$, we have the long-run
average number of disappointed customers on the left, and on the right,
the term $P(A_j = 20)$ goes to $\pi_{20}$, which we have and thus can
obtain the value regarding disappointed customers.

Let's find the long-run average number of customers who alight from the
bus.  This can be done by considering all the various cases, but
(\ref{adamsdiscrete}) really shortens our work:  

\begin{equation}
\label{eunan}
EG_n = \sum_{i=0}^{20} P(A_n = i) E(G_n | A_n = i)
\end{equation}

Given $A_n = i$, $G_n$ has a binomial distribution with i trials and
success probability 0.2, so 

\begin{equation}
E(G_n | A_n = i) = i \cdot 0.2
\end{equation}

So, the right-hand side of \ref{eunan} converges to

\begin{equation}
\sum_{i=0}^{20} \pi_i i \cdot 0.2
\end{equation}

In other words, the long-run average number alighting is 0.2 times
(\ref{avgonbus}).

\section{Example: an Inventory Model}

Consider the following simple inventory model.  A store
has 1 or 2 customers for a certain item each day, with probabilities v
and w (v+w = 1).  Each customer is allowed to buy only 1 item.

When the stock on hand reaches 0 on a day, it is replenished to r
items immediately after the store closes that day.

If at the start of a day the stock is only 1 item and 2 customers wish
to buy the item, only one customer will complete the purchase, and the
other customer will leave emptyhanded.

Let $X_n$ be the stock on hand at the end of day n ({\it after}
replenishment, if any).  Then $X_1, X_2,...$ form a Markov chain, with
state space 1,2,...,r.

The transition probabilities are easy to find.  Take $p_{21}$, for
instance.  If there is a stock of 2 items at the end of one day, what
is the (conditional) probability that there is only 1 item at the end of
the next day?  Well, for this to happen, there would have to be just 1
customer coming in, not 2, and that has probability v.  So, $p_{21} =
v$.  The same reasoning shows that $p_{2r} = w$.

Let's write a function {\bf inventory(v,w,r)} that returns the
$\pi$ vector for this Markov chain.  It will call {\bf findpi1()},
similarly to the two code snippets on page \pageref{codesnippets}.
For convenience, let's assume {\bf r} is at least 3.\footnote{If {\bf
r} is 2, then the expression 3:2 in the code evaluates to the vector
(3,2), which would not be what we want in this case.}

\begin{lstlisting}[numbers=left]
inventory <- function(v,w,r) {
   tm <- matrix(rep(0,r^2),nrow=r)
   for (i in 3:r) {
      tm[i,i-1] <- v
      tm[i,i-2] <- w
   }
   tm[2,1] <- v
   tm[2,r] <- w
   tm[1,r] <- 1
   findpi1(tm)
}
\end{lstlisting}

\section{Expected Hitting Times}

Consider an $n$-state Markov chain that is {\it irreducible},
meaning that it is possible to get from any state $i$ to any other state
$j$ in some number of steps.  Define the random variable $T_{ij}$ to be
the time needed to go from state $i$ to state $j$.  (Note that $T_{ii}$
is NOT 0, though it can be 1 if $p_{ii} > 0$.)

In many applications, we are interested in the mean time to reach a
certain state, given that we start at some specified state.  In other
words, we wish to calculate $ET_{ij}$.

The material in Section \ref{iterexp} is useful here.  In
(\ref{adamsdiscrete}), take $V = T_{ij}$ and take $U$ to be the first
state we visit after leaving state $i$ (which could be $i$ again).  Then

\begin{equation}
\label{etij}
E(T_{ij}) = \sum_{k} p_{ik} E(T_{ij} ~|~ U = k), ~~
1 \leq i,j \leq n
\end{equation}

Consider what happens if $U = k \neq j$.  Then we will have already used
up 1 time step, and {\it still} will need an average of $ET_{kj}$ more
steps to get to $j$.  In other words,

\begin{equation}
E(T_{ij} ~|~ U = k \neq j) = 1 + ET_{kj}
\end{equation}

Did you notice the Markov property being invoked?

On the other hand, the case $U = j$ is simple:

\begin{equation}
E(T_{ij} ~|~ U = j) = 1 
\end{equation}

This then implies that

\begin{equation}
\label{etij1}
E(T_{ij}) 
= 1 + \sum_{k \neq j} p_{ik} E(T_{kj}), ~~
1 \leq i,j \leq n
\end{equation}

We'll focus on the case $j = n$, i.e.\ look at how long it takes to get
to state $n$.  (We could of course do the same analysis for any other
destination state.)
Let $\eta_{i}$ denote $E(T_{in})$, and define  $\eta =
(\eta_1,\eta_2,...,\eta_{n-1})'$.  (Note that $\eta$ has only $n-1$
components!)  So,

\begin{equation}
\label{etij2}
\eta_{i}
= 1 + \sum_{k < n} p_{ik} \eta_k, ~~
1 \leq i,j \leq n
\end{equation}

Equation (\ref{etij2}) is a system of linear equations, which we can
write in matrix form.  Then the code to solve the system is
(remember, we have only an $(n-1) \times (n-1)$ system)

\begin{lstlisting}                                                              
findeta <- function(p) {                                                        
   n <- nrow(p)                                                                 
   q <- diag(n) - p                                                             
   q <- q[1:(n-1),1:(n-1)]                                                      
   ones <- rep(1,n-1)                                                           
   solve(q,ones)                                                                
}                                                                               
\end{lstlisting}                        

