\chapter{Relations \underline{Among} Variables}
\label{chap:among} 

{\it It is a very sad thing that nowadays there is so little useless
information}---Oscar Wilde, famous 19th century writer

Unlike the case of regression analysis, where the response/dependent
variable plays a central role, we are now interested in symmetric
relations among several variables.  Often our goal is {\bf dimension
reduction}, meaning compressing our data into just a few important
variables.

Dimension reduction ties in to the Oscar Wilde quote above, which is a
complaint that there is too {\it much} information of the use{\it ful}
variety.  We are concerned here with reducing the complexity of that
information to a more manageable, simple set of variables.

Here we cover two of the most widely-used methods, {\bf principal
components analysis} for continuous variables, and the {\bf log-linear
model} for the discrete case.  We also introduce {\bf clustering}.

\section{Principal Components Analysis (PCA)}
\label{pca}

Consider a random vector $X = (X_1,X_2)'$.  Suppose the two
components of X are highly correlated with each other.  
Then for some constants c and d,

\begin{equation}
\label{onedim}
X_2 \approx c + d X_1
\end{equation}

Then in a sense there is really just one random variable here, as the
second is nearly equal to some linear combination of the first.  The
second provides us with almost no new information, once we have the
first.

In other words, even though the vector X roams in two-dimensional space,
it usually sticks close to a one-dimensional object, namely the line
(\ref{onedim}).  We saw a graph illustrating this in our chapter on
multivariate distributions, page \pageref{rho2}.

In general, consider a k-component random vector 

\begin{equation}
X = (X_1,...,X_k)'
\end{equation}

We again wish to investigate whether just a few, say w, of the $X_i$ tell
almost the whole story, i.e. whether most $X_j$ can be expressed
approximately as linear combinations of these few $X_i$.  In other
words, even though X is k-dimensional, it tends to stick close to some
w-dimensional subspace.

Note that although (\ref{onedim}) is phrased in prediction terms, we are
not (or more accurately, not necessarily) interested in prediction here.
We have not designated one of the $X^{(i)}$ to be a response variable
and the rest to be predictors.

Once again, the Principle of Parsimony is key.  If we have, say, 20 or
30 variables, it would be nice if we could reduce that to, for example,
three or four.  This may be easier to understand and work with, albeit
with the complication that our new variables would be linear
combinations of the old ones.

\subsection{How to Calculate Them}

Here's how it works.  Let $\Sigma$ denote the covariance matrix of X.
The theory of linear algebra says that since $\Sigma$ is a symmetric
matrix, it is diagonalizable, i.e. there is a real matrix Q for which 

\begin{equation}
\label{qsigq}
Q' \Sigma Q = D
\end{equation}

where D is a diagonal matrix.  (A related approach is {\bf singular
value decomposition}.) The columns $C_i$ of Q are the eigenvectors of
$\Sigma$, and it turns out that they are orthogonal to each other, i.e.
their dot product is 0.

Let 

\begin{equation}
W_i = C_i'X, ~ i = 1,...,k
\end{equation}

so that the $W_i$ are scalar random variables, and set 

\begin{equation}
W = (W_1,...,W_k)'
\end{equation}

Then

\begin{equation}
W = Q' X
\end{equation}

Now, use the material on covariance matrices from our chapter on
random vectors, page \pageref{covawaprime},

\begin{equation}
Cov(W) = Cov(Q'X) = Q' Cov(X) Q = D ~~ \textrm{(from (\ref{qsigq}))}
\end{equation}

Note too that if X has a multivariate normal distribution (which we are
not assuming), then W does too.

Let's recap:

\begin{itemize}

\item We have created new random variables $W_i$ as linear combinations
of our original $X_j$.

\item The $W_i$ are uncorrelated.  Thus if in addition X has a
multivariate normal distribution, so that W does too, then the $W_i$
will be independent.

\item The variance of $W_i$ is given by the i$^{th}$ diagonal element of
D.

\end{itemize}

The $W_i$ are called the {\bf principal components} of the distribution
of X.

It is customary to relabel the $W_i$ so that $W_1$ has the largest
variance, $W_2$ has the second-largest, and so on.  We then choose those
$W_i$ that have the larger variances, and discard the others, because
the latter, having small variances, are close to constant and thus carry
no information.  

Note that an alternate definition of the first principal component is a
value of $u$ that maximizes $u'X$ subject to $u$ having length 1.  The
second principal component maximizes  $u'X$ subject to $u$ having length
1 and subject to the second component being orthogonal to the first, and
so on.

All this will become clearer in the example below.

\subsection{Example:  Forest Cover Data}

Let's try using principal component analysis on the forest cover data
set we've looked at before.  There are 10 continuous
variables.\footnote{There are also many discrete ones.}

In my R run, the data set (not restricted to just two forest cover
types, but consisting only of the first 1000 observations) was in the
object {\bf f}.  Here are the call and the results:

\begin{Verbatim}[fontsize=\relsize{-2}]
> prc <- prcomp(f[,1:10])
> summary(prc)
Importance of components:
                            PC1      PC2      PC3      PC4      PC5 PC6
Standard deviation     1812.394 1613.287 1.89e+02 1.10e+02 96.93455 30.16789
Proportion of Variance    0.552    0.438 6.01e-03 2.04e-03  0.00158 0.00015
Cumulative Proportion     0.552    0.990 9.96e-01 9.98e-01  0.99968 0.99984
                            PC7      PC8 PC9  PC10
Standard deviation     25.95478 16.78595 4.2 0.783
Proportion of Variance  0.00011  0.00005 0.0 0.000
Cumulative Proportion   0.99995  1.00000 1.0 1.000
\end{Verbatim}

You can see from the variance values here that R has scaled the $W_i$ so
that their variances sum to 1.0.  (It has not done so for the standard
deviations, which are for the nonscaled variables.) This is fine, as we
are only interested in the variances relative to each other, i.e. saving
the principal components with the larger variances.

What we see here is that eight of the 10 principal components have very
small variances, i.e. are close to constant.  In other words, though we
have 10 variables $X_1,...,X_{10}$, there is really only two
variables' worth of information carried in them.  

So for example if we wish to predict forest cover type from these 10
variables, we should only use two of them.  We could use $W_1$ and
$W_2$, but for the sake of interpretability we stick to the original X
vector.  We can use any two of the $X_i$, though typically it would be
two that have large coefficients in the top two principal components..

The coefficients of the linear combinations which produce W from X, i.e.
the Q matrix, are available via {\bf prc\$rotation}.


\subsection{Scaling}

If your original variables range quite a bit in variance, you should
have {\bf prcomp()} scale them first, so they all have standard
deviation 1.\footnote{And mean 0, though this is irrelevant, as $\Sigma$
is all that matter.}  The argument name is {\bf scale}, of course.

Without scaling, the proportion-of-total-variance type of analysis
discussed above may be misleading, as large-variance variables may
dominate.

\subsection{Scope of Application}

PCA makes no assumptions about the data.  It is strictly an
exploratory/descriptive tool.

However, it should be noted that the motivation we presented for PCA at
the beginning of this chapter involved correlations among our original
variables.  This is further highlighted by the fact that the PCs are
calculated based on the covariance matrix of the data, which except for
scale is the same as the correlation matrix.

This in turn implies that each variable is at least {\it ordinal} in
nature, i.e. that it makes sense to speak of the impact of larger or
smaller values of a variable.  

Note, though, that an indicator random variable is inherently ordinal!
So, if you have a {\it categorical} variable, i.e. one that simply codes
what category an individual falls into (such as Democratic, Republican,
independent), then you can convert it to a set of indicator variables,
and potentially get some insight into the relation between this variable
and others.  

This can be especially valuable if, as is often the case, your data
consists of a mixture of ordinal and categorical variables.

\subsection{Example:  Turkish Teaching Evaluation Data}

This data, again from the UCI Machine Learning Repository, was
introduced in Section \ref{turk0}.  Let's try PCA on it:

\begin{lstlisting}
> tpca <- prcomp(turk,scale=T)
> summary(tpca)
Importance of components:
                          PC1    PC2     PC3     PC4     PC5     PC6
PC7
Standard deviation     4.8008 1.1296 0.98827 0.62725 0.59837 0.53828 0.50587
Proportion of Variance 0.7947 0.0440 0.03368 0.01357 0.01235 0.00999 0.00882
Cumulative Proportion  0.7947 0.8387 0.87242 0.88598 0.89833 0.90832 0.91714
                           PC8     PC9    PC10    PC11    PC12    PC13
PC14
Standard deviation     0.45182 0.42784 0.41517 0.37736 0.37161 0.36957 0.3450
Proportion of Variance 0.00704 0.00631 0.00594 0.00491 0.00476 0.00471 0.0041
Cumulative Proportion  0.92418 0.93050 0.93644 0.94135 0.94611 0.95082 0.9549
                          PC15    PC16    PC17    PC18    PC19    PC20
PC21
Standard deviation     0.34114 0.33792 0.33110 0.32507 0.31687 0.30867 0.3046
Proportion of Variance 0.00401 0.00394 0.00378 0.00364 0.00346 0.00329 0.0032
Cumulative Proportion  0.95894 0.96288 0.96666 0.97030 0.97376 0.97705 0.9802
                          PC22    PC23    PC24    PC25    PC26    PC27
PC28
Standard deviation     0.29083 0.29035 0.28363 0.27815 0.26602 0.26023 0.23621
Proportion of Variance 0.00292 0.00291 0.00277 0.00267 0.00244 0.00234 0.00192
Cumulative Proportion  0.98316 0.98607 0.98884 0.99151 0.99395 0.99629 0.99821
                          PC29
Standard deviation     0.22773
Proportion of Variance 0.00179
Cumulative Proportion  1.00000
\end{lstlisting}

This is remarkable---the first PC accounts for 79\% of the variance of
the set of 29 variables.  In other words, in spite of the survey asking
supposedly 29 different aspects of the course, they can be summarized
largely in just one variable.  Let's see what that variable is:

\begin{lstlisting}
> tpca$rotation[,1]
         Q1          Q2          Q3          Q4          Q5          Q6 
-0.16974120 -0.18551431 -0.18553930 -0.18283025 -0.18973563 -0.18635256 
         Q7          Q8          Q9         Q10         Q11         Q12 
-0.18730028 -0.18559928 -0.18344211 -0.19241585 -0.18388873 -0.18184118 
        Q13         Q14         Q15         Q16         Q17         Q18 
-0.19430111 -0.19462822 -0.19401115 -0.19457451 -0.18249389 -0.19320936 
        Q19         Q20         Q21         Q22         Q23         Q24 
-0.19412781 -0.19335127 -0.19232101 -0.19232914 -0.19554282 -0.19328500 
        Q25         Q26         Q27         Q28  difficulty 
-0.19203359 -0.19186433 -0.18751777 -0.18855570 -0.01712709 
\end{lstlisting}

This is even more remarkable.  Except for the ``difficulty'' variable,
all the Qi have about the same coefficients (the same {\bf loadings}).
In other words, just one question would have been enough, and it
wouldn't matter mcuh which one were used.

The second PC, though only accounting for 4\% of the total variation, is
still worth a look:

\begin{lstlisting}
> tpca$rotation[,2]
         Q1          Q2          Q3          Q4          Q5          Q6 
 0.32009850  0.22046468  0.11432028  0.23340347  0.20236372  0.19890471 
         Q7          Q8          Q9         Q10         Q11         Q12 
 0.24025046  0.24477543  0.13198060  0.19239207  0.11064523  0.20881773 
        Q13         Q14         Q15         Q16         Q17         Q18 
-0.09943140 -0.15193169 -0.15089563 -0.03494282 -0.26163096 -0.11646066 
        Q19         Q20         Q21         Q22         Q23         Q24 
-0.14424468 -0.18729978 -0.21208705 -0.21650494 -0.09349599 -0.05372049 
        Q25         Q26         Q27         Q28  difficulty 
-0.20342350 -0.10790888 -0.05928032 -0.20370705 -0.27672177 
\end{lstlisting}

Here the ``difficulty'' variable now shows up, and some of the Qi become
unimportant.

\section{Log-Linear Models}
\label{loglin}

Here we discuss a procedure which is something of an analog of principal
components for discrete variables, especially {\it nominal} ones, i.e.
variables we would convert to dummies in a regression setting.

\subsection{The Setting}

Let's consider a variation on the software engineering example in
Section \ref{examples}.  Assume we have the factors, IDE, Language and
Education.  One change---{\bf of extreme importance}---is that we will
now assume that these factors are {\bf random}.  What does this mean?

In the original example described in Section \ref{examples}, programmers
were {\it assigned} to languages, and in our extensions of that example,
we continued to assume this.  Thus for example the number of programmers
who use an IDE and program in Java was fixed; if we repeated the
experiment, that number would stay the same.  If we were sampling from
some programmer population, our new sample would have new programmers,
but the number using and IDE and Java would be the same as before, as
our study procedure specifies this.

By contrast, let's now assume that we simply sample programmers at
random, and ask them whether they prefer to use an IDE or not, and which
language they prefer.\footnote{Other sampling schemes are possible too.}
Then for example the number of programmers who prefer to use an IDE and
program in Java will be random, not fixed; if we repeat the experiment,
we will get a different count.

Suppose we now wish to investigate relations between the factors.
Are choice of platform and language related to education, for instance?

\subsection{The Data}

Denote our three factors by $X^{(s)}$, s = 1,2,3.\footnote{All the
examples here will have three factors, but there can be any number.}
Here $X^{(1)}$, IDE, will take on the values 1 and 2 instead of 1 and 0
as before, 1 meaning that the programmer prefers to use an IDE, and 2
meaning not so.  $X^{(3)}$, Education, changes this way too, and
$X^{(2)}$ will take on the values 1 for C++, 2 for Java and 3 for C.  

Note that we no longer use indicator variables.  Indeed, the log-linear
model is in contrast to PCA, which as noted above tacitly assumes
ordinal variables.  Here we are working with strictly categorical
variables, whose values are merely labels.  We could, for example,
relabel $X^{(2)}$ to have the value 1 for Java, 2 for C++ and 3 for C,
and yet we would still have the same results (though the results would
be relabeled too).

Let $X_r^{(s)}$ denote the value of $X^{(s)}$ for the r$^{th}$
programmer in our sample, r = 1,2,...,n.  Our data are the counts

\begin{equation}
N_{ijk} = \textrm{number of r such that } X_r^{(1)} = i, X_r^{(2)} = j
\textrm{ and } X_r^{(3)} = k
\end{equation}

For instance, if we sample 100 programmers, our data might look like
this:

\begin{Verbatim}[fontsize=\relsize{-2}]
prefers to use IDE:

                   Bachelor's or less     Master's or more
           C++                     18                   15
          Java                     22                   10
             C                      6                    4
\end{Verbatim}

\begin{Verbatim}[fontsize=\relsize{-2}]
prefers not to use IDE:

                   Bachelor's or less     Master's or more
           C++                      7                    4
          Java                      6                    2
             C                      3                    3
\end{Verbatim}

So for example $N_{122} = 10$ and $N_{212} = 4$.

Here we have a three-dimensional {\bf contingency table}.  Each $N_{ijk}$
value is a {\bf cell} in the table.

\subsection{The Models}

Let $p_{ijk}$ be the population probability of a randomly-chosen
programmer falling into cell ijk, i.e.

\begin{equation}
p_{ijk} 
= P \left ( X^{(1)} = i \textrm{ and } X^{(2)} = j \textrm{ and } X^{(3)} = k \right ) 
= E(N_{ijk}) / n
\end{equation}

As mentioned, we are interested in relations among the factors, in the
form of independence, both full and partial.  Indeed, it is common for
an analyst to fit successively more refined models to the data, each
assuming a more complex dependence structure than the last.  This will
unfold in detail below.

Consider first the model that assumes full independence:

\begin{eqnarray}
p_{ijk} &=&
P \left ( X^{(1)} = i \textrm{ and } X^{(2)} = j \textrm{ and } X^{(3)} = k \right ) \\ 
&=& P \left ( X^{(1)} = i \right )
\cdot P \left ( X^{(2)} = j \right )
\cdot P \left ( X^{(3)} = k \right ) 
\label{fullind}
\end{eqnarray}

Taking logs of both sides in (\ref{fullind}), we see that independence
of the three factors is equivalent to saying 

\begin{equation}
\label{abc}
\log(p_{ijk}) = a_i + b_j + c_k
\end{equation}

for some numbers $a_i$, $b_j$ and $c_j$; e.g.

\begin{equation}
b_2 = \log[P(X^{(2)} = 2)]
\end{equation}

% The numbers must be nonpositive, and since
% 
% \begin{equation}
% \sum_{m} P(X^{(s)} = m) = 1
% \end{equation}
% 
% we must have, for instance,
% 
% \begin{equation}
% \label{sum1}
% \sum_{g=1}^2 \exp(c_g) = 1
% \end{equation}

The point is that (\ref{abc}) looks like our no-interaction linear
regression models, whose analog of noninteraction here is independence
of our variables.  On the other hand, if we assume instead that
Education is independent of IDE and Language but that IDE and Language
are not independent of each other, our model would include an i-j
interaction term, as follows.\footnote{In order to simplify the
discussion below, we will often write i as a shorthand for $X^{(i)}$ and
so on.}

We would have

\begin{eqnarray}
p_{ijk} &=& 
        P \left ( X^{(1)} = i \textrm{ and } X^{(2)} = j \right ) \cdot 
        P \left ( X^{(3)} = k \right ) 
\end{eqnarray}

so we would set 

\begin{equation}
\log(p_{ijk}) =
a_{ij} + b_k \label{semiindep}
\end{equation}

Most formal models rewrite this as

\begin{equation}
\label{anova}
a_{ij} = u + v_i + w_j + r_{ij}
\end{equation}

Here we have written $P \left ( X^{(1)} = i \textrm{ and } X^{(2)} = j
\right )$ as a sum of an ``overall effect'' u, ``main effects'' $v_i$
and $w_j$, and ``interaction effects,'' $r_{ij}$, again analogous to
linear regression.  

Note, though, that this actually gives us too many parameters.  For the
i-j part of the model, we have 2 $\times$ 3 = 6 actual probabilities,
but 1 + 2 + 3 +  2 $\times$ 3 = 12 parameters (1 for u, 2 for the $v_i$
and so on).  So the model needs constraints, e.g.

\begin{equation}
\sum v_i = 0
\end{equation}

This is similar to classical Analysis of Variance (ANOVA), not covered
in this book.  We will not state the constraints below, but they are
used in most if not all software packages for the log-linear model..

Another possible model would have IDE and Language conditionally
independent, given Education, meaning that at any level of education, a
programmer's preference to use IDE or not, and his choice of programming
language, are not related.  We'd write the model this way:

\begin{eqnarray}
p_{ijk} &=& 
   P \left ( X^{(1)} = i \textrm{ and } 
             X^{(2)} = j \textrm{ and } 
             X^{(3)} = k \right ) \\
&=& P \left ( X^{(1)} = i \textrm{ and } X^{(2)} = j ~|~ X^{(3)} = k \right)
    \cdot P \left ( X^{(3)} = k \right ) \\
&=& P \left ( X^{(1)} = i ~|~ X^{(3)} = k \right ) \cdot
    P \left ( X^{(2)} = j ~|~ X^{(3)} = k \right) \cdot
    P \left ( X^{(3)} = k \right ) 
\label{liketrickcoin}
\end{eqnarray} 

and thus set

\begin{equation}
\log(p_{ijk}) =
= u + a_i + f_{ik} + b_j + h_{jk} + c_k 
\label{semiindep2}
\end{equation}

\subsection{Interpretation of Parameters}

Note carefully that the type of independence in (\ref{semiindep2}) has a
quite different interpretation than that in (\ref{semiindep}).
Actually, our old trick coin example, Section \ref{oldtrick}, was like
(\ref{liketrickcoin}); given the choice of coin, the $B_i$ were
independent.  On the other hand, among people, height and weight are
correlated but they are presumably independent of preference for flavor
of ice cream, a situation like (\ref{semiindep}).

This is an excellent example of what can go wrong with mindless use of
packaged software.  Log-linear analysis is all about distinguishing
between various kinds of dependence.  Without an understanding of how
the models reflect this, the software can produce highly misleading
results.

So, pay close attention to which interactions are in the model, and
which are not.  In (\ref{semiindep}) we see a two-factor interaction
between the i and j factors, but no interaction with k.  So i and j
are being modeled as completely independent of k, though not with each
other.  On the other hand, in (\ref{semiindep2}), the i and j factors have
interactions with k, but not with each other.

Now consider the model

\begin{equation}
\label{harder}
\log(p_{ijk}) =
u + a_i + f_{ik} + b_j + h_{jk} + l_{ij} + c_k 
\end{equation}

Here things are a little harder to interpret.  Given k, i and j are no
longer modeled as independent.  However, the degree of ``correlation''
between i and j, given k, is modeled as being independent of k.  In
other words, the strength of the relation between i and j is the same,
for all levels of k.

If we had included an $m_{ijk}$ term---which would now make the model
{\bf full} or {\bf saturated}---then it may be possible that i and j are
highly related for some values of k, and less related for others.

Clearly, the more variables we have, and the higher the order of
interactions we include, the harder it is to interpret the model

\subsection{Parameter Estimation}

Remember, whenever we have parametric models, the statistician's ``Swiss
army knife'' is Maximum Likelihood Estimation (MLE, Section \ref{mle}).
That is what is most often used in the case of log-linear models.

How, then, do we compute the likelihood of our data, the $N_{ijk}$?
It's actually quite straightforward, because the $N_{ijk}$ have a
multinomial distribution (Section \ref{multinompmfsection}).  Then the
likelihood function is

\begin{equation}
\label{like}
L =
\frac
{n!}
{\Pi_{i,j,k} N_{ijk}!}
p_{ijk}^{N_{ijk}}
\end{equation}

We then write the $p_{ijk}$ in terms of our model parameters.
Take for example (\ref{semiindep2}), where we write

\begin{equation}
\label{xxx}
p_{ijk} 
= e^{u + v_i + w_j + r_{ij} + c_k} 
\end{equation}

We then substitute (\ref{xxx}) in (\ref{like}), and maximize the latter
with respect to the $a_i$, $b_j$, $d_{ij}$ and $c_k$, subject to
constraints as mentioned earlier.

The maximization may be messy.  But certain cases have been worked out
in closed form, and in any case today one would typically do the
computation by computer.  In R, for example, there is the {\bf loglin()}
function for this purpose, illustrated below.

Unfortunately, most books and software packages for the log-linear model
put almost all of their focus on significance testing, rather than point
estimation and confidence intervals.  In the popular {\bf loglin()}
package, for instance, the parameter estimates, e.g. $\hat{v}_i$ above,
are not even reported unless the user requests them.  Even then, no
standard errors are reported.  This is counter to the universally
recognized---though unfortunately widely ignored---point that
significance testing can be quite misleading, especially in large samples
(Section \ref{whatswrong}).

The tests themselves assess fit.  For instance, say we fit the model
(\ref{semiindep2}).  The program estimates parameters using maximum
likelihood, and then tests whether the model is ``correct.''  The actual
test will either be chi-squared (Section \ref{chisqgof}) or the
Likelihood Ratio Test (related to MLEs); both test statistics have
chi-square distributions under $H_0$.  We can call {\bf pchisq()} to
find the p-value.  Again, note that $H_0$ is that the specified model is
correct.

\subsection{Example:  Hair, Eye Color}

Here we will look at R's built-in data set {\bf HairEyeColor}.  There
are variables Hair, Eye and Sex, with 4 levels for hair color and 4 for
eye color, over 592 people.  Type {\bf ?HairEyeColor} to see the
details.

This is actually an example from the online help for the function
{\bf loglin()}, though we'll look at a different model in terms of
interactions.

\subsubsection{The loglin() Function}

We'll use the built-in R function {\bf loglin()}, whose input data must
be of class {\bf "table"}.  Let's see first how the latter works.

Say we have two variables, the first having levels 1 and 2, and the
second having levels 1, 2 and 3.  Suppose our data frame {\bf d} is

\begin{Verbatim}[fontsize=\relsize{-2}]
> d
  V1 V2
1  1  3
2  2  3
3  2  2
4  1  1
5  1  2
\end{Verbatim}

The first person (or other entity) in our dataset has $X^{(1)} = 1$, 
$X^{(2)} = 3$ and so on.  The function {\bf table()} does what its name
implies:  It tabulates the counts in each cell:

\begin{lstlisting}
> table(d)
   V2
V1  1 2 3
  1 1 1 1
  2 0 1 1
\end{lstlisting}

This says there was one instance in which $X^{(1)} = 1, X^{(2)} = 3$
etc., but no instances of $X^{(1)} = 2, X^{(2)} = 1$. 

So, our data is input as an object of type {\bf "table"}, specified in
the argument {\bf table}.  

Our model is input via the argument {\bf margin}, which is an R list of
vectors.  For instance {\bf c(1,3)} specifies an interaction between
variables 1 and 3, and {\bf c(1,2,3)} means a three-way interaction.
Once a higher-order interaction is specified, we need not specify its
lower-order ``subset.''  If, say, we specify {\bf c(2,5,6)}, we need not
specify {\bf c(2,6)}.  On the other hand, if a variable m is involved in
no interactions, we need to specify it as {\bf c(m)}, in order to keep
its individual (i.e. non-interaction) effect in the model.

The model used is actually for the cell counts, not the cell
probabilities.  Thus the constant term, e.g. u in (\ref{harder}) is
smaller by an amount equal to the log of the total number of
observations in the table.

\subsection{Hair/Eye Color Analysis}

Let's get an overview of the data first:

\begin{lstlisting}
> HairEyeColor
, , Sex = Male

       Eye
Hair    Brown Blue Hazel Green
  Black    32   11    10     3
  Brown    53   50    25    15
  Red      10   10     7     7
  Blond     3   30     5     8

, , Sex = Female

       Eye
Hair    Brown Blue Hazel Green
  Black    36    9     5     2
  Brown    66   34    29    14
  Red      16    7     7     7
  Blond     4   64     5     8
\end{lstlisting}

Note that this is a 3-dimensional array, with Hair being rows, Eye being
columns, and Sex being layers.  The data above show, for instance, that
there are 25 men with brown hair and hazel eyes.  Let's check this:

\begin{lstlisting}
> HairEyeColor[2,3,1]
[1] 25
\end{lstlisting}

Let's fit a model (as noted, for the $N_{ijk}$ rather than the
$p_{ijk}$) in which hair and eye color are independent of gender, but
not with each other, i.e. the model (\ref{anova}):

\begin{lstlisting}
> fm <- loglin(HairEyeColor, list(c(1, 2),3),param=T,fit=T)
2 iterations: deviation 5.684342e-14 
> fm
$lrt
[1] 19.85656
       
$pearson    
[1] 19.56712

$df
[1] 15
  
$margin
$margin[[1]]
[1] "Hair" "Eye"

$margin[[2]]
[1] "Sex"

$fit
, , Sex = Female

       Eye
Hair        Brown      Blue     Hazel     Green
  Black 35.952703 10.574324  7.930743  2.643581
  Brown 62.917230 44.412162 28.550676 15.332770
  Red   13.746622  8.988176  7.402027  7.402027
  Blond  3.701014 49.699324  5.287162  8.459459

     
$param      
$param$`(Intercept)`
[1] 2.494748

$param$Hair
     Black      Brown        Red      Blond 
-0.3063649  0.9520081 -0.3471908 -0.2984523 
  
$param$Eye
     Brown       Blue      Hazel      Green 
 0.3611125  0.5112173 -0.2798778 -0.5924520 
          
$param$Sex
      Male     Female 
-0.0574957  0.0574957 
          
$param$Hair.Eye
       Eye
Hair         Brown       Blue      Hazel      Green
  Black  0.9752132 -0.3986671  0.1047460 -0.6812921
  Brown  0.2764560 -0.2219556  0.1273068 -0.1818072
  Red    0.0546279 -0.5203601  0.0765790  0.3891532
  Blond -1.3062970  1.1409828 -0.3086318  0.4739461
\end{lstlisting}

The Likelihood Ratio Test, which has a chi-square distribution with 16
degrees of freedom under $H_0$ here, is not rejected:

\begin{lstlisting}
> 1 - pchisq(19.85656,16)
[1] 0.2267507
\end{lstlisting}

The p-value here is about 0.23.  So, if you allow your analyses to be
dictated by tests, you would accept the above model (which is plausible
anyway).  But even if you accept the hypothesis, you should still be
interested in the magnitudes of the effects, shown above.  Of course,
for them to make sense, you need to exponentiate them back into
probabilities, a fair amount of work.

At any rate, one generally entertains several different models for a
data set, in much the same way as one considers several different sets
of predictor variables in a regression setting.  If one does not use
significance testing for this, one goes through a less structured, but
hopefully more fruitful, process of comparing numerical model results.

Let's again consider the case of brown-haired, hazel-eyed men.  The
model fit is  

\begin{equation}
\label{howitworks}
\exp{(
\hat{u} + \hat{v}_2 + \hat{w}_3 + \hat{r}_{23}
)}
=
\exp(2.494748 + 0.9520081 - 0.279877 - 0.0574957 + 0.1273068) = 25.44934
\end{equation}

Actually, we could have gotten this directly, from

\begin{lstlisting}
> fm$fit[2,3,1]
[1] 25.44932  
\end{lstlisting}

(there is a bit of roundoff error above), but you should look at
(\ref{howitworks}) to make sure you understand what this number 25.44932
means.  It is the estimate of $E(N_{231})$ under our hypothesized model,
and (\ref{howitworks}) shows how this breaks down with respect to main
effects and interactions.

In particular, the interaction term, 0.1273068, is rather small compared
to most of the other numbers in the sum.  This suggests that although
there is some relation between hair and eye color, the relation is not
strong.  Of course, we'd have to browse through the other interaction
values to make a stronger statement.

\subsection{Obtaining Standard Errors}

It was mentioned above that the output of {\bf loglin()}, being
significance test-oriented, does not report standard errors for the
coefficients.  However, there is a ``trick'' one can use to get them, as
follows.\footnote{At the expense of computation, one can generally use
the {\bf bootstrap} for obtaining standard errors.  See Section
\ref{boot}.}

As noted earlier, our cell counts follow a multinomial distribution.
But there is another model, in which the cell counts are independent,
each one following a Poisson distribution.  (Note carefully that this is
not saying that the factors are independent.) The ``trick'' is that it
turns out that both models, multinomial and Poisson, yield the same
MLEs.  Thus even if we assume the multinomial model, we can run our
analyses as if it follows the Poisson model.

The remaining point is to note that R's {\bf glm()} function, used
before for logistic regression by setting {\bf family=binomial}, can be
used here with {\bf family=poisson}.  Since {\bf glm()} output reports
standard errors, this workaround enables us to obtain standard errors
for log-linear analysis.


% \subsection{The Goal:  Parsimony Again}
% 
% Again, we'd like ``the simplest model possible, but not simpler.''  This 
% means a model with as much independence between factors as possible,
% subject to the model being accurate.
% 
% Classical log-linear model procedures do model selection by hypothesis
% testing, testing whether various interaction terms are 0.  The tests
% often parallel linear regression testing, with chi-square distributions
% arising instead of F-distributions.

\section{Clustering}

The {\bf clustering} problem is somewhat like the reverse of the
classification problem.  In the latter, we our original data to enable
ourselves to later input new data points and output class identities.
With clustering, we use our original data to hunt for classes
themselves.

Clustering is treated as merely a technique for exploration/description,
for browsing through the data.  It is not generally regarded as
``estimating'' anything (though we will return to this issue later).

\subsection{K-Means Clustering}

This is probably the oldest clustering method, and it is still in wide
use.

\subsubsection{The Algorithm}

The method itself is quite simple, using an iterative algorithm.  The
user specifies {\bf k}, the number of clusters he hopes to find, and at
any step during the iteration process, the current {\bf k} clusters are
summarized by their centroids.  (If we have m variables, then the
centroid of a group is the m-element vector of means of those variables
within this group.)  We iterate the following:

\begin{itemize}

\item [1.]  For each data point, i.e. each row of our data matrix,
determine which centroid this point is closest to.

\item [2.]  Add this data point to the group corresponding to that
centroid.

\item [3.]  After all data points are processed in this manner, update
the centroids to reflect the current group memberships.

\item [4.]  Next iteration.

\item ]5.]  Iterate until the centroids don't change much from one
iteration to the next.

\end{itemize}

\subsubsection{Example:  the Baseball Player Data}

We use the R function {\bf kmeans()} here.  Its simplest call form is

\begin{lstlisting}
kmeans(x,k)
\end{lstlisting}

where {\bf x} is our data matrix and {\bf k} is as above.

Let's try it on our baseball player data Height, Weight and Age, which
are in columns 4-6 of our data frame:

\begin{lstlisting}
> head(baseball)
             Name Team       Position Height Weight   Age PosCategory 
1   Adam_Donachie  BAL        Catcher     74    180 22.99     Catcher 
2       Paul_Bako  BAL        Catcher     74    215 34.69     Catcher 
3 Ramon_Hernandez  BAL        Catcher     72    210 30.78     Catcher 
4    Kevin_Millar  BAL  First_Baseman     72    210 35.43   Infielder 
5     Chris_Gomez  BAL  First_Baseman     73    188 35.71   Infielder 
6   Brian_Roberts  BAL Second_Baseman     69    176 29.39   Infielder 
kmb <- kmeans(baseball[,4:6],4)
\end{lstlisting}

The return value, which we've stored here in {\bf kmb}, is of class (of
course) {\bf "kmeans"}.  One of its components is the final centroids:

\begin{lstlisting}
> kmb$centers
    Height   Weight      Age
1 73.81267 202.6584 29.00882
2 72.45013 180.3181 27.90776
3 74.79147 221.7725 29.48867
4 76.30000 244.4571 29.04129
\end{lstlisting}

Cluster 2 seems interesting, consisting of shorter, lighter and younger
players, while Cluster 4 seems the opposite.  Clusters 1 and 3 are
similar except in weight.

Let's see if these clusters correlate to player position.  Note that
we've consolidated the original Position variable into PosCategory,
which consists of catchers, pitchers, infielders and outfielders.  Let's
take a look at the relation to the clusters.  The {\bf cluster}
component of the object returned by the function shows which cluster
each observations fell into.  Here's are table of the results:

\begin{lstlisting}
> kmb$size  # number of points in each cluster
[1] 363 371 211  70
> baseball$cls <- kmb$cluster  
> table(baseball$cls,baseball$PosCategory)
   
    Catcher Infielder Outfielder Pitcher
  1      31        69         74     201
  2      32        42         51     143
  3       9        84         57     107
  4       4        15         12      84
\end{lstlisting}

Clusters 1 and 2 were the largest overall, but there is an interesting
pattern here.  Catchers were almost entirely in Clusters 1 and 2, while
the other position categories, especially infielders, were more evenly
dispersed.

\subsection{Mixture Models}
\label{mixclust}

Even though k-means and other clustering methods don't claim to be
estimating anything---again, they are just intended as exploratory
tools---the basic motivation is that each cluster represents some
multivariate distribution, e.g. trivariate normal in the baseball
example above.  Let $q_i$ be the probability that an individual is in
cluster i.  Then the overall distribution has the form (\ref{hsum})---a
mixture!

So, we really are modeling the population by a mixture of
subpopulations, and the EM algorithm for mixtures (Section \ref{emalg})
can be used to estimate the centroids and the mixture weights $q_i$.  in
particular, the {\bf mixtools} library includes a function
{\bf mvnormalmixEM()} which does exactly this.

\subsection{Spectral Models}

The word {\it spectral} here refers in turn to {\it spectrum}, which in
linear algebra terms is the set of eigenvalues of a matrix.  It turns
out that this is useful for clustering.

The details are complex, and many variant algorithms exist, but here is
an overview:

Start with a {\bf similarity matrix} S, calculated from your original
data matrix X.  $S_{ij}$ is defined in terms of some distance between
rows i and j in X.  One then computes

\begin{equation}
L = I - D^{-1/2} S D^{-1/2}
\end{equation}

where D is a diagonal matrix whose (i,i) element is $\sum_{j} S_{ij}$.

One then finds an eigenvector v corresponding to the second-smallest
eigenvalue of L.  One then partitions the rows of X according to some
criterion involving v.

Then one partitions those partitions! The same method is used, and in
this way, the data points are chopped up into groups, our clusters.
This is an example of what is called {\bf hierarchical} clustering.

\subsection{Other R Functions}

The CRAN code repository includes a number of clustering packages.  See
the CRAN Task View: Cluster Analysis \& Finite Mixture
Models,\footnote{CRAN has a number of these ``task views'' on various
topics.} \url{http://cran.r-project.org/web/views/Cluster.html}.

\subsection{Further Reading}

UCD professor Catherine Yang is a specialist in clustering.  See for
instance Segmenting Customer Transactions Using a Pattern-Based
Clustering Approach, {\it Proceedings of the Thrid IEEE International
Conference on Data Mining}.

\section{Simpson's (Non-)Paradox}

Every serious user of statistics must keep in mind {\bf Simpson's
Paradox} at all times.  It's an example of ``what can go wrong'' in
multivariate studies.  Its central importance is reflected, for
instance, in an extensive entry in Wikipedia.

And yet...a very simple precaution will enable you to avoid the problem.

So, what is this paradox?  In short, the relation between variables X
and Y can be positive, conditional on every level of a third variable Z,
and yet be negative overall.  (Or vice versa, i.e. change from negative
to positive.)

As you will see below, it arguably is not really a paradox at all.
Instead, the real problem is a failure to examine the more important
variables before the less important ones.

\subsection{Example:  UC Berkeley Graduate Admission Data}

This is a very famous data set, much analyzed over the years, well known
partly because a lawsuit was involved.

\subsubsection{Overview}

The suit claimed that UC Berkeley graduate admissions practices
discriminated against women.  The plaintiffs pointed out that male
applicants had a 44\% acceptance rate, compared to only 35\% for women.

On the surface, the claims of discrimination seemed quite plausible.
Here X, called Admit, was being admitted to graduate school, and Y,
called Gender, was an indicator variable for being male.  X and Y
appeared to be positively related.  

However, things changed a lot when a third variable Z, called Dept, was
brought in.  This coded which department the applicant applied to.  Upon
closer inspection by UCB statistics professors, it was found that
conditional on Z, X and Y were actually \underline{negatively} related
for (almost) every value of Z.  In other words, it turned out that in
(almost) every department, the women applicants were actually being
slightly fared better than the men.\footnote{One department was an
exception, but the difference was small and possibly due to sampling
variation.  Thus most accounts treat this as an instance of Simpson's
Paradox, rather than ``almost'' an example.}

\subsubsection{Log-Linear Analysis}

Let's analyze this with a log-linear model.  As it happens, the data is
one of R's built-in data sets, {\bf UCBAdmissions} (which we will copy
to a new variable {\bf ucb}, for brevity):

\begin{lstlisting}
> ucb <- UCBAdmissions
\end{lstlisting}

As noted, our ``Z,'' is Dept; six departments were represented in the
data, i.e. six different graduate programs.  

We can easily determine how many applicants there were:

\begin{lstlisting}
> sum(ucb)
[1] 4526
\end{lstlisting}

Since the Admit and Gender variables had two levels each, there were $2
\times 2 \times 6 = 24$ different cells in the table.  In other words,
our log-linear analysis is modeling 24 different cell probabilities
(actually 23, since they must sum to 1.0).

So, even a saturated model would easily satisfy our rule of thumb
(Section \ref{thumb}) regarding overfitting, which recommended keeping
the number of parameters below the square root of the sample size.

Let's fit a model that includes all two-way interactions.\footnote{A
full model was fit, with similar patterns.}

\begin{lstlisting}
> llout <- loglin(ucb,list(1:2,c(1,3),2:3),param=T)
2 iterations: deviation 3.552714e-15 
> llout$param
$`(Intercept)`
[1] 4.802457

$Admit
  Admitted   Rejected 
-0.3212111  0.3212111 

$Gender
      Male     Female 
 0.3287569 -0.3287569 

$Dept
          A           B           C           D           E           F 
 0.15376258 -0.76516841  0.53972054  0.43021534 -0.02881353 -0.32971651 

$Admit.Gender
          Gender
Admit             Male      Female
  Admitted -0.02493703  0.02493703
  Rejected  0.02493703 -0.02493703

$Admit.Dept
          Dept
Admit
           A          B          C            D           E          F
 Admitted  0.6371804  0.6154772  0.005914624 -0.01010004 -0.2324371 -1.016035
 Rejected -0.6371804 -0.6154772 -0.005914624  0.01010004  0.2324371  1.016035

$Gender.Dept
        Dept
Gender           A         B          C          D          E          F
  Male    0.6954949  1.232906 -0.6370521 -0.2836477 -0.7020726 -0.3056282
  Female -0.6954949 -1.232906  0.6370521  0.2836477  0.7020726  0.3056282
\end{lstlisting}

Again, there is a lot here, but it's not hard to spot the salient
numbesr.  Let's look at the main effects first, labeled {\bf \$Admit}.
{\bf \$Gender} and {\bf \$Dept}, near the top of the output, reproduced
here for convenience:

\begin{lstlisting}
$Admit
  Admitted   Rejected 
-0.3212111  0.3212111 

$Gender
      Male     Female 
 0.3287569 -0.3287569 

$Dept
          A           B           C           D           E           F 
 0.15376258 -0.76516841  0.53972054  0.43021534 -0.02881353 -0.32971651 
\end{lstlisting}

One sees that the main effect for Gender, $\pm$0.3287569, is small
compared to the main effects for some of the departments.  So some
departments attract a lot of applicants, notably C, while others such as
F attract fewer than average.  If this varies by gender, then it's
important to take department into account in our analysis.



In view of that latter fact, in viewing the two-way interactions,
consider the interaction between Admit and Dept.  The Admit values were
strongly positive for Departments A and B, small for C and D, and
moderately to strongly negative for E and F.  In other words, A and B
were especially easy to get into, while E and F were particularly
difficult.

But now turn to the Gender.Dept interaction, showing the pattern of
which departments were applied to by which genders.  The men were extra
likely to apply to Departments A and B, while for the women it was C, D,
E and F.  In other words, men were more likely to apply to the easier
departments, while the women were more likely to apply to the difficult
ones!  No wonder the women seemed to be faring poorly overall in the
data presented in the lawsuit, which did not break things down by
department.

The Admit.Gender interaction says it all:  Everything else equal, the
women were slightly more \underline{likely} to be admitted than men.

\subsection{Toward Making It Simpson's NON-Paradox}

The above example shows that Simpson's Paradox really isn't a paradox at
all.  The so-called ``anomaly'' occurred simply because an omitted
variable (department) was a stronger factor than was one of the included
variables (gender).

Indeed, it can be argued that the ``paradox,'' i.e. the reversal of
signs, wouldn't occur if one used the correct ordering when bringing
factors into consideration, as follows.  Let's use the UCB data above to
make this concrete.

Recall the method of {\it forward stepwise regression} (Section
\ref{forwardbackward}).  The analog here might be start by considering
just two of the three variables at a time, before looking at
three-variable models.  For the UCB data, we would first look at Admit and
Gender, then look at Admit and Dept.  For example, we would fit the
latter model by running

\begin{lstlisting}
> loglin(ucb,list(c(1,3)),param=TRUE)
\end{lstlisting}

The results are not shown here, but they reveal that indeed Dept has a
much stronger association with Admit than does Gender.  We would thus
take Admit-Dept as our "first-step" model, and then add Gender to get
our second-step model (again, "step" as in the spirit of stepwise
regression).  That would produce the model we analyzed above, WITH
NO PARADOXICAL RESULT.


% 
% Here's the Admit and Gender analysis:
% 
% \begin{lstlisting}
% > loglin(ucb,list(c(1,2)),param=T)
% 2 iterations: deviation 2.273737e-13 
% $lrt
% [1] 2163.744
% 
% $pearson
% [1] 1919.241
% 
% $df
% [1] 20
% 
% $margin
% $margin[[1]]
% [1] "Admit"  "Gender"
% 
% 
% $param
% $param$`(Intercept)`
% [1] 5.176383
% 
% $param$Admit
%   Admitted   Rejected 
% -0.2626551  0.2626551 
% 
% $param$Gender
%       Male     Female 
%  0.2303337 -0.2303337 
% 
% $param$Admit.Gender
%           Gender
% Admit            Male     Female
%   Admitted  0.1525881 -0.1525881
%   Rejected -0.1525881  0.1525881
% \end{lstlisting}
% 
% And now the Admit and Dept analysis:
% 
% \begin{lstlisting}
% > loglin(ucb,list(c(1,3)),param=T)
% 2 iterations: deviation 0 
% $lrt
% [1] 1405.224
% 
% $pearson
% [1] 1201.405
% 
% $df
% [1] 12
% 
% $margin
% $margin[[1]]
% [1] "Admit" "Dept" 
% 
% 
% $param
% $param$`(Intercept)`
% [1] 5.049374
% 
% $param$Admit
%   Admitted   Rejected 
% -0.3257818  0.3257818 
% 
% $param$Dept
%          A          B          C          D          E          F 
%  0.3593440 -0.1004504  0.3398752  0.1846255 -0.2073381 -0.5760562 
% 
% $param$Admit.Dept
%           Dept
% Admit               A          B           C            D          E         F
%   Admitted  0.6225118  0.5972143  0.01793724 -0.006653238 -0.2189685 -1.012042
%   Rejected -0.6225118 -0.5972143 -0.01793724  0.006653238  0.2189685  1.012042
% \end{lstlisting}
% 
% It can immediately be seen that the second model generally has larger
% coefficients.  In other words, for admissions, department matters more
% than gender.
% 
% The point is that once we note that, our three-variable model seen
% earlier {\it avoids ``sign reversals''}.  


% \subsection{When to Watch for Simpson's}
% 
% If some statistical result seems counterintuitive, check for Simpson's
% Paradox.
% 
% A particularly dangerous situation is one in which a regression analysis
% is performed with a very large number of predictor variables.  In this
% setting, there are lots of potential ``X, Y and Z'' situations of the
% type described above.
% 
% None of this is to say one should reject any counterintuitive
% statistical result.  Many such results are genuine.  But do be ccareful.


% For example, say X, Y and Z are indicator variables.  We might have
% 
% Suppose each individual in a population either possesses or does not
% possess traits {\it A}, {\it  B} and {\it C}, and that we wish to
% predict trait {\it A}.  Let $\bar{A}$, $\bar{B}$ and $\bar{C}$ denote
% the situations in which the individual does not possess the given trait.
% Simpson's Paradox then describes a situation in which
% 
% \begin{equation}
% P(A|B) > P(A|\bar{B})
% \end{equation}
% 
% \noindent and yet
% 
% \begin{equation}
% P(A|B,C) < P(A|\bar{B},C)
% \end{equation}
% 
% In other words, the possession of trait $B$ seems to have a positive
% predictive power for $A$ by itself, but when in addition trait $C$ is
% held constant, the relation between $B$ and $A$ turns negative.
% 
% An example is given by Fabris and Freitas,\footnote{C.C. Fabris and A.A.
% Freitas. Discovering Surprising Patterns by Detecting Occurrences of
% Simpson's Paradox.  In {\it Research and Development in Intelligent
% Systems XVI (Proc. ES99, The 19th SGES Int. Conf. on Knowledge-Based
% Systems  and  Applied Artificial Intelligence)}, 148-160.
% Springer-Verlag, 1999 } concerning a classic study of tuberculosis
% mortality in 1910.  Here the attribute $A$ is mortality, $B$ is city
% (Richmond, with $\bar{B}$ being New York), and $C$ is race
% (African-American, with $\bar{C}$ being Caucasian).  In probability
% terms, the data show that (these of course are sample
% estimates)
% 
% \begin{itemize}
% 
% \item P(mortality $|$ Richmond) = 0.0022
% 
% \item P(mortality $|$ New York) = 0.0019
% 
% \item P(mortality $|$ Richmond, black) = 0.0033
% 
% \item P(mortality $|$ New York, black) = 0.0056
% 
% \item P(mortality $|$ Richmond, white) = 0.0016
% 
% \item P(mortality $|$ New York, white) = 0.0018
% 
% \end{itemize}
% 
% \noindent The data also show that
% 
% \begin{itemize}
% 
% \item P(black $|$ Richmond) = 0.37
% 
% \item P(black $|$ New York) = 0.002
% 
% \end{itemize}
% 
% \noindent a point which will become relevant below.
% 
% At first, New York looks like it did a better job than Richmond.
% However, once one accounts for race, we find that New York is actually
% worse than Richmond.  Why the reversal?  The answer stems from the fact
% that racial inequities being what they were at the time, blacks with the
% disease fared much worse than whites.  Richmond's population was 37\%
% black, proportionally far more than New York's 0.2\%.  So, Richmond's
% heavy concentration of blacks made its overall mortality rate look worse
% than New York's, even though things were actually much worse in New
% York.
% 
% But is this really a ``paradox''?   Closer consideration of this example
% reveals that the only reason this example (and others like it) is
% surprising is that the predictors were used in the wrong order.  One
% normally looks for predictors one at a time, first finding the best
% single predictor, then the best pair of predictors, and so on.  If this
% were done on the above data set, the first predictor variable chosen
% would be race, not city.  In other words, the sequence of analysis would
% look something like this: 
% 
% \begin{itemize}
% 
% \item P(mortality $|$ black) = 0.0048
% 
% \item P(mortality $|$ white) = 0.0018
% 
% \item P(mortality $|$ black, Richmond) = 0.0033
% 
% \item P(mortality $|$ black, New York) = 0.0056
% 
% \item P(mortality $|$ white, Richmond) = 0.0016
% 
% \item P(mortality $|$ white, New York) = 0.0018
% 
% \end{itemize}
% 
% The analyst would have seen that race is a better predictor than city,
% and thus would have chosen race as the best single predictor.  The
% analyst would then investigate the race/city predictor pair, and
% would never reach a point in which city alone were in the
% selected predictor set.  Thus no anomalies would arise.

